% Apendice
\chapter{Metaheurísticas híbridas y no híbridas}
\label{apendicec}

	Se buscó determinar diferencias en la calidad de los resultados de cada
una de las metaheurísticas hibridas (\emph{GAH}, \emph{NPSOH}, \emph{WPSOH},
\emph{NDEH}, \emph{SDEH}, \emph{BeeH}, \emph{AntH}) con respecto a sus
contrapartes no hibridadas (\emph{GA}, \emph{NPSO}, \emph{WPSO},
\emph{NDE}, \emph{SDE}, \emph{Bee}, \emph{Ant}).

	Los resultados analizados en esta sección fueron obtenidos a partir de la
ejecución de las diferentes metaheurísticas con la imagen \textbf{Lenna}
(figura \ref{fig:lenna}).

	No se incluyó el análisis de los resultados para el archivo de datos númericos
\textbf{Iris} (sección \ref{test:iris}), debido a que las conclusiones obtenidas
a partir del mismo eran similares a las conclusiones obtenidas con \textbf{Lenna}.

\section{Comparación del índice $DB$ y evaluaciones de la función de \emph{fitness}}

	Si la media de los valores del índice $DB$ de cada metaheurística híbrida
es estadísticamente diferente a la media de su contraparte no híbrida, entonces
existe una diferencia en la calidad de sus soluciones finales. Tomando como
premisa que a menores valores de índice $DB$, mejor calidad de partición,
entonces la metaheurística con menor índice $DB$ promedio será la mejor.

	En constraste, si son estadísticamente iguales, la diferencia puede yacer en
el rendimiento de la metaheurística. Si la media de la cantidad de evaluaciones
de la función de \emph{fitness} de cada metaheurística híbrida es estadísticamente
diferente a la media de su contraparte no híbrida, entonces existe una diferencia
en el rendimiento de las mismas. Tomando como premisa que mientras menos
evaluaciones de la función de \emph{fitness}, más rápido es el algoritmo, entonces
la metaheurística con menor cantidad de evaluaciones de la función de \emph{fitness}
será la mejor.

\section{\emph{GAH} vs. \emph{GA}}

	Se buscó determinar una diferencia en la calidad de los resultados de las
metaheurísticas \emph{GAH} (metaheurística híbrida con K-means) y
\emph{GA} (metaheurística no híbrida).

\subsection{Prueba de diferencia de medias para el índice $DB$}

    Para determinar diferencias en las calidades de las soluciones finales de las
metaheurísticas \emph{GAH} y \emph{GA}, se realizó una prueba de hipótesis de
diferencia de medias del índice $DB$:
\begin{itemize}
    \item \emph{Hipótesis nula}: $\bar{X}_{GAH} - \bar{X}_{GA} = 0$
    \item \emph{Hipótesis altenativa}: $\bar{X}_{GAH} - \bar{X}_{GA} \neq 0$
\end{itemize}
donde $\bar{X}_{GAH}$ y $\bar{X}_{GA}$ son las medias muestrales del índice
$DB$ de las metaheurísticas \emph{GAH} y \emph{GA} respectivamente.

    Para esto se utilizó la prueba $t$ de \emph{Welch} \cite{AB_0} la cual fue
elegida por no ser sensible a la varianza de los datos.

	Como se puede observar en el Resultado \ref{c-gas_img}, ambas medias son
estadísticamente iguales ($p-valor = 0.2913$). { \bf Esto significa que ambas
metaheurísticas obtienen soluciones finales con similar calidad}.

\begin{lstlisting}[float=h!, caption={Diferencia de Medias: Índice \emph{DB}}, label=c-gas_img]
> t.test(db ~ name)

	Welch Two Sample t-test

data:  db by name 
t = -1.0654, df = 55.552, p-value = 0.2913
alternative hypothesis: true difference in means is not equal to 0 
95 percent confidence interval:
 -0.025548237  0.007810306 
sample estimates:
   mean in group GAH mean in group GA
           0.7852724        0.7941414 
\end{lstlisting}


\subsection{Cantidad de evaluaciones de la función de \emph{fitness}}

	En la sección anterior, se demostró que ambas metaheurísticas obtienen
soluciones finales con similar calidad. Si no existe una diferencia en
rendimiento, entonces ambos algoritmos son iguales y cualquiera de ellos puede
ser un buen candidato para competir con las demás metaheurísticas.

	Como las metaheurísticas \emph{GAH} y \emph{GA} utilizan la misma función de
\emph{fitness} y esta es la función más costosa durante la ejecución de las mismas,
se buscó determinar una diferencia en la cantidad de evaluaciones de la función
de \emph{fitness} de las metaheurísticas. Para esto se realizó una prueba de
hipótesis de diferencia de medias:
\begin{itemize}
    \item \emph{Hipótesis nula}: $\bar{X}_{GAH} - \bar{X}_{GA} = 0$
    \item \emph{Hipótesis altenativa}: $\bar{X}_{GAH} - \bar{X}_{GA} \neq 0$
\end{itemize}
donde $\bar{X}_{GAH}$ y $\bar{X}_{GA}$ son las medias muestrales de la cantidad
de evaluaciones de la función de fitness de las metaheurísticas \emph{GAH} y
\emph{GA} respectivamente.

    Para esto se utilizó la prueba $t$ de \emph{Welch} \cite{AB_0} la cual fue
elegida por no ser sensible a la varianza de los datos.

	Como se puede observar en el Resultado \ref{c-gas_img1}, ambas medias son
estadísticamente iguales ($p-valor = 0.08436$). {\bf Esto significa que ambas
metaheurísticas tienen un rendimiento similar}. Sin embargo, \emph{GAH} tiene
más oportunidad de mejorar sus soluciones parciales al estar hibridado con el
algoritmo \emph{K-means} que su contraparte no hibridada \emph{GA}. {\bf Por lo
tanto, \emph{GAH} es una metaheurística más confiable}.

\begin{lstlisting}[float=h!, caption={Diferencia de medias: Cantidad de Evaluaciones}, label=c-gas_img1]
> t.test(eval ~ name)

	Welch Two Sample t-test

data:  eval by name 
t = 1.7571, df = 55.986, p-value = 0.08436
alternative hypothesis: true difference in means is not equal to 0 
95 percent confidence interval:
 -0.65687 10.03618 
sample estimates:
   mean in group GAH mean in group GA
            35.20690         30.51724 
\end{lstlisting}

\section{\emph{NPSOH} vs. \emph{NPSO}}

	Se buscó determinar una diferencia en la calidad de los resultados de las
metaheurísticas \emph{NPSOH} (metaheurística híbrida con K-means) y
\emph{NPSO} (metaheurística no híbrida).

\subsection{Prueba de diferencia de medias para el índice $DB$}

    Para determinar diferencias en las calidades de las soluciones finales de las
metaheurísticas \emph{NPSOH} y \emph{NPSO}, se realizó una prueba de hipótesis de
diferencia de medias del índice $DB$:
\begin{itemize}
    \item \emph{Hipótesis nula}: $\bar{X}_{NPSOH} - \bar{X}_{NPSO} = 0$
    \item \emph{Hipótesis altenativa}: $\bar{X}_{NPSOH} - \bar{X}_{NPSO} \neq 0$
\end{itemize}
donde $\bar{X}_{NPSOH}$ y $\bar{X}_{NPSO}$ son las medias muestrales del índice
$DB$ de las metaheurísticas \emph{NPSOH} y \emph{NPSO} respectivamente.

    Para esto se utilizó la prueba $t$ de \emph{Welch} \cite{AB_0} la cual fue
elegida por no ser sensible a la varianza de los datos.

	Como se puede observar en el Resultado \ref{c-npsos_img}, ambas medias son
estadísticamente diferentes con un nivel de significancia menor a 5\%
($p-valor < 2.2 \cdot 10^{-16}$). { \bf Esto significa que la mejor metaheurística
es \emph{NPSOH} ya que la media de los valores de su índice $DB$ es menor que la
de \emph{NPSO}}.

\begin{lstlisting}[float=h!, caption={Diferencia de Medias: Índice \emph{DB}}, label=c-npsos_img]
> t.test(db ~ name)

	Welch Two Sample t-test

data:  db by name 
t = -19.7045, df = 29.118, p-value < 2.2e-16
alternative hypothesis: true difference in means is not equal to 0 
95 percent confidence interval:
 -2.935532 -2.383535 
sample estimates:
   mean in group NPSOH mean in group NPSO
             0.8296867          3.4892200
\end{lstlisting}

\section{\emph{WPSOH} vs. \emph{WPSO}}

	Se buscó determinar una diferencia en la calidad de los resultados de las
metaheurísticas \emph{WPSOH} (metaheurística híbrida con K-means) y
\emph{WPSO} (metaheurística no híbrida).

\subsection{Prueba de diferencia de medias para el índice $DB$}

    Para determinar diferencias en las calidades de las soluciones finales de las
metaheurísticas \emph{WPSOH} y \emph{WPSO}, se realizó una prueba de hipótesis de
diferencia de medias del índice $DB$:
\begin{itemize}
    \item \emph{Hipótesis nula}: $\bar{X}_{WPSOH} - \bar{X}_{WPSO} = 0$
    \item \emph{Hipótesis altenativa}: $\bar{X}_{WPSOH} - \bar{X}_{WPSO} \neq 0$
\end{itemize}
donde $\bar{X}_{WPSOH}$ y $\bar{X}_{WPSO}$ son las medias muestrales del índice
$DB$ de las metaheurísticas \emph{WPSOH} y \emph{WPSO} respectivamente.

    Para esto se utilizó la prueba $t$ de \emph{Welch} \cite{AB_0} la cual fue
elegida por no ser sensible a la varianza de los datos.

	Como se puede observar en el Resultado \ref{c-wpsos_img}, ambas medias son
estadísticamente diferentes con un nivel de significancia menor a 5\%
($p-valor < 2.2 \cdot 10^{-16}$). { \bf Esto significa que la mejor metaheurística
es \emph{WPSOH} ya que la media de los valores de su índice $DB$ es menor que la
de \emph{WPSO}}.

\begin{lstlisting}[float=h!, caption={Diferencia de Medias: Índice \emph{DB}}, label=c-wpsos_img]
> t.test(db ~ name)

	Welch Two Sample t-test

data:  db by name 
t = -24.1583, df = 29.241, p-value < 2.2e-16
alternative hypothesis: true difference in means is not equal to 0 
95 percent confidence interval:
 -2.92364 -2.46740 
sample estimates:
   mean in group WPSOH mean in group WPSO
             0.8347833          3.5303033 

\end{lstlisting}

\section{\emph{NDEH} vs. \emph{NDE}}

	Se buscó determinar una diferencia en la calidad de los resultados de las
metaheurísticas \emph{NDEH} (metaheurística híbrida con K-means) y
\emph{NDE} (metaheurística no híbrida).

\subsection{Prueba de diferencia de medias para el índice $DB$}

    Para determinar diferencias en las calidades de las soluciones finales de las
metaheurísticas \emph{NDEH} y \emph{NDE}, se realizó una prueba de hipótesis de
diferencia de medias del índice $DB$:
\begin{itemize}
    \item \emph{Hipótesis nula}: $\bar{X}_{NDEH} - \bar{X}_{NDE} = 0$
    \item \emph{Hipótesis altenativa}: $\bar{X}_{NDEH} - \bar{X}_{NDE} \neq 0$
\end{itemize}
donde $\bar{X}_{NDEH}$ y $\bar{X}_{NDE}$ son las medias muestrales del índice
$DB$ de las metaheurísticas \emph{NDEH} y \emph{NDE} respectivamente.

    Para esto se utilizó la prueba $t$ de \emph{Welch} \cite{AB_0} la cual fue
elegida por no ser sensible a la varianza de los datos.

	Como se puede observar en el Resultado \ref{c-ndes_img}, ambas medias son
estadísticamente diferentes con un nivel de significancia menor a 5\%
($p-valor= 1.443 \cdot 10^{-6}$). { \bf Esto significa que la mejor metaheurística
es \emph{NDEH} ya que la media de los valores de su índice $DB$ es menor que la
de \emph{NDE}}.

\begin{lstlisting}[float=h!, caption={Diferencia de Medias: Índice \emph{DB}}, label=c-ndes_img]
> t.test(db ~ name)

	Welch Two Sample t-test

data:  db by name 
t = -5.6727, df = 39.277, p-value = 1.443e-06
alternative hypothesis: true difference in means is not equal to 0 
95 percent confidence interval:
 -0.11984078 -0.05685255 
sample estimates:
   mean in group NDEH mean in group NDE 
            0.8255833         0.9139300
\end{lstlisting}

\section{\emph{SDEH} vs. \emph{SDE}}

	Se buscó determinar una diferencia en la calidad de los resultados de las
metaheurísticas \emph{SDEH} (metaheurística híbrida con K-means) y
\emph{SDE} (metaheurística no híbrida).

\subsection{Prueba de diferencia de medias para el índice $DB$}

    Para determinar diferencias en las calidades de las soluciones finales de las
metaheurísticas \emph{SDEH} y \emph{SDE}, se realizó una prueba de hipótesis de
diferencia de medias del índice $DB$:
\begin{itemize}
    \item \emph{Hipótesis nula}: $\bar{X}_{SDEH} - \bar{X}_{SDE} = 0$
    \item \emph{Hipótesis altenativa}: $\bar{X}_{SDEH} - \bar{X}_{SDE} \neq 0$
\end{itemize}
doSDE $\bar{X}_{SDEH}$ y $\bar{X}_{SDE}$ son las medias muestrales del índice
$DB$ de las metaheurísticas \emph{SDEH} y \emph{SDE} respectivamente.

    Para esto se utilizó la prueba $t$ de \emph{Welch} \cite{AB_0} la cual fue
elegida por no ser sensible a la varianza de los datos.

	Como se puede observar en el Resultado \ref{c-sdes_img}, ambas medias son
estadísticamente diferentes con un nivel de significancia menor a 5\%
($p-valor = 1.058 \cdot 10^{-8}$). { \bf Esto significa que la mejor metaheurística
es \emph{SDEH} ya que la media de los valores de su índice $DB$ es menor que la
de \emph{SDE}}.

\begin{lstlisting}[float=h!, caption={Diferencia de Medias: Índice \emph{DB}}, label=c-sdes_img]
> t.test(db ~ name)

	Welch Two Sample t-test

data:  db by name 
t = -6.6778, df = 57.443, p-value = 1.058e-08
alternative hypothesis: true difference in means is not equal to 0 
95 percent confidence interval:
 -0.1149125 -0.0619008 
sample estimates:
   mean in group SDEH mean in group SDE
            0.8490733         0.9374800
\end{lstlisting}

\section{\emph{BeeH} vs. \emph{Bee}}

	Se buscó determinar una diferencia en la calidad de los resultados de las
metaheurísticas \emph{BeeH} (metaheurística híbrida con K-means) y
\emph{Bee} (metaheurística no híbrida).

\subsection{Prueba de diferencia de medias para el índice $DB$}

    Para determinar diferencias en las calidades de las soluciones finales de las
metaheurísticas \emph{BeeH} y \emph{Bee}, se realizó una prueba de hipótesis de
diferencia de medias del índice $DB$:
\begin{itemize}
    \item \emph{Hipótesis nula}: $\bar{X}_{BeeH} - \bar{X}_{Bee} = 0$
    \item \emph{Hipótesis altenativa}: $\bar{X}_{BeeH} - \bar{X}_{Bee} \neq 0$
\end{itemize}
doBee $\bar{X}_{BeeH}$ y $\bar{X}_{Bee}$ son las medias muestrales del índice
$DB$ de las metaheurísticas \emph{BeeH} y \emph{Bee} respectivamente.

    Para esto se utilizó la prueba $t$ de \emph{Welch} \cite{AB_0} la cual fue
elegida por no ser sensible a la varianza de los datos.

	Como se puede observar en el Resultado \ref{c-bees_img}, ambas medias son
estadísticamente iguales ($p-valor = 1$). { \bf Esto significa que ambas
metaheurísticas obtienen soluciones finales con similar calidad}.

\begin{lstlisting}[float=h!, caption={Diferencia de Medias: Índice \emph{DB}}, label=c-bees_img]
> t.test(db ~ name)

	Welch Two Sample t-test

data:  db by name 
t = 0, df = 58, p-value = 1
alternative hypothesis: true difference in means is not equal to 0 
95 percent confidence interval:
 -0.006853032  0.006853032 
sample estimates:
   mean in group BeeH mean in group Bee 
               0.7664            0.7664
\end{lstlisting}

\subsection{Cantidad de evaluaciones de la función de \emph{fitness}}

	En la sección anterior, se demostró que ambas metaheurísticas obtienen
soluciones finales con similar calidad. Si no existe una diferencia en
rendimiento, entonces ambos algoritmos son iguales y cualquiera de ellos puede
ser un buen candidato para competir con las demás metaheurísticas.

	Como las metaheurísticas \emph{BeeH} y \emph{Bee} utilizan la misma función de
\emph{fitness} y esta es la función más costosa durante la ejecución de las mismas,
se buscó determinar una diferencia en la cantidad de evaluaciones de la función
de \emph{fitness} de las metaheurísticas. Para esto se realizó una prueba de
hipótesis de diferencia de medias:
\begin{itemize}
    \item \emph{Hipótesis nula}: $\bar{X}_{BeeH} - \bar{X}_{Bee} = 0$
    \item \emph{Hipótesis altenativa}: $\bar{X}_{BeeH} - \bar{X}_{Bee} \neq 0$
\end{itemize}
donde $\bar{X}_{BeeH}$ y $\bar{X}_{Bee}$ son las medias muestrales de la cantidad
de evaluaciones de la función de fitness de las metaheurísticas \emph{BeeH} y
\emph{Bee} respectivamente.

    Para esto se utilizó la prueba $t$ de \emph{Welch} \cite{AB_0} la cual fue
elegida por no ser sensible a la varianza de los datos.

	Como se puede observar en el Resultado \ref{c-bees_img1}, ambas medias son
estadísticamente iguales ($p-valor = 0.8853$). {\bf Esto significa que ambas
metaheurísticas tienen un rendimiento similar}. Sin embargo, \emph{BeeH} tiene
más oportunidad de mejorar sus soluciones parciales al estar hibridado con el
algoritmo \emph{K-means} que su contraparte no hibridada \emph{Bee}. {\bf Por lo
tanto, \emph{BeeH} es una metaheurística más confiable}.

\begin{lstlisting}[float=h!, caption={Diferencia de medias: Cantidad de Evaluaciones}, label=c-bees_img1]
> t.test(eval ~ name)

	Welch Two Sample t-test

data:  eval by name 
t = 0.1449, df = 58, p-value = 0.8853
alternative hypothesis: true difference in means is not equal to 0 
95 percent confidence interval:
 -51.26416  59.26416 
sample estimates:
   mean in group BeeH mean in group Bee 
             342.3333          338.3333 
\end{lstlisting}

\section{\emph{AntH} vs. \emph{Ant}}

	Se buscó determinar una diferencia en la calidad de los resultados de las
metaheurísticas \emph{AntH} (metaheurística híbrida con K-means) y
\emph{Ant} (metaheurística no híbrida).

\subsection{Prueba de diferencia de medias para el índice $DB$}

    Para determinar diferencias en las calidades de las soluciones finales de las
metaheurísticas \emph{AntH} y \emph{Ant}, se realizó una prueba de hipótesis de
diferencia de medias del índice $DB$:
\begin{itemize}
    \item \emph{Hipótesis nula}: $\bar{X}_{AntH} - \bar{X}_{Ant} = 0$
    \item \emph{Hipótesis altenativa}: $\bar{X}_{AntH} - \bar{X}_{Ant} \neq 0$
\end{itemize}
doAnt $\bar{X}_{AntH}$ y $\bar{X}_{Ant}$ son las medias muestrales del índice
$DB$ de las metaheurísticas \emph{AntH} y \emph{Ant} respectivamente.

    Para esto se utilizó la prueba $t$ de \emph{Welch} \cite{AB_0} la cual fue
elegida por no ser sensible a la varianza de los datos.

	Como se puede observar en el Resultado \ref{c-ants_img}, ambas medias son
estadísticamente diferentes con un nivel de significancia menor a 5\%
($p-valor = 9.22 \cdot 10^{-5}$). { \bf Esto significa que la mejor metaheurística
es \emph{AntH} ya que la media de los valores de su índice $DB$ es menor que la
de \emph{Ant}}.

\begin{lstlisting}[float=h!, caption={Diferencia de Medias: Índice \emph{DB}}, label=c-ants_img]
> t.test(db ~ name)

	Welch Two Sample t-test

data:  db by name 
t = -4.5349, df = 29, p-value = 9.22e-05
alternative hypothesis: true difference in means is not equal to 0 
95 percent confidence interval:
 -7455055911 -2820701719 
sample estimates:
   mean in group AntH mean in group Ant
         9.483033e-01      5.137879e+09 

\end{lstlisting}

\section{Conclusión}

	Las metaheurísticas híbridas con el algoritmo \emph{K-means} son las que
reportan mejor calidad en sus soluciones finales (\emph{GAH}, \emph{NPSOH},
\emph{WPSOH}, \emph{NDEH}, \emph{SDEH}, \emph{BeeH}, \emph{AntH}) con respecto a
sus contrapartes no híbridas (\emph{GA}, \emph{NPSO}, \emph{WPSO}, \emph{NDE},
\emph{SDE}, \emph{Bee}, \emph{Ant}). Por lo tanto, las metaheurísticas híbridas
fueron elegidas para hacer comparaciones de calidad y rendimiento entre ellas en
el capítulo \ref{chap:analisis}.
