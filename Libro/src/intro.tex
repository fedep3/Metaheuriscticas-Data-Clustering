 \addcontentsline{toc}{chapter}{Introducci\'on}

% Titulo de la introduccion.
\begin{center}
	{\bf Introducci\'on} \label{chap:intro}
\end{center}

% Contenido de la introduccion.

\label{sect:motivacion}
%Puedes quitar esto(es opcional)
\vspace{5 mm}

AQU\'I VA EL CONTENIDO DE ANTECEDENTES. 

%Puedes quitar esto(es opcional)
\vspace{5 mm}

\label{sect:justificacion}
%Puedes quitar esto(es opcional)
\vspace{5 mm}

AQU\'I VA EL CONTENIDO DE LA JUSTIFICACI\'ON.
%Puedes quitar esto(es opcional)
\vspace{5 mm}

En la Universidad de Tohoku de Sendai, Jap'on, en el laboratorio Kinoshita de
la facultad de Ciencias de la Computaci'on, se tiene un {\it framework} con el
cual han creado un {\it Active Information Resource} (AIR) - {\it Network
Management System} (NMS) \cite{kinoshitaNMSAIR} para gestionar las redes de
computadoras. Mediante el lenguaje de programaci'on orientado a agentes
basado en Java, DASH ({\it Repository-based Agent Framework}) y su ambiente
de dise~no interactivo IDEA ({\it Interactive Design Environment}), la idea
final es que la red y los elementos que la componen se pueda auto-gestionar
mediante los agentes instalados en los elementos \cite{545112}, sin la
necesidad de tener un administrador. El objectivo es identificar y solucionar
los problemas de la red en tiempo real \cite{AbarAK04} \cite{KonnoAIK07}.

%Puedes quitar esto(es opcional)
\vspace{5 mm}


\label{sect:planteamiento}
%Puedes quitar esto(es opcional)
\vspace{5 mm}

AQU\'I VA EL CONTENIDO DEL PLANTEAMIENTO DEL PROBLEMA.

En los 'ultimos a~nos ha habido un gran inter'es por determinar si los nuevos
modelos basados en la autosimilaridad y la dependencia de largo alcance pueden
ser utilizados para la detecci'on de anomal'ias en las redes de datos
\cite{5475821}. Entre las anomal'ias m'as destacadas se encuentran las de
ataques de denegaci'on de servicio y su ampliaci'on: el llamado ataque
distribuido de denegaci'on de servicio \cite{mingliddos} \cite{xiang:292}.

En el trabajo de \cite{mingliddos} se utiliz'o, con buenos resultados, el
c'alculo del grado de autosimilaridad para detectar anomal'ias en la red. Se
realiz'o una implementaci'on para un {\it framework} de autogesti'on de redes
desarrollado en el laboratorio Kinoshita de la Universidad de Tohoku. Como se
explica en la pr'oxima secci'on la idea del proyecto de grado fue mejorar esta
implementaci'on.

%Puedes quitar esto(es opcional)
\vspace{5 mm}

\label{sect:objetivo_general}
%Puedes quitar esto(es opcional)
\vspace{5 mm}

AQU\'I VA EL OBJETIVO GENERAL.

El objetivo principal de este proyecto es el desarrollo de una soluci'on
flexible de l'inea de comando que estime el par'ametro de Hurst para medir el
grado de autosimilaridad de una traza de red. La estimaci'on del par'ametro de
Hurst deber'a ser calculado en tiempo real, sin comprometer la precisi'on de la
estimaci'on. Para probar la factibilidad de la soluci'on en tiempo real los
miembros del laboratorio Kinoshita de la Universidad de Tohoku propusieron
trabajar con el mecanismo de ventanas deslizantes como se presenta en el
art'iculo \cite{TakahashiAkinori:20070913}.


%Puedes quitar esto(es opcional)
\vspace{5 mm}

\label{sect:objetivos_especificos}
%Puedes quitar esto(es opcional)
\vspace{5 mm}

AQU\'I VAN LOS OBJETIVOS ESPEC\'IFICOS.

La implementaci'on deber'a tomar como entrada una traza creada por la librer'ia
{\tt libpcap}\footnote{Esta librer'ia provee una interfaz de alto nivel para la
captura y manipulaci'on de los paquetes a bajo nivel pudiendo manipular todos 
los paquetes, incluidos aquellos destinados a otros nodos de red. La librer'ia
se encuentra disponible en \url{http://www.tcpdump.org}.} y poder
crear f'acilmente una serie de tiempo para la estimaci'on del par'ametro de
Hurst, haciendo flexible la estimaci'on con diferentes series de tiempo creadas
a partir de la misma traza. Como salida, el programa debe graficar los
resultados de las estimaciones del par'ametro de Hurst, incluyendo el cambio del
par'ametro en el tiempo, para as'i poder verificar de forma r'apida los
resultados. 

Una vez validada la estimaci'on del par'ametro de Hurst, se deber'a verificar si
esta estimaci'on puede ser utilizada para detectar las anomal'ias en tiempo
real creadas por un ataque distribuido de denegaci'on de servicio en las trazas
de red.


