 \addcontentsline{toc}{chapter}{Introducci\'on}

% Titulo de la introduccion.
\begin{center}
	{\bf Introducci\'on} \label{chap:intro}
\end{center}

% Contenido de la introduccion.

\label{sect:motivacion}
%Puedes quitar esto(es opcional)
\vspace{5 mm}

En la Universidad Sim\'on Bol\'ivar es dictada la materia electiva
Dise\~no de Algoritmos II, en la cual se enseñan diversas metaheur\'isticas.
El curso es dividido en grupos (normalmente en parejas) y a cada uno se le asigna un problema 
de optimizaci\'on NP-hard para resolver mediante estos m\'etodos. Uno de \'estos es Data Clustering.
Consiste m\'etodo de crear grupos de objetos,
de tal manera que los objetos dentro de un cluster sean similares y 
en clusters distintos sean diferentes. \cite{GaChJi2007}

%Puedes quitar esto(es opcional)
\vspace{5 mm}

\label{sect:justificacion}
%Puedes quitar esto(es opcional)
\vspace{5 mm}

Se sabe que hoy en d\'ia hay muchos avances y usos del reconocimiento de im\'agenes.
Ejemplos son Google Goggles, diversos m\'odulos de JDownloader con el objetivo de reconocer
captchas, reconocimiento de caras por parte de las c\'amaras, etc. Un algoritmo de
Data Clustering sirve como entrada para un modelo basado en sistemas de
reconocimiento de objetos. Tambi\'en el poder encontrar patrones escondido en
conjuntos de datos es muy \'util: una empresa podr\'ia mejorar su ventas.

%Puedes quitar esto(es opcional)
\vspace{5 mm}

\label{sect:planteamiento}
%Puedes quitar esto(es opcional)
\vspace{5 mm}

Actualmente existe una ardua investigaci\'on en diversas metaheur\'isticas para
resolver diversos problemas optimizaci\'on. La idea es obtener en un tiempo
considerable una respuesta \'optima o lo m\'as cercano posible a \'esta. Por
ello muchos cient\'ificos han empezado a inspirarse en la naturaleza.
Una profunda obsevaci\'on en la relaci\'on subyacente 
entre \'esta y optimizaci\'on ha llevado al desarrollo de nuevos paradigmas
para lograr este objetivo\cite{SwAjAm2009}. De ac\'a surgen las basadas en poblaci\'on e inteligencia
colectiva.

Los m\'as prominentes y prometedores son el algoritmo g\'enetico, abeja, DE, hormiga
y PSO. Todos constan de un grupo de individuos que van a ir modificandose a trav\'es
del tiempo. La forma en que se den estos cambios depende de cada algoritmo.

%Puedes quitar esto(es opcional)
\vspace{5 mm}

\label{sect:objetivo_general}
%Puedes quitar esto(es opcional)
\vspace{5 mm}

El objetivo principal es la implementaci\'on de \'estas cinco metaheur\'isticas
y el K-means, el algoritmo m\'as famoso para data clustering, para ver la calidad
de cada uno y compararlos entre si. \'Esta debe ser eficiente, rapida y flexible
con el objetivo de que pueda ser usada a futuro por otras personas y lograr
comprar los algoritmos de la forma m\'as efectiva.


%Puedes quitar esto(es opcional)
\vspace{5 mm}

\label{sect:objetivos_especificos}
%Puedes quitar esto(es opcional)
\vspace{5 mm}

El programa debe tener la capcidad de leer archivos en formato CSV e im\'agenes
en formato PNG y TIFF (las im\'agenes son datos n\'umericos). La salida tiene que ser
concorde al tipo de cada archivo, para un csv un archivo de texto y
en el otro caso una imagen con un color para cada grupo creado. Es necesario que
el lenguaje que se use tenga la capcidad de compilar a c\'odigo de m\'aquina
en vista que se busca eficiencia en tiempo.

Para comparar las diversas metaheur\'sticas es esencial tener una m\'etrica com\'un
con ese fin. Adem\'as se debe colocar un limite de tiempo en las corridas para 
obtener una buena soluci\'on en un tiempo considerable.


