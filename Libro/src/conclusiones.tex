% Conclusiones
\chapter{Conclusiones} \label{chap:conclusiones}

    En este capítulo, se presentan los hallazgos y contribuciones de este
proyecto y se dan algunas recomendaciones para futuros trabajos:
\begin{comment}
    Se evaluó el desempeño de los algoritmos \emph{K-means}, \emph{GAH},
\emph{NPSOH}, \emph{WPSOH}, \emph{NDEH}, \emph{SDEH}, \emph{BeeH}, \emph{AntH}.
Estos ocho algoritmos fueron comparados mediante el índice de validez $DB$ y la
cuantificación del error $J_e$ para determinar la calidad de sus soluciones
finales. Además, se comparó el rendimiento de las metaheurísticas \emph{GAH} y
\emph{BeeH} por medio de la cantidad de evaluaciones de su función de \emph{fitness}.
Así que, en función del análisis hecho en el capítulo anterior (ver capítulo \ref{chap:analisis}),
se puede concluir:
\end{comment}

\begin{itemize}
    \item Se llevó a cabo un estudio comparativo entre los algoritmos \emph{K-means},
          \emph{GAH}, \emph{NPSOH}, \emph{WPSOH}, \emph{NDEH}, \emph{SDEH},
          \emph{BeeH} y \emph{AntH} donde \emph{BeeH} y \emph{GAH} resultaron
          ser las mejores metaheurísticas.

    \item Se encontró que \emph{GAH} necesita probabilidades de mutación altas
          ($\geq 0.5$) para que sus soluciones finales sean de calidad.

    \item Se recomienda usar \emph{GAH} cuando no se tenga conocimiento a priori
          del número de clusters.

    \item El índice de validez $DB$ es más eficaz para medir la calidad de una
          partición que la cuantificación de error $J_e$. Esto se debe a que $J_e$
          se ve afectado por la cardinalidad de los clusters.

    \item Las soluciones finales de las metaheurísticas híbridas \emph{NPSOH},
          \emph{WPSOH}, \emph{NDEH}, \emph{SDEH} son similares en calidad. Esto
          puede deberse a que comparten procedimientos.

    \item El \emph{AntH} fue la peor metaheurística implementada. Este comportamiento
          se le atribuye a que no se logró realizar una implementación que
          reprodujera los resultados del artículo \cite{OuBa2007}.

    \item Las metaheurísticas híbridas arrojaron mejores resultados que sus
          contrapartes no híbridas.
          
\end{itemize}

\begin{comment}
\begin{itemize}

	\item El \emph{BeeH} y el \emph{GAH} resultaron ser las mejores metaheurísticas.
    Sin embargo, el \emph{GAH} consiguió soluciones finales con una calidad parecida
    a las del \emph{BeeH} utilizando menos evaluaciones de la función de
    \emph{fitness}.

	\item El \emph{GAH} es la única metaheurística que no requiere la cantidad
    de clusters lo cual es una ventaja cuando no se conoce \emph{a priori}.

	\item Tener una alta probabilidad de mutación ($\geq 0.5$) en el \emph{GAH}
	genera mejores resultados, indicando que el problema de \emph{data clustering}
    necesita grandes modificaciones en los vectores de centroides de sus
    individuos para obtener buenas soluciones.

    \item El índice de validez $DB$ es más eficaz para medir la calidad de una
    partición que la cuantificación del error $J_e$. Esto se debe a que el error
    $J_e$ sugería que la metaheurística \emph{AntH} era la mejor de todas las
    metaheurísticas implementadas. Sin embargo, resultados como los mostrados en
    las figuras \subref{fig:lennaant} y \subref{fig:trivialant}, donde se observa
    mucho ruido, demuestran lo contrario.
    

	\item Las metaheurísticas \emph{NDEH}, \emph{SDEH}, \emph{NPSOH} y \emph{WPSOH}
    comparten un procedimiento para mantener el número de clusters fijo y esto
	podría estar afectando su desempeño, haciendo que sus soluciones sean similares
    entre sí.

	\item El \emph{AntH} fue la peor metaheurística implementada. Este comportamiento
    puede tener su origen en que no se logró hacer una implementación que
    reprodujera los resultados reportados en el artículo base \cite{OuBa2007}.
    En este artículo, no habían suficientes detalles sobre la implementación
    de la memoria de las hormigas y esto pudo haber llevado a una implementación
    ineficiente de la metaheurística \emph{AntH}.
\end{itemize}

\end{comment}

	A partir de los resultados y conclusiones de este proyecto de grado, se pueden
dar las siguientes recomendaciones:

	Primero, es posible que los resultados de las metaheurísticas sean sensibles
a los conjuntos de datos a particionar. Es por esto que todas las metaheurísticas deberían ser probadas
con otros conjuntos de datos, además de \textbf{Lenna} e \textbf{Iris}, de modo
que se corrobore lo dicho en este proyecto de grado.

	Segundo, buscar un nuevo enfoque para la memoria del \emph{AntH}, de modo que
se puedan reproducir los resultados del artículo \cite{OuBa2007}. En este
artículo, no habían suficientes detalles sobre la misma y la implementación
realizada pudo haber afectado el desempeño de esta metaheurística.

	Por último, el procedimiento para mantener los clusters fijos que comparten
las metaheurísticas \emph{NDEH}, \emph{SDEH}, \emph{NPSOH} y \emph{WPSOH} pudo
haber afectado su desempeño. Buscar un nuevo enfoque para este procedimiento,
podría lograr una mejora en los resultados de estas metaheurísticas.
