% Apendice
\chapter{Pruebas exhaustivas}
\label{apendicea}

	Se usaron dos archivos de pruebas: la imagen \textbf{Lenna} (ver figura \ref{fig:lenna})
y el archivo de datos numéricos \textbf{Iris} (ver sección \ref{test:iris}).

    Para las pruebas, se recabó información de cada metaheurística de la
siguiente forma:
\begin{enumerate}
    \item Se generaron valores para el conjunto de parámetros de cada
metaheurística a partir de la permutación de rangos discretos válidos para cada
uno de los mismos (véanse las secciones  \ref{sect:agenetico}, \ref{sect:anpso},
\ref{sect:awpso}, \ref{sect:ande}, \ref{sect:asde}, \ref{sect:abee},
\ref{sect:aant}).
    \item Se ejecutó cinco veces cada metaheurística con cada uno de los conjunto de
valores generados anteriormente.
    \item De los resultados obtenidos, se promediaron las cinco corridas de cada
metaheurística y se tomaron los veinte mejores conjunto de valores.
    \item Cada metaheurística fue ejecutada nuevamente con cada uno de los veinte
mejores conjuntos de valores para ella. Fueron ejecutadas treinta veces para
cada conjunto de valores.
\end{enumerate}

	Algunos parámetros de las metaheurísticas se mantuvieron fijos:
\begin{itemize}
	\item {\bf Número de clusters iniciales ($K$):} Para la imagen {\bf Lenna} se
tomó $K=9$ y para el conjunto de datos {\bf Iris}, $K=3$.

	\item {\bf Número de veces sin mejoras:} Es la condición de parada de los
siguientes algoritmos: \emph{Kmeans}, \emph{GAH}, \emph{BeeH} y \emph{PSOH} (en sus
dos versiones: \emph{WPSOH} y \emph{NPSOH}). Para todos ellos, se fijó en 3
repeticiones.

	\item {\bf Número de iteraciones:} Es la condición de parada de los siguientes
algoritmos:
    \begin{itemize}
        \item \emph{DEH}: en sus dos versiones \emph{NDE} y \emph{SDE}, se fijó
    en 10 iteraciones.
        \item \emph{AntH}: Se fijaron en 10000000 iteraciones para {\bf Lenna} y
    100000 iteraciones para {\bf Iris}.
    \end{itemize}
\end{itemize}

	Las tablas que se presentan a continuación son un resumen de los resultados
de cada una de las metaheurísticas con sus 20 mejores conjuntos de parámetros.
Están ordenadas por el valor de la función de \emph{fitness} (\textbf{FO}) de
forma descendiente para los mejores resultados de cada conjunto de parámetros.

\section{Algoritmo Genético (\emph{GAH})}\label{sect:agenetico}

    Las variables del \emph{GAH}(\ref{sect:igenetico}) son las siguientes:
    \begin{itemize}
        \item $I$: tamaño de la población. Se varió su valor en el rango
    $[5, 10, \cdots, 40]$.
        \item $tt$: tamaño del torneo. Se varió su valor en el rango
    $[4, 6, \cdots, 40]$.
        \item $pc$: probabilidad de cruce. Se varió su valor en el rango
    $[0.1; 0.2; \cdots, 1.0]$.
        \item $pm$: probabilidad de mutación. Se varió su valor en el rango
    $[0.1; 0.2; \cdots, 1.0]$.
    \end{itemize}

\begin{table}[h!]
    \footnotesize
    \begin{center}
        \begin{tabular}{|c|c|c|c|c|c|c|c|c|c|}
        \hline
            & {\bf FO} & {\bf DB} & $J_e$ & {\bf E} & {\bf T} & $I$ & $tt$ & $pc$ & $pm$ \\
        \hline
        \hline
            Promedio  & 1.2614 & 0.7941 & 17.6439 & 30.5172 & 0.0411 &  &  &  & \\
            \cline{1-6}
            Mejor & 1.4059 & 0.7113  & 19.0892 & 33 & 0.0551 & 10 & 4 & 0.8 & 1.0\\
            \cline{1-6}
            Peor & 1.1648 & 0.8586  & 17.2709 & 22 & 0.0285 &  &  &  & \\
        \hline
        \hline
            Promedio  & 1.2682 & 0.7896 & 18.2618 & 26.1379 & 0.0345 &  &  &  & \\
            \cline{1-6}
            Mejor & 1.3746 & 0.7275  & 18.1834 & 32 & 0.0416 & 5 & 4 & 0.1 & 0.8\\
            \cline{1-6}
            Peor & 1.1469 & 0.8719  & 18.001 & 21 & 0.0339 &  &  &  & \\
        \hline
        \hline
            Promedio  & 1.2526 & 0.7995 & 17.0793 & 39.7667 & 0.0543 &  &  &  & \\
            \cline{1-6}
            Mejor & 1.365 & 0.7326  & 16.74 & 36 & 0.0473 & 20 & 8 & 0.1 & 0.7\\
            \cline{1-6}
            Peor & 1.1562 & 0.8649  & 17.8023 & 31 & 0.0467 &  &  &  & \\
        \hline
        \hline
            Promedio  & 1.2492 & 0.8014 & 17.3988 & 27.6333 & 0.0376 &  &  &  & \\
            \cline{1-6}
            Mejor & 1.3526 & 0.7393  & 15.8534 & 36 & 0.0452 & 10 & 4 & 0.5 & 0.6\\
            \cline{1-6}
            Peor & 1.1907 & 0.8399  & 17.269 & 18 & 0.0235 &  &  &  & \\
        \hline
        \hline
            Promedio  & 1.2617 & 0.794 & 17.4689 & 32.4667 & 0.0424 &  &  &  & \\
            \cline{1-6}
            Mejor & 1.3516 & 0.7399  & 16.6283 & 48 & 0.0659 & 10 & 6 & 0.7 & 1.0\\
            \cline{1-6}
            Peor & 1.133 & 0.8826  & 17.3372 & 21 & 0.0292 &  &  &  & \\
        \hline
        \hline
            Promedio  & 1.2562 & 0.7968 & 17.08 & 57.6 & 0.0788 &  &  &  & \\
            \cline{1-6}
            Mejor & 1.3455 & 0.7432  & 17.0687 & 49 & 0.0657 & 40 & 24 & 0.8 & 0.6\\
            \cline{1-6}
            Peor & 1.1859 & 0.8432  & 17.4103 & 49 & 0.0665 &  &  &  & \\
        \hline
        \hline
            Promedio  & 1.2588 & 0.7952 & 17.2521 & 55.0333 & 0.0756 &  &  &  & \\
            \cline{1-6}
            Mejor & 1.3455 & 0.7432  & 17.0687 & 45 & 0.0601 & 35 & 24 & 0.5 & 0.9\\
            \cline{1-6}
            Peor & 1.1601 & 0.862  & 17.4504 & 46 & 0.0757 &  &  &  & \\
        \hline
        \hline
            Promedio  & 1.2701 & 0.7881 & 17.7955 & 27.7 & 0.037 &  &  &  & \\
            \cline{1-6}
            Mejor & 1.336 & 0.7485  & 18.37 & 24 & 0.0308 & 5 & 4 & 0.1 & 0.9\\
            \cline{1-6}
            Peor & 1.1577 & 0.8638  & 19.469 & 25 & 0.0318 &  &  &  & \\
        \hline
        \hline
            Promedio  & 1.2565 & 0.7969 & 17.2945 & 40.5667 & 0.056 &  &  &  & \\
            \cline{1-6}
            Mejor & 1.3352 & 0.7489  & 16.7962 & 56 & 0.0888 & 20 & 10 & 0.3 & 1.0\\
            \cline{1-6}
            Peor & 1.1426 & 0.8752  & 16.7225 & 29 & 0.0384 &  &  &  & \\
        \hline
        \hline
            Promedio  & 1.25 & 0.8008 & 17.2049 & 29.2667 & 0.0387 &  &  &  & \\
            \cline{1-6}
            Mejor & 1.3348 & 0.7492  & 16.9786 & 33 & 0.0422 & 10 & 8 & 0.2 & 0.6\\
            \cline{1-6}
            Peor & 1.1485 & 0.8707  & 17.2326 & 19 & 0.0253 &  &  &  & \\
        \hline
        \hline
            Promedio  & 1.2665 & 0.7902 & 17.1386 & 48.5333 & 0.0666 &  &  &  & \\
            \cline{1-6}
            Mejor & 1.3287 & 0.7526  & 19.294 & 66 & 0.0928 & 25 & 10 & 0.6 & 0.9\\
            \cline{1-6}
            Peor & 1.1827 & 0.8455  & 17.6733 & 36 & 0.052 &  &  &  & \\
        \hline
        \hline
            Promedio  & 1.261 & 0.7935 & 17.207 & 40.6667 & 0.0546 &  &  &  & \\
            \cline{1-6}
            Mejor & 1.3284 & 0.7528  & 18.5047 & 41 & 0.0534 & 20 & 16 & 0.5 & 1.0\\
            \cline{1-6}
            Peor & 1.1993 & 0.8338  & 17.0813 & 36 & 0.0476 &  &  &  & \\
        \hline
        \hline
            Promedio  & 1.2663 & 0.7902 & 17.1675 & 39.4667 & 0.0529 &  &  &  & \\
            \cline{1-6}
            Mejor & 1.3284 & 0.7528  & 18.5047 & 41 & 0.0533 & 20 & 16 & 0.5 & 0.9\\
            \cline{1-6}
            Peor & 1.1993 & 0.8338  & 17.0813 & 36 & 0.0474 &  &  &  & \\
        \hline
        \hline
            Promedio  & 1.2553 & 0.7975 & 17.674 & 25.5714 & 0.0339 &  &  &  & \\
            \cline{1-6}
            Mejor & 1.3266 & 0.7538  & 18.3825 & 44 & 0.0566 & 5 & 4 & 0.8 & 1.0\\
            \cline{1-6}
            Peor & 1.1289 & 0.8858  & 16.4444 & 16 & 0.0209 &  &  &  & \\
        \hline
        \hline
            Promedio  & 1.2452 & 0.805 & 17.4667 & 27.3103 & 0.0353 &  &  &  & \\
            \cline{1-6}
            Mejor & 1.3208 & 0.7571  & 17.2918 & 44 & 0.0545 & 5 & 4 & 0.2 & 0.9\\
            \cline{1-6}
            Peor & 1.0922 & 0.9156  & 16.5765 & 17 & 0.022 &  &  &  & \\
        \hline
        \end{tabular}
        \caption{Resultados de las mejores corridas de \emph{GA} no hibridado para {\bf Lenna}}
        \label{tb:tablegaalgimg}
    \end{center}
\end{table}


\begin{table}[h!]
    \footnotesize
    \begin{center}
        \begin{tabular}{|c|c|c|c|c|c|c|c|c|c|}
        \hline
            & {\bf FO} & {\bf DB} & $J_e$ & {\bf E} & {\bf T} & $I$ & $tt$ & $pc$ & $pm$ \\
        \hline
        \hline
            Promedio  & 1.2527 & 0.7992 & 17.0987 & 32.8333 & 0.044 &  &  &  & \\
            \cline{1-6}
            Mejor & 1.3185 & 0.7584  & 17.8374 & 49 & 0.0625 & 10 & 6 & 0.8 & 0.5\\
            \cline{1-6}
            Peor & 1.133 & 0.8826  & 17.3372 & 21 & 0.0309 &  &  &  & \\
        \hline
        \hline
            Promedio  & 1.2518 & 0.7996 & 17.083 & 37.4667 & 0.0514 &  &  &  & \\
            \cline{1-6}
            Mejor & 1.3152 & 0.7604  & 16.5793 & 33 & 0.0425 & 15 & 14 & 1.0 & 0.9\\
            \cline{1-6}
            Peor & 1.1753 & 0.8509  & 17.1032 & 25 & 0.0392 &  &  &  & \\
        \hline
        \hline
            Promedio  & 1.247 & 0.8031 & 17.0098 & 36.7586 & 0.0497 &  &  &  & \\
            \cline{1-6}
            Mejor & 1.3152 & 0.7604  & 16.5793 & 33 & 0.0428 & 15 & 14 & 1.0 & 1.0\\
            \cline{1-6}
            Peor & 1.0866 & 0.9203  & 16.8801 & 25 & 0.0332 &  &  &  & \\
        \hline
        \hline
            Promedio  & 1.2508 & 0.8005 & 17.0714 & 45.6 & 0.0626 &  &  &  & \\
            \cline{1-6}
            Mejor & 1.3152 & 0.7604  & 16.5793 & 36 & 0.048 & 25 & 10 & 0.6 & 0.8\\
            \cline{1-6}
            Peor & 1.1063 & 0.9039  & 17.0056 & 33 & 0.0441 &  &  &  & \\
        \hline
        \hline
            Promedio  & 1.2419 & 0.8059 & 17.0402 & 47.8621 & 0.0649 &  &  &  & \\
            \cline{1-6}
            Mejor & 1.3097 & 0.7635  & 16.8631 & 40 & 0.0539 & 30 & 16 & 0.5 & 0.7\\
            \cline{1-6}
            Peor & 1.1664 & 0.8573  & 16.6878 & 41 & 0.0543 &  &  &  & \\
        \hline
        \end{tabular}
        \caption{Continuacion resultados de las mejores corridas de \emph{GA} no hibridado para {\bf Lenna}}
        \label{tb:tablegaalgimgc}
    \end{center}
\end{table}


\begin{table}[h!]
    \footnotesize
    \begin{center}
        \begin{tabular}{|c|c|c|c|c|c|c|c|c|c|c|c|}
        \hline
            & {\bf FO} & {\bf DB} & $J_e$ & {\bf E} & {\bf T} & {\bf KE} & {\bf KO} & $I$ & $tt$ & $pc$ & $pm$ \\
        \hline
        \hline
            Promedio  & 1.2753 & 0.7853 & 17.5582 & 35.2069 & 0.0051 & 8.6897 & $[5-10]$ &  &  &  & \\
            \cline{1-8}
            Mejor & 1.4104 & 0.709  & 19.4009 & 38 & 0.0045 & 7 & $[5-10]$ & 10 & 4 & 0.8 & 1.0\\
            \cline{1-8}
            Peor & 1.179 & 0.8481  & 17.6034 & 32 & 0.0055 & 9 & $[5-10]$ &  &  &  & \\
        \hline
        \hline
            Promedio  & 1.2848 & 0.7793 & 18.1359 & 34.2414 & 0.0084 & 8.1034 & $[5-10]$ &  &  &  & \\
            \cline{1-8}
            Mejor & 1.3747 & 0.7274  & 21.2439 & 50 & 0.0175 & 6 & $[5-10]$ & 5 & 4 & 0.1 & 0.8\\
            \cline{1-8}
            Peor & 1.1577 & 0.8638  & 19.469 & 29 & 0.0034 & 7 & $[5-10]$ &  &  &  & \\
        \hline
        \hline
            Promedio  & 1.2657 & 0.7909 & 17.032 & 46.0 & 0.0085 & 9.0 & $[5-10]$ &  &  &  & \\
            \cline{1-8}
            Mejor & 1.365 & 0.7326  & 16.74 & 40 & 0.0041 & 9 & $[5-10]$ & 20 & 8 & 0.1 & 0.7\\
            \cline{1-8}
            Peor & 1.1745 & 0.8514  & 17.4607 & 36 & 0.0055 & 9 & $[5-10]$ &  &  &  & \\
        \hline
        \hline
            Promedio  & 1.281 & 0.7817 & 17.4177 & 38.6333 & 0.0071 & 8.6667 & $[5-10]$ &  &  &  & \\
            \cline{1-8}
            Mejor & 1.3627 & 0.7339  & 21.2016 & 40 & 0.0133 & 6 & $[5-10]$ & 10 & 6 & 0.7 & 1.0\\
            \cline{1-8}
            Peor & 1.1696 & 0.855  & 17.3385 & 25 & 0.0041 & 9 & $[5-10]$ &  &  &  & \\
        \hline
        \hline
            Promedio  & 1.2652 & 0.7914 & 17.39 & 33.6333 & 0.0068 & 8.8 & $[5-10]$ &  &  &  & \\
            \cline{1-8}
            Mejor & 1.3566 & 0.7371  & 18.1806 & 40 & 0.0062 & 8 & $[5-10]$ & 10 & 4 & 0.5 & 0.6\\
            \cline{1-8}
            Peor & 1.1907 & 0.8399  & 17.269 & 22 & 0.0043 & 9 & $[5-10]$ &  &  &  & \\
        \hline
        \hline
            Promedio  & 1.2642 & 0.7916 & 17.0933 & 62.9333 & 0.0063 & 8.9333 & $[5-10]$ &  &  &  & \\
            \cline{1-8}
            Mejor & 1.3455 & 0.7432  & 17.0687 & 53 & 0.0041 & 9 & $[5-10]$ & 40 & 24 & 0.8 & 0.6\\
            \cline{1-8}
            Peor & 1.186 & 0.8432  & 16.7878 & 57 & 0.0043 & 9 & $[5-10]$ &  &  &  & \\
        \hline
        \hline
            Promedio  & 1.2711 & 0.7873 & 17.3011 & 62.0667 & 0.008 & 8.7333 & $[5-10]$ &  &  &  & \\
            \cline{1-8}
            Mejor & 1.3455 & 0.7432  & 17.0687 & 49 & 0.0043 & 9 & $[5-10]$ & 35 & 24 & 0.5 & 0.9\\
            \cline{1-8}
            Peor & 1.2017 & 0.8321  & 17.8428 & 55 & 0.0106 & 9 & $[5-10]$ &  &  &  & \\
        \hline
        \hline
            Promedio  & 1.2824 & 0.7807 & 17.7856 & 34.4 & 0.0074 & 8.4333 & $[5-10]$ &  &  &  & \\
            \cline{1-8}
            Mejor & 1.3377 & 0.7475  & 19.4622 & 25 & 0.008 & 7 & $[5-10]$ & 5 & 4 & 0.1 & 0.9\\
            \cline{1-8}
            Peor & 1.1577 & 0.8638  & 19.469 & 29 & 0.0034 & 7 & $[5-10]$ &  &  &  & \\
        \hline
        \hline
            Promedio  & 1.2688 & 0.7887 & 17.3029 & 48.2333 & 0.0092 & 8.8 & $[5-10]$ &  &  &  & \\
            \cline{1-8}
            Mejor & 1.3352 & 0.7489  & 16.7962 & 60 & 0.0041 & 9 & $[5-10]$ & 20 & 10 & 0.3 & 1.0\\
            \cline{1-8}
            Peor & 1.2101 & 0.8264  & 17.2791 & 70 & 0.0188 & 9 & $[5-10]$ &  &  &  & \\
        \hline
        \hline
            Promedio  & 1.2612 & 0.7934 & 17.2133 & 34.8333 & 0.0062 & 8.9 & $[5-10]$ &  &  &  & \\
            \cline{1-8}
            Mejor & 1.3348 & 0.7492  & 16.9786 & 37 & 0.0043 & 9 & $[5-10]$ & 10 & 8 & 0.2 & 0.6\\
            \cline{1-8}
            Peor & 1.2006 & 0.8329  & 19.6998 & 26 & 0.0035 & 7 & $[5-10]$ &  &  &  & \\
        \hline
        \hline
            Promedio  & 1.2578 & 0.7966 & 17.4566 & 34.8276 & 0.0087 & 8.6207 & $[5-10]$ &  &  &  & \\
            \cline{1-8}
            Mejor & 1.3306 & 0.7516  & 19.3433 & 29 & 0.0046 & 7 & $[5-10]$ & 5 & 4 & 0.2 & 0.9\\
            \cline{1-8}
            Peor & 1.0922 & 0.9156  & 16.5765 & 21 & 0.0041 & 9 & $[5-10]$ &  &  &  & \\
        \hline
        \hline
            Promedio  & 1.2758 & 0.7841 & 17.1277 & 53.9333 & 0.006 & 8.8333 & $[5-10]$ &  &  &  & \\
            \cline{1-8}
            Mejor & 1.3297 & 0.7521  & 19.443 & 71 & 0.0045 & 7 & $[5-10]$ & 25 & 10 & 0.6 & 0.9\\
            \cline{1-8}
            Peor & 1.2328 & 0.8112  & 16.8988 & 43 & 0.0045 & 9 & $[5-10]$ &  &  &  & \\
        \hline
        \hline
            Promedio  & 1.2654 & 0.7906 & 17.1834 & 46.6 & 0.0067 & 8.8 & $[5-10]$ &  &  &  & \\
            \cline{1-8}
            Mejor & 1.3286 & 0.7527  & 18.9198 & 46 & 0.0045 & 7 & $[5-10]$ & 20 & 16 & 0.5 & 1.0\\
            \cline{1-8}
            Peor & 1.2222 & 0.8182  & 19.445 & 51 & 0.0045 & 7 & $[5-10]$ &  &  &  & \\
        \hline
        \hline
            Promedio  & 1.2731 & 0.7858 & 17.1636 & 45.8667 & 0.0074 & 8.8333 & $[5-10]$ &  &  &  & \\
            \cline{1-8}
            Mejor & 1.3286 & 0.7527  & 18.9198 & 46 & 0.0045 & 7 & $[5-10]$ & 20 & 16 & 0.5 & 0.9\\
            \cline{1-8}
            Peor & 1.2298 & 0.8131  & 17.1532 & 39 & 0.0041 & 9 & $[5-10]$ &  &  &  & \\
        \hline
        \hline
            Promedio  & 1.2616 & 0.7933 & 17.6891 & 30.25 & 0.0049 & 8.5357 & $[5-10]$ &  &  &  & \\
            \cline{1-8}
            Mejor & 1.3266 & 0.7538  & 18.3825 & 48 & 0.0038 & 8 & $[5-10]$ & 5 & 4 & 0.8 & 1.0\\
            \cline{1-8}
            Peor & 1.1571 & 0.8642  & 16.9283 & 19 & 0.0041 & 9 & $[5-10]$ &  &  &  & \\
        \hline
        \end{tabular}
        \caption{Resultados de las mejores corridas de \emph{GA} hibridado para {\bf Lenna}}
        \label{tb:tablegahibimg}
    \end{center}
\end{table}


\begin{table}[h!]
    \footnotesize
    \begin{center}
        \begin{tabular}{|c|c|c|c|c|c|c|c|c|c|c|c|}
        \hline
            & {\bf FO} & {\bf DB} & $J_e$ & {\bf E} & {\bf T} & {\bf KE} & {\bf KO} & $I$ & $tt$ & $pc$ & $pm$ \\
        \hline
        \hline
            Promedio  & 1.2599 & 0.7944 & 17.0722 & 44.0333 & 0.0081 & 8.9333 & $[5-10]$ &  &  &  & \\
            \cline{1-8}
            Mejor & 1.3216 & 0.7566  & 18.3059 & 44 & 0.0078 & 8 & $[5-10]$ & 15 & 14 & 1.0 & 0.9\\
            \cline{1-8}
            Peor & 1.1753 & 0.8509  & 17.1032 & 29 & 0.0042 & 9 & $[5-10]$ &  &  &  & \\
        \hline
        \hline
            Promedio  & 1.2621 & 0.7929 & 17.0944 & 38.9667 & 0.0072 & 8.9667 & $[5-10]$ &  &  &  & \\
            \cline{1-8}
            Mejor & 1.3185 & 0.7584  & 17.8374 & 53 & 0.0037 & 8 & $[5-10]$ & 10 & 6 & 0.8 & 0.5\\
            \cline{1-8}
            Peor & 1.1854 & 0.8436  & 17.7186 & 35 & 0.0042 & 9 & $[5-10]$ &  &  &  & \\
        \hline
        \hline
            Promedio  & 1.2586 & 0.7951 & 16.9909 & 43.8621 & 0.009 & 8.9655 & $[5-10]$ &  &  &  & \\
            \cline{1-8}
            Mejor & 1.3152 & 0.7604  & 16.5793 & 37 & 0.0042 & 9 & $[5-10]$ & 15 & 14 & 1.0 & 1.0\\
            \cline{1-8}
            Peor & 1.1753 & 0.8509  & 17.1032 & 29 & 0.0043 & 9 & $[5-10]$ &  &  &  & \\
        \hline
        \hline
            Promedio  & 1.2684 & 0.7888 & 17.0053 & 51.6333 & 0.007 & 8.9667 & $[5-10]$ &  &  &  & \\
            \cline{1-8}
            Mejor & 1.3152 & 0.7604  & 16.5793 & 40 & 0.0042 & 9 & $[5-10]$ & 25 & 10 & 0.6 & 0.8\\
            \cline{1-8}
            Peor & 1.2047 & 0.8301  & 17.3055 & 47 & 0.0066 & 9 & $[5-10]$ &  &  &  & \\
        \hline
        \hline
            Promedio  & 1.2556 & 0.7971 & 17.0358 & 54.2414 & 0.0076 & 8.9655 & $[5-10]$ &  &  &  & \\
            \cline{1-8}
            Mejor & 1.3097 & 0.7635  & 16.8631 & 44 & 0.0042 & 9 & $[5-10]$ & 30 & 16 & 0.5 & 0.7\\
            \cline{1-8}
            Peor & 1.1664 & 0.8573  & 16.6878 & 45 & 0.0042 & 9 & $[5-10]$ &  &  &  & \\
        \hline
        \end{tabular}
        \caption{Continuacion resultados de las mejores corridas de \emph{GA} hibridado para {\bf Lenna}}
        \label{tb:tablegahibimgc}
    \end{center}
\end{table}


\begin{table}[h!]
    \footnotesize
    \begin{center}
        \begin{tabular}{|c|c|c|c|c|c|c|c|c|c|}
        \hline
            & {\bf FO} & {\bf DB} & $J_e$ & {\bf E} & {\bf T} & $I$ & $tt$ & $pc$ & $pm$ \\
        \hline
        \hline
            Promedio  & 2.6521 & 0.3771 & 0.5029 & 43.0 & 0.0003 &  &  &  & \\
            \cline{1-6}
            Mejor & 2.6521 & 0.3771  & 0.5029 & 43 & 0.0003 & 25 & 12 & 0.2 & 1.0\\
            \cline{1-6}
            Peor & 2.6521 & 0.3771  & 0.5029 & 43 & 0.0003 &  &  &  & \\
        \hline
        \hline
            Promedio  & 2.0137 & 0.5178 & 0.6215 & 41.6 & 0.0007 &  &  &  & \\
            \cline{1-6}
            Mejor & 2.5335 & 0.3947  & 0.5089 & 44 & 0.0011 & 30 & 16 & 0.8 & 1.0\\
            \cline{1-6}
            Peor & 1.6672 & 0.5998  & 0.6965 & 40 & 0.0009 &  &  &  & \\
        \hline
        \hline
            Promedio  & 1.9605 & 0.5171 & 0.6061 & 49.4091 & 0.0008 &  &  &  & \\
            \cline{1-6}
            Mejor & 2.2421 & 0.446  & 0.5258 & 50 & 0.0006 & 40 & 26 & 0.8 & 0.5\\
            \cline{1-6}
            Peor & 1.7628 & 0.5673  & 0.6937 & 49 & 0.0011 &  &  &  & \\
        \hline
        \hline
            Promedio  & 1.9605 & 0.5171 & 0.6061 & 49.5455 & 0.0007 &  &  &  & \\
            \cline{1-6}
            Mejor & 2.2421 & 0.446  & 0.5258 & 51 & 0.0006 & 40 & 16 & 0.5 & 1.0\\
            \cline{1-6}
            Peor & 1.7628 & 0.5673  & 0.6937 & 49 & 0.0011 &  &  &  & \\
        \hline
        \hline
            Promedio  & 1.9471 & 0.5205 & 0.61 & 49.0952 & 0.0007 &  &  &  & \\
            \cline{1-6}
            Mejor & 2.2421 & 0.446  & 0.5258 & 50 & 0.0005 & 40 & 16 & 0.5 & 0.9\\
            \cline{1-6}
            Peor & 1.7628 & 0.5673  & 0.6937 & 49 & 0.0011 &  &  &  & \\
        \hline
        \hline
            Promedio  & 1.9471 & 0.5205 & 0.61 & 49.0952 & 0.0007 &  &  &  & \\
            \cline{1-6}
            Mejor & 2.2421 & 0.446  & 0.5258 & 50 & 0.0005 & 40 & 16 & 0.5 & 0.8\\
            \cline{1-6}
            Peor & 1.7628 & 0.5673  & 0.6937 & 49 & 0.0011 &  &  &  & \\
        \hline
        \hline
            Promedio  & 1.8937 & 0.5373 & 0.6183 & 49.1304 & 0.0005 &  &  &  & \\
            \cline{1-6}
            Mejor & 2.2421 & 0.446  & 0.5258 & 49 & 0.0004 & 40 & 16 & 0.5 & 0.7\\
            \cline{1-6}
            Peor & 1.6672 & 0.5998  & 0.6965 & 50 & 0.0006 &  &  &  & \\
        \hline
        \hline
            Promedio  & 1.9471 & 0.5205 & 0.61 & 44.9524 & 0.0004 &  &  &  & \\
            \cline{1-6}
            Mejor & 2.2421 & 0.446  & 0.5258 & 44 & 0.0003 & 35 & 6 & 0.1 & 0.6\\
            \cline{1-6}
            Peor & 1.7628 & 0.5673  & 0.6937 & 46 & 0.001 &  &  &  & \\
        \hline
        \hline
            Promedio  & 1.7108 & 0.585 & 0.6656 & 51.1429 & 0.0007 &  &  &  & \\
            \cline{1-6}
            Mejor & 1.7689 & 0.5653  & 0.6245 & 50 & 0.0006 & 40 & 28 & 0.4 & 0.6\\
            \cline{1-6}
            Peor & 1.6672 & 0.5998  & 0.6965 & 52 & 0.0007 &  &  &  & \\
        \hline
        \hline
            Promedio  & 1.7281 & 0.5791 & 0.675 & 50.0 & 0.0007 &  &  &  & \\
            \cline{1-6}
            Mejor & 1.7689 & 0.5653  & 0.6245 & 50 & 0.0006 & 40 & 18 & 0.1 & 0.8\\
            \cline{1-6}
            Peor & 1.6672 & 0.5998  & 0.6965 & 50 & 0.0006 &  &  &  & \\
        \hline
        \hline
            Promedio  & 1.7281 & 0.5791 & 0.675 & 49.619 & 0.0007 &  &  &  & \\
            \cline{1-6}
            Mejor & 1.7689 & 0.5653  & 0.6245 & 50 & 0.0006 & 40 & 18 & 0.1 & 0.6\\
            \cline{1-6}
            Peor & 1.6672 & 0.5998  & 0.6965 & 49 & 0.0006 &  &  &  & \\
        \hline
        \hline
            Promedio  & 1.7656 & 0.5664 & 0.6618 & 45.5385 & 0.0006 &  &  &  & \\
            \cline{1-6}
            Mejor & 1.7689 & 0.5653  & 0.6245 & 45 & 0.0005 & 35 & 28 & 0.6 & 1.0\\
            \cline{1-6}
            Peor & 1.7628 & 0.5673  & 0.6937 & 46 & 0.001 &  &  &  & \\
        \hline
        \hline
            Promedio  & 1.7656 & 0.5664 & 0.6618 & 46.0 & 0.0006 &  &  &  & \\
            \cline{1-6}
            Mejor & 1.7689 & 0.5653  & 0.6245 & 46 & 0.0004 & 35 & 28 & 0.6 & 0.9\\
            \cline{1-6}
            Peor & 1.7628 & 0.5673  & 0.6937 & 46 & 0.001 &  &  &  & \\
        \hline
        \hline
            Promedio  & 1.7656 & 0.5664 & 0.6618 & 46.0769 & 0.0005 &  &  &  & \\
            \cline{1-6}
            Mejor & 1.7689 & 0.5653  & 0.6245 & 45 & 0.0004 & 35 & 28 & 0.5 & 1.0\\
            \cline{1-6}
            Peor & 1.7628 & 0.5673  & 0.6937 & 47 & 0.001 &  &  &  & \\
        \hline
        \hline
            Promedio  & 1.7125 & 0.5843 & 0.6904 & 40.2 & 0.0007 &  &  &  & \\
            \cline{1-6}
            Mejor & 1.7628 & 0.5673  & 0.6937 & 39 & 0.0009 & 30 & 16 & 0.9 & 0.8\\
            \cline{1-6}
            Peor & 1.679 & 0.5956  & 0.6882 & 41 & 0.0009 &  &  &  & \\
        \hline
        \end{tabular}
        \caption{Resultados de las mejores corridas de \emph{GA} no hibridado para {\bf Iris}}
        \label{tb:tablegaalgcsv}
    \end{center}
\end{table}


\begin{table}[h!]
    \footnotesize
    \begin{center}
        \begin{tabular}{|c|c|c|c|c|c|c|c|c|c|}
        \hline
            & {\bf FO} & {\bf DB} & $J_e$ & {\bf E} & {\bf T} & $I$ & $tt$ & $pc$ & $pm$ \\
        \hline
        \hline
            Promedio  & 1.7125 & 0.5843 & 0.6904 & 40.2 & 0.0007 &  &  &  & \\
            \cline{1-6}
            Mejor & 1.7628 & 0.5673  & 0.6937 & 39 & 0.0009 & 30 & 16 & 0.9 & 0.7\\
            \cline{1-6}
            Peor & 1.679 & 0.5956  & 0.6882 & 41 & 0.0009 &  &  &  & \\
        \hline
        \hline
            Promedio  & 1.7125 & 0.5843 & 0.6904 & 40.2 & 0.0007 &  &  &  & \\
            \cline{1-6}
            Mejor & 1.7628 & 0.5673  & 0.6937 & 39 & 0.0009 & 30 & 16 & 0.9 & 0.6\\
            \cline{1-6}
            Peor & 1.679 & 0.5956  & 0.6882 & 41 & 0.0009 &  &  &  & \\
        \hline
        \hline
            Promedio  & 1.709 & 0.5856 & 0.6953 & 45.3125 & 0.0003 &  &  &  & \\
            \cline{1-6}
            Mejor & 1.7628 & 0.5673  & 0.6937 & 47 & 0.001 & 35 & 4 & 0.6 & 0.8\\
            \cline{1-6}
            Peor & 1.6672 & 0.5998  & 0.6965 & 44 & 0.0003 &  &  &  & \\
        \hline
        \hline
            Promedio  & 1.709 & 0.5856 & 0.6953 & 44.0 & 0.0004 &  &  &  & \\
            \cline{1-6}
            Mejor & 1.7628 & 0.5673  & 0.6937 & 44 & 0.001 & 35 & 4 & 0.4 & 0.8\\
            \cline{1-6}
            Peor & 1.6672 & 0.5998  & 0.6965 & 44 & 0.0003 &  &  &  & \\
        \hline
        \hline
            Promedio  & 1.6672 & 0.5998 & 0.6965 & 40.0 & 0.0006 &  &  &  & \\
            \cline{1-6}
            Mejor & 1.6672 & 0.5998  & 0.6965 & 40 & 0.0009 & 30 & 16 & 0.8 & 0.9\\
            \cline{1-6}
            Peor & 1.6672 & 0.5998  & 0.6965 & 40 & 0.0009 &  &  &  & \\
        \hline
        \end{tabular}
        \caption{Continuacion resultados de las mejores corridas de \emph{GA} no hibridado para {\bf Iris}}
        \label{tb:tablegaalgcsvc}
    \end{center}
\end{table}


\begin{table}[h!]
    \footnotesize
    \begin{center}
        \begin{tabular}{|c|c|c|c|c|c|c|c|c|c|c|c|}
        \hline
            & {\bf FO} & {\bf DB} & $J_e$ & {\bf E} & {\bf T} & {\bf KE} & {\bf KO} & $I$ & $tt$ & $pc$ & $pm$ \\
        \hline
        \hline
            Promedio  & 2.6521 & 0.3771 & 0.5029 & 47.0 & 0.0 & 3.0 & 3 &  &  &  & \\
            \cline{1-8}
            Mejor & 2.6521 & 0.3771  & 0.5029 & 47 & 0.0 & 3 & 3 & 25 & 12 & 0.2 & 1.0\\
            \cline{1-8}
            Peor & 2.6521 & 0.3771  & 0.5029 & 47 & 0.0 & 3 & 3 &  &  &  & \\
        \hline
        \hline
            Promedio  & 2.0137 & 0.5178 & 0.6215 & 45.6 & 0.0 & 3.0 & 3 &  &  &  & \\
            \cline{1-8}
            Mejor & 2.5335 & 0.3947  & 0.5089 & 48 & 0.0001 & 3 & 3 & 30 & 16 & 0.8 & 1.0\\
            \cline{1-8}
            Peor & 1.6672 & 0.5998  & 0.6965 & 44 & 0.0001 & 3 & 3 &  &  &  & \\
        \hline
        \hline
            Promedio  & 1.9605 & 0.5171 & 0.6061 & 53.4091 & 0.0 & 3.0 & 3 &  &  &  & \\
            \cline{1-8}
            Mejor & 2.2421 & 0.446  & 0.5258 & 54 & 0.0 & 3 & 3 & 40 & 26 & 0.8 & 0.5\\
            \cline{1-8}
            Peor & 1.7628 & 0.5673  & 0.6937 & 53 & 0.0001 & 3 & 3 &  &  &  & \\
        \hline
        \hline
            Promedio  & 1.9605 & 0.5171 & 0.6061 & 53.5455 & 0.0 & 3.0 & 3 &  &  &  & \\
            \cline{1-8}
            Mejor & 2.2421 & 0.446  & 0.5258 & 55 & 0.0 & 3 & 3 & 40 & 16 & 0.5 & 1.0\\
            \cline{1-8}
            Peor & 1.7628 & 0.5673  & 0.6937 & 53 & 0.0001 & 3 & 3 &  &  &  & \\
        \hline
        \hline
            Promedio  & 1.9471 & 0.5205 & 0.61 & 53.0952 & 0.0 & 3.0 & 3 &  &  &  & \\
            \cline{1-8}
            Mejor & 2.2421 & 0.446  & 0.5258 & 54 & 0.0 & 3 & 3 & 40 & 16 & 0.5 & 0.9\\
            \cline{1-8}
            Peor & 1.7628 & 0.5673  & 0.6937 & 53 & 0.0001 & 3 & 3 &  &  &  & \\
        \hline
        \hline
            Promedio  & 1.9471 & 0.5205 & 0.61 & 53.0952 & 0.0 & 3.0 & 3 &  &  &  & \\
            \cline{1-8}
            Mejor & 2.2421 & 0.446  & 0.5258 & 54 & 0.0 & 3 & 3 & 40 & 16 & 0.5 & 0.8\\
            \cline{1-8}
            Peor & 1.7628 & 0.5673  & 0.6937 & 53 & 0.0001 & 3 & 3 &  &  &  & \\
        \hline
        \hline
            Promedio  & 1.8937 & 0.5373 & 0.6183 & 53.1304 & 0.0 & 3.0 & 3 &  &  &  & \\
            \cline{1-8}
            Mejor & 2.2421 & 0.446  & 0.5258 & 53 & 0.0 & 3 & 3 & 40 & 16 & 0.5 & 0.7\\
            \cline{1-8}
            Peor & 1.6672 & 0.5998  & 0.6965 & 54 & 0.0 & 3 & 3 &  &  &  & \\
        \hline
        \hline
            Promedio  & 1.9471 & 0.5205 & 0.61 & 48.9524 & 0.0 & 3.0 & 3 &  &  &  & \\
            \cline{1-8}
            Mejor & 2.2421 & 0.446  & 0.5258 & 48 & 0.0 & 3 & 3 & 35 & 6 & 0.1 & 0.6\\
            \cline{1-8}
            Peor & 1.7628 & 0.5673  & 0.6937 & 50 & 0.0001 & 3 & 3 &  &  &  & \\
        \hline
        \hline
            Promedio  & 1.7108 & 0.585 & 0.6656 & 55.1429 & 0.0 & 3.0 & 3 &  &  &  & \\
            \cline{1-8}
            Mejor & 1.7689 & 0.5653  & 0.6245 & 54 & 0.0 & 3 & 3 & 40 & 28 & 0.4 & 0.6\\
            \cline{1-8}
            Peor & 1.6672 & 0.5998  & 0.6965 & 56 & 0.0 & 3 & 3 &  &  &  & \\
        \hline
        \hline
            Promedio  & 1.7281 & 0.5791 & 0.675 & 54.0 & 0.0 & 3.0 & 3 &  &  &  & \\
            \cline{1-8}
            Mejor & 1.7689 & 0.5653  & 0.6245 & 54 & 0.0 & 3 & 3 & 40 & 18 & 0.1 & 0.8\\
            \cline{1-8}
            Peor & 1.6672 & 0.5998  & 0.6965 & 54 & 0.0 & 3 & 3 &  &  &  & \\
        \hline
        \hline
            Promedio  & 1.7281 & 0.5791 & 0.675 & 53.619 & 0.0 & 3.0 & 3 &  &  &  & \\
            \cline{1-8}
            Mejor & 1.7689 & 0.5653  & 0.6245 & 54 & 0.0 & 3 & 3 & 40 & 18 & 0.1 & 0.6\\
            \cline{1-8}
            Peor & 1.6672 & 0.5998  & 0.6965 & 53 & 0.0 & 3 & 3 &  &  &  & \\
        \hline
        \hline
            Promedio  & 1.7656 & 0.5664 & 0.6618 & 49.5385 & 0.0 & 3.0 & 3 &  &  &  & \\
            \cline{1-8}
            Mejor & 1.7689 & 0.5653  & 0.6245 & 49 & 0.0 & 3 & 3 & 35 & 28 & 0.6 & 1.0\\
            \cline{1-8}
            Peor & 1.7628 & 0.5673  & 0.6937 & 50 & 0.0001 & 3 & 3 &  &  &  & \\
        \hline
        \hline
            Promedio  & 1.7656 & 0.5664 & 0.6618 & 50.0 & 0.0 & 3.0 & 3 &  &  &  & \\
            \cline{1-8}
            Mejor & 1.7689 & 0.5653  & 0.6245 & 50 & 0.0 & 3 & 3 & 35 & 28 & 0.6 & 0.9\\
            \cline{1-8}
            Peor & 1.7628 & 0.5673  & 0.6937 & 50 & 0.0001 & 3 & 3 &  &  &  & \\
        \hline
        \hline
            Promedio  & 1.7656 & 0.5664 & 0.6618 & 50.0769 & 0.0 & 3.0 & 3 &  &  &  & \\
            \cline{1-8}
            Mejor & 1.7689 & 0.5653  & 0.6245 & 49 & 0.0 & 3 & 3 & 35 & 28 & 0.5 & 1.0\\
            \cline{1-8}
            Peor & 1.7628 & 0.5673  & 0.6937 & 51 & 0.0001 & 3 & 3 &  &  &  & \\
        \hline
        \hline
            Promedio  & 1.7125 & 0.5843 & 0.6904 & 44.2 & 0.0001 & 3.0 & 3 &  &  &  & \\
            \cline{1-8}
            Mejor & 1.7628 & 0.5673  & 0.6937 & 43 & 0.0001 & 3 & 3 & 30 & 16 & 0.9 & 0.8\\
            \cline{1-8}
            Peor & 1.679 & 0.5956  & 0.6882 & 45 & 0.0001 & 3 & 3 &  &  &  & \\
        \hline
        \end{tabular}
        \caption{Resultados de las mejores corridas de \emph{GA} hibridado para {\bf Iris}}
        \label{tb:tablegahibcsv}
    \end{center}
\end{table}


\begin{table}[h!]
    \footnotesize
    \begin{center}
        \begin{tabular}{|c|c|c|c|c|c|c|c|c|c|c|c|}
        \hline
            & {\bf FO} & {\bf DB} & $J_e$ & {\bf E} & {\bf T} & {\bf KE} & {\bf KO} & $I$ & $tt$ & $pc$ & $pm$ \\
        \hline
        \hline
            Promedio  & 1.7125 & 0.5843 & 0.6904 & 44.2 & 0.0001 & 3.0 & 3 &  &  &  & \\
            \cline{1-8}
            Mejor & 1.7628 & 0.5673  & 0.6937 & 43 & 0.0001 & 3 & 3 & 30 & 16 & 0.9 & 0.7\\
            \cline{1-8}
            Peor & 1.679 & 0.5956  & 0.6882 & 45 & 0.0001 & 3 & 3 &  &  &  & \\
        \hline
        \hline
            Promedio  & 1.7125 & 0.5843 & 0.6904 & 44.2 & 0.0001 & 3.0 & 3 &  &  &  & \\
            \cline{1-8}
            Mejor & 1.7628 & 0.5673  & 0.6937 & 43 & 0.0001 & 3 & 3 & 30 & 16 & 0.9 & 0.6\\
            \cline{1-8}
            Peor & 1.679 & 0.5956  & 0.6882 & 45 & 0.0001 & 3 & 3 &  &  &  & \\
        \hline
        \hline
            Promedio  & 1.709 & 0.5856 & 0.6953 & 49.3125 & 0.0 & 3.0 & 3 &  &  &  & \\
            \cline{1-8}
            Mejor & 1.7628 & 0.5673  & 0.6937 & 51 & 0.0001 & 3 & 3 & 35 & 4 & 0.6 & 0.8\\
            \cline{1-8}
            Peor & 1.6672 & 0.5998  & 0.6965 & 48 & 0.0 & 3 & 3 &  &  &  & \\
        \hline
        \hline
            Promedio  & 1.709 & 0.5856 & 0.6953 & 48.0 & 0.0 & 3.0 & 3 &  &  &  & \\
            \cline{1-8}
            Mejor & 1.7628 & 0.5673  & 0.6937 & 48 & 0.0001 & 3 & 3 & 35 & 4 & 0.4 & 0.8\\
            \cline{1-8}
            Peor & 1.6672 & 0.5998  & 0.6965 & 48 & 0.0 & 3 & 3 &  &  &  & \\
        \hline
        \hline
            Promedio  & 1.6672 & 0.5998 & 0.6965 & 44.0 & 0.0 & 3.0 & 3 &  &  &  & \\
            \cline{1-8}
            Mejor & 1.6672 & 0.5998  & 0.6965 & 44 & 0.0001 & 3 & 3 & 30 & 16 & 0.8 & 0.9\\
            \cline{1-8}
            Peor & 1.6672 & 0.5998  & 0.6965 & 44 & 0.0001 & 3 & 3 &  &  &  & \\
        \hline
        \end{tabular}
        \caption{Continuacion resultados de las mejores corridas de \emph{GA} hibridado para {\bf Iris}}
        \label{tb:tablegahibcsvc}
    \end{center}
\end{table}


\section{Optimizador de Enjambre de Partículas (\emph{PSOH})}

    \subsection{\emph{NPSOH}}\label{sect:anpso}

        Las variables del \emph{NPSOH}(\ref{sect:inpso}) son las siguientes:
        \begin{itemize}
            \item $I$: tamaño de la población. Se varió su valor en el rango
        $[5, 10, \cdots, 40]$.
            \item $c_1$: factor de la componente cognitiva. Se varió su valor en el
        $[0.5; 0.8; \cdots; 2.0]$
            \item $c_2$: factor de la componente social. Se varió en el rango
        $[0.5; 0.8; \cdots; 2.0]$
            \item $W$: peso inercial de la partícula. Se varió su valor en el rango
        rango $[0.5; 0.8; 1.1]$
            \item $vmx$: velocidad máxima de la partícula. Se varió su valor en el rango
        $[0.5; 0.7; 0.9]$.
        \end{itemize}

        \begin{table}[h!]
    \footnotesize
    \begin{center}
        \begin{tabular}{|c|c|c|c|c|c|c|c|c|c|c|}
        \hline
            & {\bf FO} & {\bf DB} & $J_e$ & {\bf E} & {\bf T} & $I$ & W & $c_1$ & $c_2$ & $vmx$ \\
        \hline
        \hline
            Promedio  & 0.6074 & 3.5347 & 112.2039 & 259.0 & 0.503 &  &  &  &  & \\
            \cline{1-6}
            Mejor & 0.7166 & 2.75  & 85.1979 & 245 & 0.4754 & 35 & 0.8 & 0.8 & 2.0 & 0.5\\
            \cline{1-6}
            Peor & 0.4485 & 3.7306  & 111.479 & 245 & 0.4722 &  &  &  &  & \\
        \hline
        \hline
            Promedio  & 0.6027 & 3.578 & 114.2426 & 257.8333 & 0.501 &  &  &  &  & \\
            \cline{1-6}
            Mejor & 0.7166 & 2.75  & 85.1979 & 245 & 0.4692 & 35 & 0.5 & 1.1 & 0.5 & 0.9\\
            \cline{1-6}
            Peor & 0.4485 & 3.7306  & 111.479 & 245 & 0.4773 &  &  &  &  & \\
        \hline
        \hline
            Promedio  & 0.594 & 3.69 & 116.6383 & 314.6667 & 0.6107 &  &  &  &  & \\
            \cline{1-6}
            Mejor & 0.7224 & 4.2337  & 136.2445 & 240 & 0.4681 & 40 & 1.1 & 1.7 & 1.7 & 0.9\\
            \cline{1-6}
            Peor & 0.5167 & 5.4688  & 172.6691 & 240 & 0.4701 &  &  &  &  & \\
        \hline
        \hline
            Promedio  & 0.6004 & 3.7573 & 120.0457 & 243.8333 & 0.4807 &  &  &  &  & \\
            \cline{1-6}
            Mejor & 0.7343 & 4.3673  & 122.0613 & 210 & 0.4033 & 35 & 0.8 & 1.7 & 1.7 & 0.5\\
            \cline{1-6}
            Peor & 0.4902 & 3.5687  & 98.2432 & 420 & 0.8114 &  &  &  &  & \\
        \hline
        \hline
            Promedio  & 0.6051 & 3.6738 & 117.4314 & 224.0 & 0.44 &  &  &  &  & \\
            \cline{1-6}
            Mejor & 0.7458 & 3.8823  & 128.5931 & 240 & 0.4621 & 30 & 0.8 & 1.1 & 1.7 & 0.9\\
            \cline{1-6}
            Peor & 0.5159 & 4.5416  & 150.6795 & 150 & 0.2972 &  &  &  &  & \\
        \hline
        \hline
            Promedio  & 0.5998 & 3.6531 & 117.7363 & 272.0 & 0.5335 &  &  &  &  & \\
            \cline{1-6}
            Mejor & 0.7535 & 3.8339  & 115.4865 & 240 & 0.4792 & 40 & 0.5 & 0.8 & 1.4 & 0.5\\
            \cline{1-6}
            Peor & 0.4818 & 4.0052  & 119.2749 & 400 & 0.7641 &  &  &  &  & \\
        \hline
        \hline
            Promedio  & 0.5913 & 3.4892 & 111.7419 & 284.0 & 0.5545 &  &  &  &  & \\
            \cline{1-6}
            Mejor & 0.7535 & 3.8339  & 115.4865 & 240 & 0.48 & 40 & 0.5 & 1.4 & 0.8 & 0.7\\
            \cline{1-6}
            Peor & 0.4966 & 2.0707  & 75.2603 & 240 & 0.4698 &  &  &  &  & \\
        \hline
        \hline
            Promedio  & 0.617 & 3.8375 & 114.9038 & 227.0 & 0.4427 &  &  &  &  & \\
            \cline{1-6}
            Mejor & 0.7632 & 2.9613  & 85.456 & 180 & 0.3564 & 30 & 0.8 & 2.0 & 1.4 & 0.7\\
            \cline{1-6}
            Peor & 0.5213 & 3.9071  & 119.2842 & 150 & 0.3019 &  &  &  &  & \\
        \hline
        \hline
            Promedio  & 0.6059 & 3.5132 & 110.87 & 223.0 & 0.437 &  &  &  &  & \\
            \cline{1-6}
            Mejor & 0.7716 & 4.0813  & 143.7277 & 180 & 0.356 & 30 & 1.1 & 0.5 & 0.5 & 0.7\\
            \cline{1-6}
            Peor & 0.531 & 4.0323  & 123.7195 & 330 & 0.6285 &  &  &  &  & \\
        \hline
        \hline
            Promedio  & 0.6154 & 3.6019 & 112.8467 & 243.0 & 0.4749 &  &  &  &  & \\
            \cline{1-6}
            Mejor & 0.7716 & 4.0813  & 143.7277 & 180 & 0.361 & 30 & 1.1 & 0.5 & 0.5 & 0.5\\
            \cline{1-6}
            Peor & 0.531 & 4.0323  & 123.7195 & 330 & 0.6338 &  &  &  &  & \\
        \hline
        \hline
            Promedio  & 0.6347 & 3.5997 & 118.2336 & 167.5 & 0.3353 &  &  &  &  & \\
            \cline{1-6}
            Mejor & 0.7826 & 3.0617  & 120.3298 & 125 & 0.2879 & 25 & 0.5 & 0.5 & 2.0 & 0.5\\
            \cline{1-6}
            Peor & 0.517 & 4.4256  & 143.4976 & 175 & 0.3337 &  &  &  &  & \\
        \hline
        \hline
            Promedio  & 0.6187 & 3.6273 & 115.6737 & 238.0 & 0.4669 &  &  &  &  & \\
            \cline{1-6}
            Mejor & 0.7988 & 2.8773  & 87.5435 & 175 & 0.3476 & 35 & 0.5 & 1.1 & 1.1 & 0.7\\
            \cline{1-6}
            Peor & 0.5012 & 3.4334  & 107.2576 & 280 & 0.5412 &  &  &  &  & \\
        \hline
        \hline
            Promedio  & 0.6198 & 3.6931 & 114.3386 & 176.6667 & 0.3472 &  &  &  &  & \\
            \cline{1-6}
            Mejor & 0.8355 & 6.0653  & 109.9886 & 100 & 0.2063 & 25 & 1.1 & 1.4 & 0.8 & 0.7\\
            \cline{1-6}
            Peor & 0.4583 & 4.5459  & 135.1973 & 125 & 0.2481 &  &  &  &  & \\
        \hline
        \hline
            Promedio  & 0.6344 & 3.7053 & 114.5877 & 166.6667 & 0.3295 &  &  &  &  & \\
            \cline{1-6}
            Mejor & 0.8435 & 4.0008  & 112.1406 & 150 & 0.2895 & 25 & 0.5 & 0.5 & 1.7 & 0.7\\
            \cline{1-6}
            Peor & 0.5303 & 2.6343  & 85.9846 & 125 & 0.2585 &  &  &  &  & \\
        \hline
        \hline
            Promedio  & 0.5943 & 3.6214 & 116.951 & 260.0 & 0.5172 &  &  &  &  & \\
            \cline{1-6}
            Mejor & 0.8554 & 4.0482  & 109.0654 & 200 & 0.4184 & 40 & 0.8 & 1.4 & 0.8 & 0.7\\
            \cline{1-6}
            Peor & 0.4581 & 3.0394  & 100.3034 & 320 & 0.617 &  &  &  &  & \\
        \hline
        \end{tabular}
        \caption{Resultados de las mejores corridas de \emph{NPSO} no hibridado para {\bf Lenna}}
        \label{tb:tablepsoalgimg}
    \end{center}
\end{table}


\begin{table}[h!]
    \footnotesize
    \begin{center}
        \begin{tabular}{|c|c|c|c|c|c|c|c|c|c|c|}
        \hline
            & {\bf FO} & {\bf DB} & $J_e$ & {\bf E} & {\bf T} & $I$ & W & $c_1$ & $c_2$ & $vmx$ \\
        \hline
        \hline
            Promedio  & 0.6085 & 3.7936 & 119.0667 & 266.6667 & 0.524 &  &  &  &  & \\
            \cline{1-6}
            Mejor & 0.8554 & 4.0482  & 109.0654 & 200 & 0.4305 & 40 & 0.5 & 0.8 & 1.4 & 0.7\\
            \cline{1-6}
            Peor & 0.5008 & 4.7481  & 142.2292 & 240 & 0.4646 &  &  &  &  & \\
        \hline
        \hline
            Promedio  & 0.6698 & 3.4729 & 112.7044 & 99.5 & 0.2011 &  &  &  &  & \\
            \cline{1-6}
            Mejor & 0.8627 & 3.7729  & 104.8662 & 75 & 0.1475 & 15 & 1.1 & 2.0 & 1.1 & 0.7\\
            \cline{1-6}
            Peor & 0.5077 & 3.6008  & 107.3644 & 120 & 0.2463 &  &  &  &  & \\
        \hline
        \hline
            Promedio  & 0.6803 & 3.6715 & 112.4042 & 100.0 & 0.1992 &  &  &  &  & \\
            \cline{1-6}
            Mejor & 0.9182 & 3.2981  & 111.0983 & 105 & 0.2015 & 15 & 1.1 & 2.0 & 1.1 & 0.5\\
            \cline{1-6}
            Peor & 0.5522 & 3.1622  & 104.2767 & 75 & 0.1477 &  &  &  &  & \\
        \hline
        \hline
            Promedio  & 0.6721 & 3.6445 & 114.4676 & 99.0 & 0.1998 &  &  &  &  & \\
            \cline{1-6}
            Mejor & 0.9182 & 3.2981  & 111.0983 & 105 & 0.2013 & 15 & 1.1 & 2.0 & 0.8 & 0.9\\
            \cline{1-6}
            Peor & 0.5522 & 3.1622  & 104.2767 & 75 & 0.1557 &  &  &  &  & \\
        \hline
        \hline
            Promedio  & 0.6936 & 3.6512 & 115.5525 & 75.0 & 0.1497 &  &  &  &  & \\
            \cline{1-6}
            Mejor & 1.0113 & 3.5483  & 111.1966 & 50 & 0.096 & 10 & 0.8 & 2.0 & 1.1 & 0.9\\
            \cline{1-6}
            Peor & 0.5261 & 3.2999  & 96.0397 & 50 & 0.1017 &  &  &  &  & \\
        \hline
        \end{tabular}
        \caption{Continuacion resultados de las mejores corridas de \emph{NPSO} no hibridado para {\bf Lenna}}
        \label{tb:tablepsoalgimgc}
    \end{center}
\end{table}

        
        \begin{table}[h!]
    \footnotesize
    \begin{center}
        \begin{tabular}{|c|c|c|c|c|c|c|c|c|c|c|c|c|}
        \hline
            & {\bf FO} & {\bf DB} & $J_e$ & {\bf E} & {\bf T} & {\bf KE} & {\bf KO} & $I$ & W & $c_1$ & $c_2$ & $vmx$ \\
        \hline
        \hline
            Promedio  & 1.2071 & 0.8297 & 17.5799 & 258.1 & 0.0184 & 9.0 & $[5-10]$ &  &  &  &  & \\
            \cline{1-8}
            Mejor & 1.2968 & 0.7711  & 17.3749 & 220 & 0.0123 & 9 & $[5-10]$ & 35 & 0.8 & 1.7 & 1.7 & 0.5\\
            \cline{1-8}
            Peor & 1.1113 & 0.8999  & 19.3463 & 180 & 0.0057 & 9 & $[5-10]$ &  &  &  &  & \\
        \hline
        \hline
            Promedio  & 1.2063 & 0.8303 & 17.5049 & 181.8 & 0.0184 & 9.0 & $[5-10]$ &  &  &  &  & \\
            \cline{1-8}
            Mejor & 1.2891 & 0.7758  & 17.2038 & 183 & 0.0095 & 9 & $[5-10]$ & 25 & 0.5 & 0.5 & 2.0 & 0.5\\
            \cline{1-8}
            Peor & 1.0617 & 0.9419  & 18.4034 & 131 & 0.0068 & 9 & $[5-10]$ &  &  &  &  & \\
        \hline
        \hline
            Promedio  & 1.1984 & 0.8357 & 17.611 & 114.9 & 0.0192 & 9.0 & $[5-10]$ &  &  &  &  & \\
            \cline{1-8}
            Mejor & 1.2863 & 0.7774  & 17.3962 & 95 & 0.0056 & 9 & $[5-10]$ & 15 & 1.1 & 2.0 & 1.1 & 0.5\\
            \cline{1-8}
            Peor & 1.0864 & 0.9205  & 18.6775 & 125 & 0.0054 & 9 & $[5-10]$ &  &  &  &  & \\
        \hline
        \hline
            Promedio  & 1.1939 & 0.8389 & 17.6423 & 111.9667 & 0.0163 & 9.0 & $[5-10]$ &  &  &  &  & \\
            \cline{1-8}
            Mejor & 1.2863 & 0.7774  & 17.3962 & 95 & 0.0055 & 9 & $[5-10]$ & 15 & 1.1 & 2.0 & 0.8 & 0.9\\
            \cline{1-8}
            Peor & 1.0864 & 0.9205  & 18.6775 & 125 & 0.0055 & 9 & $[5-10]$ &  &  &  &  & \\
        \hline
        \hline
            Promedio  & 1.2011 & 0.8344 & 17.653 & 329.5333 & 0.0189 & 9.0 & $[5-10]$ &  &  &  &  & \\
            \cline{1-8}
            Mejor & 1.2845 & 0.7785  & 17.9677 & 325 & 0.0055 & 9 & $[5-10]$ & 40 & 1.1 & 1.7 & 1.7 & 0.9\\
            \cline{1-8}
            Peor & 0.9938 & 1.0063  & 18.1528 & 245 & 0.0056 & 9 & $[5-10]$ &  &  &  &  & \\
        \hline
        \hline
            Promedio  & 1.1869 & 0.8442 & 17.7796 & 278.8 & 0.0152 & 9.0 & $[5-10]$ &  &  &  &  & \\
            \cline{1-8}
            Mejor & 1.2829 & 0.7795  & 16.8104 & 213 & 0.0168 & 9 & $[5-10]$ & 40 & 0.5 & 0.8 & 1.4 & 0.7\\
            \cline{1-8}
            Peor & 1.0729 & 0.9321  & 18.0326 & 330 & 0.0122 & 9 & $[5-10]$ &  &  &  &  & \\
        \hline
        \hline
            Promedio  & 1.1861 & 0.8453 & 17.7004 & 249.4333 & 0.0142 & 9.0 & $[5-10]$ &  &  &  &  & \\
            \cline{1-8}
            Mejor & 1.2822 & 0.7799  & 17.2126 & 327 & 0.015 & 9 & $[5-10]$ & 35 & 0.5 & 1.1 & 1.1 & 0.7\\
            \cline{1-8}
            Peor & 1.0616 & 0.942  & 18.4608 & 250 & 0.0056 & 9 & $[5-10]$ &  &  &  &  & \\
        \hline
        \hline
            Promedio  & 1.1881 & 0.8437 & 17.8916 & 178.3 & 0.0144 & 9.0 & $[5-10]$ &  &  &  &  & \\
            \cline{1-8}
            Mejor & 1.2774 & 0.7829  & 18.7638 & 130 & 0.0054 & 9 & $[5-10]$ & 25 & 0.5 & 0.5 & 1.7 & 0.7\\
            \cline{1-8}
            Peor & 1.0182 & 0.9822  & 19.4121 & 205 & 0.0054 & 9 & $[5-10]$ &  &  &  &  & \\
        \hline
        \hline
            Promedio  & 1.1786 & 0.8513 & 17.7658 & 191.4333 & 0.0187 & 9.0 & $[5-10]$ &  &  &  &  & \\
            \cline{1-8}
            Mejor & 1.2763 & 0.7835  & 17.2461 & 165 & 0.019 & 9 & $[5-10]$ & 25 & 1.1 & 1.4 & 0.8 & 0.7\\
            \cline{1-8}
            Peor & 1.0247 & 0.9759  & 19.4323 & 182 & 0.0081 & 9 & $[5-10]$ &  &  &  &  & \\
        \hline
        \hline
            Promedio  & 1.1716 & 0.8559 & 17.828 & 270.9667 & 0.0167 & 9.0 & $[5-10]$ &  &  &  &  & \\
            \cline{1-8}
            Mejor & 1.2763 & 0.7835  & 17.3136 & 256 & 0.0138 & 9 & $[5-10]$ & 35 & 0.5 & 1.1 & 0.5 & 0.9\\
            \cline{1-8}
            Peor & 1.0325 & 0.9686  & 18.9037 & 254 & 0.0106 & 9 & $[5-10]$ &  &  &  &  & \\
        \hline
        \hline
            Promedio  & 1.1789 & 0.85 & 17.9633 & 236.4 & 0.0157 & 9.0 & $[5-10]$ &  &  &  &  & \\
            \cline{1-8}
            Mejor & 1.2754 & 0.7841  & 17.2208 & 254 & 0.018 & 9 & $[5-10]$ & 30 & 0.8 & 1.1 & 1.7 & 0.9\\
            \cline{1-8}
            Peor & 1.0697 & 0.9348  & 20.1761 & 155 & 0.01 & 9 & $[5-10]$ &  &  &  &  & \\
        \hline
        \hline
            Promedio  & 1.1948 & 0.8383 & 17.6074 & 88.6667 & 0.0176 & 9.0 & $[5-10]$ &  &  &  &  & \\
            \cline{1-8}
            Mejor & 1.275 & 0.7843  & 17.2122 & 57 & 0.0079 & 9 & $[5-10]$ & 10 & 0.8 & 2.0 & 1.1 & 0.9\\
            \cline{1-8}
            Peor & 1.0818 & 0.9244  & 17.9611 & 89 & 0.0109 & 9 & $[5-10]$ &  &  &  &  & \\
        \hline
        \hline
            Promedio  & 1.1796 & 0.8497 & 17.6705 & 239.7 & 0.022 & 9.0 & $[5-10]$ &  &  &  &  & \\
            \cline{1-8}
            Mejor & 1.2663 & 0.7897  & 17.9028 & 278 & 0.0093 & 9 & $[5-10]$ & 30 & 1.1 & 0.5 & 0.5 & 0.7\\
            \cline{1-8}
            Peor & 1.0 & 1.0  & 19.197 & 157 & 0.0082 & 9 & $[5-10]$ &  &  &  &  & \\
        \hline
        \hline
            Promedio  & 1.1825 & 0.8478 & 17.8413 & 271.6333 & 0.0144 & 9.0 & $[5-10]$ &  &  &  &  & \\
            \cline{1-8}
            Mejor & 1.2652 & 0.7904  & 16.9427 & 329 & 0.0112 & 9 & $[5-10]$ & 40 & 0.8 & 1.4 & 0.8 & 0.7\\
            \cline{1-8}
            Peor & 1.0313 & 0.9696  & 19.3459 & 325 & 0.0054 & 9 & $[5-10]$ &  &  &  &  & \\
        \hline
        \hline
            Promedio  & 1.1783 & 0.851 & 17.8211 & 255.1667 & 0.0152 & 9.0 & $[5-10]$ &  &  &  &  & \\
            \cline{1-8}
            Mejor & 1.2617 & 0.7926  & 17.6096 & 252 & 0.0154 & 9 & $[5-10]$ & 30 & 1.1 & 0.5 & 0.5 & 0.5\\
            \cline{1-8}
            Peor & 1.0 & 1.0  & 19.197 & 157 & 0.0082 & 9 & $[5-10]$ &  &  &  &  & \\
        \hline
        \end{tabular}
        \caption{Resultados de las mejores corridas de \emph{NPSO} hibridado para {\bf Lenna}}
        \label{tb:tablepsohibimg}
    \end{center}
\end{table}


\begin{table}[h!]
    \footnotesize
    \begin{center}
        \begin{tabular}{|c|c|c|c|c|c|c|c|c|c|c|c|c|}
        \hline
            & {\bf FO} & {\bf DB} & $J_e$ & {\bf E} & {\bf T} & {\bf KE} & {\bf KO} & $I$ & W & $c_1$ & $c_2$ & $vmx$ \\
        \hline
        \hline
            Promedio  & 1.1912 & 0.841 & 17.7963 & 238.0 & 0.0138 & 9.0 & $[5-10]$ &  &  &  &  & \\
            \cline{1-8}
            Mejor & 1.2586 & 0.7945  & 17.0288 & 292 & 0.0286 & 9 & $[5-10]$ & 30 & 0.8 & 2.0 & 1.4 & 0.7\\
            \cline{1-8}
            Peor & 1.0477 & 0.9545  & 17.2479 & 156 & 0.0069 & 9 & $[5-10]$ &  &  &  &  & \\
        \hline
        \hline
            Promedio  & 1.1936 & 0.8391 & 17.6011 & 286.1667 & 0.018 & 9.0 & $[5-10]$ &  &  &  &  & \\
            \cline{1-8}
            Mejor & 1.2518 & 0.7988  & 17.036 & 215 & 0.0193 & 9 & $[5-10]$ & 40 & 0.5 & 0.8 & 1.4 & 0.5\\
            \cline{1-8}
            Peor & 1.094 & 0.9141  & 17.4741 & 247 & 0.0626 & 9 & $[5-10]$ &  &  &  &  & \\
        \hline
        \hline
            Promedio  & 1.1887 & 0.8423 & 17.6782 & 295.3333 & 0.0144 & 9.0 & $[5-10]$ &  &  &  &  & \\
            \cline{1-8}
            Mejor & 1.251 & 0.7993  & 17.0162 & 220 & 0.0262 & 9 & $[5-10]$ & 40 & 0.5 & 1.4 & 0.8 & 0.7\\
            \cline{1-8}
            Peor & 1.0992 & 0.9097  & 17.8992 & 205 & 0.0055 & 9 & $[5-10]$ &  &  &  &  & \\
        \hline
        \hline
            Promedio  & 1.1926 & 0.84 & 17.8407 & 110.6 & 0.0139 & 9.0 & $[5-10]$ &  &  &  &  & \\
            \cline{1-8}
            Mejor & 1.2476 & 0.8016  & 17.4607 & 99 & 0.0109 & 9 & $[5-10]$ & 15 & 1.1 & 2.0 & 1.1 & 0.7\\
            \cline{1-8}
            Peor & 1.0406 & 0.961  & 19.7912 & 95 & 0.0054 & 9 & $[5-10]$ &  &  &  &  & \\
        \hline
        \hline
            Promedio  & 1.1837 & 0.8468 & 17.7679 & 273.7667 & 0.0188 & 9.0 & $[5-10]$ &  &  &  &  & \\
            \cline{1-8}
            Mejor & 1.2465 & 0.8022  & 17.1374 & 378 & 0.0361 & 9 & $[5-10]$ & 35 & 0.8 & 0.8 & 2.0 & 0.5\\
            \cline{1-8}
            Peor & 1.0325 & 0.9686  & 18.9037 & 254 & 0.0106 & 9 & $[5-10]$ &  &  &  &  & \\
        \hline
        \end{tabular}
        \caption{Continuacion resultados de las mejores corridas de \emph{NPSO} hibridado para {\bf Lenna}}
        \label{tb:tablepsohibimgc}
    \end{center}
\end{table}


        \begin{table}[h!]
    \footnotesize
    \begin{center}
        \begin{tabular}{|c|c|c|c|c|c|c|c|c|c|c|}
        \hline
            & {\bf FO} & {\bf DB} & $J_e$ & {\bf E} & {\bf T} & $I$ & W & $c_1$ & $c_2$ & $vmx$ \\
        \hline
        \hline
            Promedio  & 0.3594 & 2.266 & 3.799 & 130.8333 & 0.0025 &  &  &  &  & \\
            \cline{1-6}
            Mejor & 0.381 & 2.7145  & 4.1582 & 100 & 0.0013 & 25 & 1.1 & 1.7 & 0.5 & 0.5\\
            \cline{1-6}
            Peor & 0.3319 & 2.4244  & 3.4813 & 100 & 0.0017 &  &  &  &  & \\
        \hline
        \hline
            Promedio  & 0.3599 & 2.2684 & 3.7962 & 131.6667 & 0.0025 &  &  &  &  & \\
            \cline{1-6}
            Mejor & 0.381 & 2.7145  & 4.1582 & 100 & 0.0013 & 25 & 1.1 & 1.4 & 2.0 & 0.9\\
            \cline{1-6}
            Peor & 0.3319 & 2.4244  & 3.4813 & 100 & 0.0017 &  &  &  &  & \\
        \hline
        \hline
            Promedio  & 0.3599 & 2.2684 & 3.7962 & 131.6667 & 0.0024 &  &  &  &  & \\
            \cline{1-6}
            Mejor & 0.381 & 2.7145  & 4.1582 & 100 & 0.0012 & 25 & 1.1 & 1.4 & 2.0 & 0.7\\
            \cline{1-6}
            Peor & 0.3319 & 2.4244  & 3.4813 & 100 & 0.0018 &  &  &  &  & \\
        \hline
        \hline
            Promedio  & 0.3599 & 2.2684 & 3.7962 & 131.6667 & 0.0022 &  &  &  &  & \\
            \cline{1-6}
            Mejor & 0.381 & 2.7145  & 4.1582 & 100 & 0.0012 & 25 & 1.1 & 1.4 & 2.0 & 0.5\\
            \cline{1-6}
            Peor & 0.3319 & 2.4244  & 3.4813 & 100 & 0.0017 &  &  &  &  & \\
        \hline
        \hline
            Promedio  & 0.3594 & 2.3084 & 3.8043 & 127.5 & 0.0022 &  &  &  &  & \\
            \cline{1-6}
            Mejor & 0.381 & 2.7145  & 4.1582 & 100 & 0.0012 & 25 & 1.1 & 1.4 & 1.7 & 0.9\\
            \cline{1-6}
            Peor & 0.3319 & 2.4244  & 3.4813 & 100 & 0.0017 &  &  &  &  & \\
        \hline
        \hline
            Promedio  & 0.3594 & 2.3084 & 3.8043 & 127.5 & 0.0021 &  &  &  &  & \\
            \cline{1-6}
            Mejor & 0.381 & 2.7145  & 4.1582 & 100 & 0.001 & 25 & 1.1 & 1.4 & 1.7 & 0.7\\
            \cline{1-6}
            Peor & 0.3319 & 2.4244  & 3.4813 & 100 & 0.0017 &  &  &  &  & \\
        \hline
        \hline
            Promedio  & 0.3594 & 2.3084 & 3.8043 & 127.5 & 0.0021 &  &  &  &  & \\
            \cline{1-6}
            Mejor & 0.381 & 2.7145  & 4.1582 & 100 & 0.0012 & 25 & 1.1 & 1.4 & 1.7 & 0.5\\
            \cline{1-6}
            Peor & 0.3319 & 2.4244  & 3.4813 & 100 & 0.0016 &  &  &  &  & \\
        \hline
        \hline
            Promedio  & 0.3594 & 2.3084 & 3.8043 & 127.5 & 0.002 &  &  &  &  & \\
            \cline{1-6}
            Mejor & 0.381 & 2.7145  & 4.1582 & 100 & 0.001 & 25 & 1.1 & 1.4 & 1.4 & 0.9\\
            \cline{1-6}
            Peor & 0.3319 & 2.4244  & 3.4813 & 100 & 0.0015 &  &  &  &  & \\
        \hline
        \hline
            Promedio  & 0.3594 & 2.2767 & 3.769 & 130.0 & 0.0021 &  &  &  &  & \\
            \cline{1-6}
            Mejor & 0.381 & 2.7145  & 4.1582 & 100 & 0.0015 & 25 & 1.1 & 1.4 & 1.4 & 0.7\\
            \cline{1-6}
            Peor & 0.3319 & 2.4244  & 3.4813 & 100 & 0.0015 &  &  &  &  & \\
        \hline
        \hline
            Promedio  & 0.3594 & 2.2767 & 3.769 & 130.0 & 0.0021 &  &  &  &  & \\
            \cline{1-6}
            Mejor & 0.381 & 2.7145  & 4.1582 & 100 & 0.0015 & 25 & 1.1 & 1.4 & 1.4 & 0.5\\
            \cline{1-6}
            Peor & 0.3319 & 2.4244  & 3.4813 & 100 & 0.0016 &  &  &  &  & \\
        \hline
        \hline
            Promedio  & 0.3604 & 2.2736 & 3.7969 & 130.8333 & 0.002 &  &  &  &  & \\
            \cline{1-6}
            Mejor & 0.381 & 2.7145  & 4.1582 & 100 & 0.0013 & 25 & 1.1 & 1.4 & 1.1 & 0.9\\
            \cline{1-6}
            Peor & 0.3319 & 2.4244  & 3.4813 & 100 & 0.0014 &  &  &  &  & \\
        \hline
        \hline
            Promedio  & 0.3604 & 2.2736 & 3.7969 & 130.8333 & 0.0019 &  &  &  &  & \\
            \cline{1-6}
            Mejor & 0.381 & 2.7145  & 4.1582 & 100 & 0.0013 & 25 & 1.1 & 1.4 & 1.1 & 0.7\\
            \cline{1-6}
            Peor & 0.3319 & 2.4244  & 3.4813 & 100 & 0.0014 &  &  &  &  & \\
        \hline
        \hline
            Promedio  & 0.3604 & 2.2736 & 3.7969 & 130.8333 & 0.0018 &  &  &  &  & \\
            \cline{1-6}
            Mejor & 0.381 & 2.7145  & 4.1582 & 100 & 0.0014 & 25 & 1.1 & 1.4 & 1.1 & 0.5\\
            \cline{1-6}
            Peor & 0.3319 & 2.4244  & 3.4813 & 100 & 0.0013 &  &  &  &  & \\
        \hline
        \hline
            Promedio  & 0.3604 & 2.2736 & 3.7969 & 130.8333 & 0.0018 &  &  &  &  & \\
            \cline{1-6}
            Mejor & 0.381 & 2.7145  & 4.1582 & 100 & 0.0014 & 25 & 1.1 & 1.4 & 0.8 & 0.9\\
            \cline{1-6}
            Peor & 0.3319 & 2.4244  & 3.4813 & 100 & 0.0013 &  &  &  &  & \\
        \hline
        \hline
            Promedio  & 0.3604 & 2.2736 & 3.7969 & 130.8333 & 0.0017 &  &  &  &  & \\
            \cline{1-6}
            Mejor & 0.381 & 2.7145  & 4.1582 & 100 & 0.0013 & 25 & 1.1 & 1.4 & 0.8 & 0.7\\
            \cline{1-6}
            Peor & 0.3319 & 2.4244  & 3.4813 & 100 & 0.0012 &  &  &  &  & \\
        \hline
        \end{tabular}
        \caption{Resultados de las mejores corridas de \emph{NPSO} no hibridado para {\bf Iris}}
        \label{tb:tablepsoalgcsv}
    \end{center}
\end{table}


\begin{table}[h!]
    \footnotesize
    \begin{center}
        \begin{tabular}{|c|c|c|c|c|c|c|c|c|c|c|}
        \hline
            & {\bf FO} & {\bf DB} & $J_e$ & {\bf E} & {\bf T} & $I$ & W & $c_1$ & $c_2$ & $vmx$ \\
        \hline
        \hline
            Promedio  & 0.3604 & 2.2736 & 3.7969 & 130.8333 & 0.0016 &  &  &  &  & \\
            \cline{1-6}
            Mejor & 0.381 & 2.7145  & 4.1582 & 100 & 0.0012 & 25 & 1.1 & 1.4 & 0.8 & 0.5\\
            \cline{1-6}
            Peor & 0.3319 & 2.4244  & 3.4813 & 100 & 0.0012 &  &  &  &  & \\
        \hline
        \hline
            Promedio  & 0.3604 & 2.2736 & 3.7969 & 130.8333 & 0.0016 &  &  &  &  & \\
            \cline{1-6}
            Mejor & 0.381 & 2.7145  & 4.1582 & 100 & 0.0012 & 25 & 1.1 & 1.4 & 0.5 & 0.9\\
            \cline{1-6}
            Peor & 0.3319 & 2.4244  & 3.4813 & 100 & 0.0012 &  &  &  &  & \\
        \hline
        \hline
            Promedio  & 0.3604 & 2.2736 & 3.7969 & 130.8333 & 0.0014 &  &  &  &  & \\
            \cline{1-6}
            Mejor & 0.381 & 2.7145  & 4.1582 & 100 & 0.0012 & 25 & 1.1 & 1.4 & 0.5 & 0.7\\
            \cline{1-6}
            Peor & 0.3319 & 2.4244  & 3.4813 & 100 & 0.001 &  &  &  &  & \\
        \hline
        \hline
            Promedio  & 0.3604 & 2.2736 & 3.7969 & 130.8333 & 0.0013 &  &  &  &  & \\
            \cline{1-6}
            Mejor & 0.381 & 2.7145  & 4.1582 & 100 & 0.001 & 25 & 1.1 & 1.4 & 0.5 & 0.5\\
            \cline{1-6}
            Peor & 0.3319 & 2.4244  & 3.4813 & 100 & 0.001 &  &  &  &  & \\
        \hline
        \hline
            Promedio  & 0.3604 & 2.2736 & 3.7969 & 130.8333 & 0.0014 &  &  &  &  & \\
            \cline{1-6}
            Mejor & 0.381 & 2.7145  & 4.1582 & 100 & 0.001 & 25 & 1.1 & 1.1 & 2.0 & 0.9\\
            \cline{1-6}
            Peor & 0.3319 & 2.4244  & 3.4813 & 100 & 0.001 &  &  &  &  & \\
        \hline
        \end{tabular}
        \caption{Continuacion resultados de las mejores corridas de \emph{NPSO} no hibridado para {\bf Iris}}
        \label{tb:tablepsoalgcsvc}
    \end{center}
\end{table}

        
        \begin{table}[h!]
    \footnotesize
    \begin{center}
        \begin{tabular}{|c|c|c|c|c|c|c|c|c|c|c|c|c|}
        \hline
            & {\bf FO} & {\bf DB} & $J_e$ & {\bf E} & {\bf T} & {\bf KE} & {\bf KO} & $I$ & W & $c_1$ & $c_2$ & $vmx$ \\
        \hline
        \hline
            Promedio  & 1.5654 & 0.6395 & 0.6606 & 136.0667 & 0.0001 & 3.0 & 3 &  &  &  &  & \\
            \cline{1-8}
            Mejor & 1.6626 & 0.6015  & 0.6807 & 130 & 0.0001 & 3 & 3 & 25 & 1.1 & 1.7 & 0.5 & 0.5\\
            \cline{1-8}
            Peor & 1.5006 & 0.6664  & 0.649 & 105 & 0.0001 & 3 & 3 &  &  &  &  & \\
        \hline
        \hline
            Promedio  & 1.5666 & 0.639 & 0.6609 & 136.9 & 0.0001 & 3.0 & 3 &  &  &  &  & \\
            \cline{1-8}
            Mejor & 1.6626 & 0.6015  & 0.6807 & 130 & 0.0001 & 3 & 3 & 25 & 1.1 & 1.4 & 2.0 & 0.9\\
            \cline{1-8}
            Peor & 1.5006 & 0.6664  & 0.649 & 105 & 0.0001 & 3 & 3 &  &  &  &  & \\
        \hline
        \hline
            Promedio  & 1.5666 & 0.639 & 0.6609 & 136.9 & 0.0001 & 3.0 & 3 &  &  &  &  & \\
            \cline{1-8}
            Mejor & 1.6626 & 0.6015  & 0.6807 & 130 & 0.0001 & 3 & 3 & 25 & 1.1 & 1.4 & 2.0 & 0.7\\
            \cline{1-8}
            Peor & 1.5006 & 0.6664  & 0.649 & 105 & 0.0001 & 3 & 3 &  &  &  &  & \\
        \hline
        \hline
            Promedio  & 1.5666 & 0.639 & 0.6609 & 136.9 & 0.0001 & 3.0 & 3 &  &  &  &  & \\
            \cline{1-8}
            Mejor & 1.6626 & 0.6015  & 0.6807 & 130 & 0.0001 & 3 & 3 & 25 & 1.1 & 1.4 & 2.0 & 0.5\\
            \cline{1-8}
            Peor & 1.5006 & 0.6664  & 0.649 & 105 & 0.0001 & 3 & 3 &  &  &  &  & \\
        \hline
        \hline
            Promedio  & 1.5668 & 0.639 & 0.661 & 132.7 & 0.0001 & 3.0 & 3 &  &  &  &  & \\
            \cline{1-8}
            Mejor & 1.6626 & 0.6015  & 0.6807 & 130 & 0.0001 & 3 & 3 & 25 & 1.1 & 1.4 & 1.7 & 0.9\\
            \cline{1-8}
            Peor & 1.5006 & 0.6664  & 0.649 & 105 & 0.0001 & 3 & 3 &  &  &  &  & \\
        \hline
        \hline
            Promedio  & 1.5668 & 0.639 & 0.661 & 132.7 & 0.0001 & 3.0 & 3 &  &  &  &  & \\
            \cline{1-8}
            Mejor & 1.6626 & 0.6015  & 0.6807 & 130 & 0.0001 & 3 & 3 & 25 & 1.1 & 1.4 & 1.7 & 0.7\\
            \cline{1-8}
            Peor & 1.5006 & 0.6664  & 0.649 & 105 & 0.0001 & 3 & 3 &  &  &  &  & \\
        \hline
        \hline
            Promedio  & 1.5668 & 0.639 & 0.661 & 132.7 & 0.0001 & 3.0 & 3 &  &  &  &  & \\
            \cline{1-8}
            Mejor & 1.6626 & 0.6015  & 0.6807 & 130 & 0.0001 & 3 & 3 & 25 & 1.1 & 1.4 & 1.7 & 0.5\\
            \cline{1-8}
            Peor & 1.5006 & 0.6664  & 0.649 & 105 & 0.0 & 3 & 3 &  &  &  &  & \\
        \hline
        \hline
            Promedio  & 1.5668 & 0.639 & 0.661 & 132.7 & 0.0001 & 3.0 & 3 &  &  &  &  & \\
            \cline{1-8}
            Mejor & 1.6626 & 0.6015  & 0.6807 & 130 & 0.0001 & 3 & 3 & 25 & 1.1 & 1.4 & 1.4 & 0.9\\
            \cline{1-8}
            Peor & 1.5006 & 0.6664  & 0.649 & 105 & 0.0 & 3 & 3 &  &  &  &  & \\
        \hline
        \hline
            Promedio  & 1.5635 & 0.6403 & 0.6601 & 135.2333 & 0.0 & 3.0 & 3 &  &  &  &  & \\
            \cline{1-8}
            Mejor & 1.6626 & 0.6015  & 0.6807 & 130 & 0.0001 & 3 & 3 & 25 & 1.1 & 1.4 & 1.4 & 0.7\\
            \cline{1-8}
            Peor & 1.5006 & 0.6664  & 0.649 & 105 & 0.0 & 3 & 3 &  &  &  &  & \\
        \hline
        \hline
            Promedio  & 1.5635 & 0.6403 & 0.6601 & 135.2333 & 0.0 & 3.0 & 3 &  &  &  &  & \\
            \cline{1-8}
            Mejor & 1.6626 & 0.6015  & 0.6807 & 130 & 0.0001 & 3 & 3 & 25 & 1.1 & 1.4 & 1.4 & 0.5\\
            \cline{1-8}
            Peor & 1.5006 & 0.6664  & 0.649 & 105 & 0.0001 & 3 & 3 &  &  &  &  & \\
        \hline
        \hline
            Promedio  & 1.5689 & 0.6382 & 0.6612 & 136.0667 & 0.0 & 3.0 & 3 &  &  &  &  & \\
            \cline{1-8}
            Mejor & 1.6626 & 0.6015  & 0.6807 & 130 & 0.0001 & 3 & 3 & 25 & 1.1 & 1.4 & 1.1 & 0.9\\
            \cline{1-8}
            Peor & 1.5006 & 0.6664  & 0.649 & 105 & 0.0 & 3 & 3 &  &  &  &  & \\
        \hline
        \hline
            Promedio  & 1.5689 & 0.6382 & 0.6612 & 136.0667 & 0.0 & 3.0 & 3 &  &  &  &  & \\
            \cline{1-8}
            Mejor & 1.6626 & 0.6015  & 0.6807 & 130 & 0.0001 & 3 & 3 & 25 & 1.1 & 1.4 & 1.1 & 0.7\\
            \cline{1-8}
            Peor & 1.5006 & 0.6664  & 0.649 & 105 & 0.0 & 3 & 3 &  &  &  &  & \\
        \hline
        \hline
            Promedio  & 1.5689 & 0.6382 & 0.6612 & 136.0667 & 0.0 & 3.0 & 3 &  &  &  &  & \\
            \cline{1-8}
            Mejor & 1.6626 & 0.6015  & 0.6807 & 130 & 0.0001 & 3 & 3 & 25 & 1.1 & 1.4 & 1.1 & 0.5\\
            \cline{1-8}
            Peor & 1.5006 & 0.6664  & 0.649 & 105 & 0.0 & 3 & 3 &  &  &  &  & \\
        \hline
        \hline
            Promedio  & 1.5689 & 0.6382 & 0.6612 & 136.0667 & 0.0 & 3.0 & 3 &  &  &  &  & \\
            \cline{1-8}
            Mejor & 1.6626 & 0.6015  & 0.6807 & 130 & 0.0001 & 3 & 3 & 25 & 1.1 & 1.4 & 0.8 & 0.9\\
            \cline{1-8}
            Peor & 1.5006 & 0.6664  & 0.649 & 105 & 0.0 & 3 & 3 &  &  &  &  & \\
        \hline
        \hline
            Promedio  & 1.5689 & 0.6382 & 0.6612 & 136.0667 & 0.0 & 3.0 & 3 &  &  &  &  & \\
            \cline{1-8}
            Mejor & 1.6626 & 0.6015  & 0.6807 & 130 & 0.0001 & 3 & 3 & 25 & 1.1 & 1.4 & 0.8 & 0.7\\
            \cline{1-8}
            Peor & 1.5006 & 0.6664  & 0.649 & 105 & 0.0 & 3 & 3 &  &  &  &  & \\
        \hline
        \end{tabular}
        \caption{Resultados de las mejores corridas de \emph{NPSO} hibridado para {\bf Iris}}
        \label{tb:tablepsohibcsv}
    \end{center}
\end{table}


\begin{table}[h!]
    \footnotesize
    \begin{center}
        \begin{tabular}{|c|c|c|c|c|c|c|c|c|c|c|c|c|}
        \hline
            & {\bf FO} & {\bf DB} & $J_e$ & {\bf E} & {\bf T} & {\bf KE} & {\bf KO} & $I$ & W & $c_1$ & $c_2$ & $vmx$ \\
        \hline
        \hline
            Promedio  & 1.5689 & 0.6382 & 0.6612 & 136.0667 & 0.0 & 3.0 & 3 &  &  &  &  & \\
            \cline{1-8}
            Mejor & 1.6626 & 0.6015  & 0.6807 & 130 & 0.0001 & 3 & 3 & 25 & 1.1 & 1.4 & 0.8 & 0.5\\
            \cline{1-8}
            Peor & 1.5006 & 0.6664  & 0.649 & 105 & 0.0 & 3 & 3 &  &  &  &  & \\
        \hline
        \hline
            Promedio  & 1.5689 & 0.6382 & 0.6612 & 136.0667 & 0.0 & 3.0 & 3 &  &  &  &  & \\
            \cline{1-8}
            Mejor & 1.6626 & 0.6015  & 0.6807 & 130 & 0.0001 & 3 & 3 & 25 & 1.1 & 1.4 & 0.5 & 0.9\\
            \cline{1-8}
            Peor & 1.5006 & 0.6664  & 0.649 & 105 & 0.0 & 3 & 3 &  &  &  &  & \\
        \hline
        \hline
            Promedio  & 1.5689 & 0.6382 & 0.6612 & 136.0667 & 0.0 & 3.0 & 3 &  &  &  &  & \\
            \cline{1-8}
            Mejor & 1.6626 & 0.6015  & 0.6807 & 130 & 0.0 & 3 & 3 & 25 & 1.1 & 1.4 & 0.5 & 0.7\\
            \cline{1-8}
            Peor & 1.5006 & 0.6664  & 0.649 & 105 & 0.0 & 3 & 3 &  &  &  &  & \\
        \hline
        \hline
            Promedio  & 1.5689 & 0.6382 & 0.6612 & 136.0667 & 0.0 & 3.0 & 3 &  &  &  &  & \\
            \cline{1-8}
            Mejor & 1.6626 & 0.6015  & 0.6807 & 130 & 0.0 & 3 & 3 & 25 & 1.1 & 1.4 & 0.5 & 0.5\\
            \cline{1-8}
            Peor & 1.5006 & 0.6664  & 0.649 & 105 & 0.0 & 3 & 3 &  &  &  &  & \\
        \hline
        \hline
            Promedio  & 1.5689 & 0.6382 & 0.6612 & 136.0667 & 0.0 & 3.0 & 3 &  &  &  &  & \\
            \cline{1-8}
            Mejor & 1.6626 & 0.6015  & 0.6807 & 130 & 0.0001 & 3 & 3 & 25 & 1.1 & 1.1 & 2.0 & 0.9\\
            \cline{1-8}
            Peor & 1.5006 & 0.6664  & 0.649 & 105 & 0.0 & 3 & 3 &  &  &  &  & \\
        \hline
        \end{tabular}
        \caption{Continuacion resultados de las mejores corridas de \emph{NPSO} hibridado para {\bf Iris}}
        \label{tb:tablepsohibcsvc}
    \end{center}
\end{table}


    \subsection{\emph{WPSOH}}\label{sect:awpso}

        Las variables del \emph{WPSOH}(\ref{sect:iwpso}) son las siguientes:
        \begin{itemize}
            \item $I$: tamaño de la población. Se varió su valor en el rango
        $[5, 10, \cdots, 40]$.
            \item $c_1$: factor de la componente cognitiva. Se varió su valor en el
        $[0.5; 0.8; \cdots; 2.0]$
            \item $c_2$: factor de la componente social. Se varió en el rango
        $[0.5; 0.8; \cdots; 2.0]$
            \item $W$: peso inercial de la partícula. Se varió su valor en el rango
        rango $[0.5; 0.8; 1.1]$
            \item $vmx$: velocidad máxima de la partícula. Se varió su valor en el rango
        $[0.5; 0.7; 0.9]$.
	        \item $w_1$: peso de la distancia promedio entre elementos de un
        mismo cluster (distancia intra-cluster promedio). Se varió su valor en el rango
        $[0.0; 0.2; \cdots; 1.0]$
	        \item $w_2$: peso de la distancia promedio entre los diferentes
        clusters (distancia inter-cluster promedio). Se varió su valor en el rango
        $[0.0; 0.2, \cdots; 1.0]$
	        \item $w_3$: peso de el error. Se varió su valor en el rango
        $[0.0; 0.2; \cdots; 1.0]$
        \end{itemize}

        \begin{landscape}

        \begin{table}[h!]
    \footnotesize
    \begin{center}
        \begin{tabular}{|c|c|c|c|c|c|c|c|c|c|c|c|c|c|}
        \hline
            & {\bf FO} & {\bf DB} & $J_e$ & {\bf E} & {\bf T} & $I$ & $w_1$ & $w_2$ & $w_3$ & $W$ & $c_1$ & $c_2$ & $vmx$ \\
        \hline
        \hline
            Promedio  & 18.4658 & 14.0986 & 97.1735 & 120.0 & 0.2523 &  &  &  &  &  &  &  & \\
            \cline{1-6}
            Mejor & 19.4539 & 56.5882  & 105.6645 & 120 & 0.2517 & 30 & 0.0 & 0.0 & 1.0 & 1.1 & 1.7 & 1.4 & 0.9\\
            \cline{1-6}
            Peor & 17.2308 & 8.0981  & 71.8033 & 120 & 0.2687 &  &  &  &  &  &  &  & \\
        \hline
        \hline
            Promedio  & 18.7088 & 11.0352 & 91.8654 & 80.0 & 0.1689 &  &  &  &  &  &  &  & \\
            \cline{1-6}
            Mejor & 19.6336 & 11.1146  & 134.183 & 80 & 0.1711 & 20 & 0.0 & 0.0 & 1.0 & 0.5 & 0.8 & 1.7 & 0.9\\
            \cline{1-6}
            Peor & 17.7858 & 20.1442  & 127.2173 & 80 & 0.1618 &  &  &  &  &  &  &  & \\
        \hline
        \hline
            Promedio  & 19.1403 & 11.2419 & 84.017 & 40.0 & 0.0863 &  &  &  &  &  &  &  & \\
            \cline{1-6}
            Mejor & 20.2896 & 11.0883  & 84.9975 & 40 & 0.0838 & 10 & 0.0 & 0.0 & 1.0 & 0.8 & 0.5 & 1.1 & 0.7\\
            \cline{1-6}
            Peor & 18.3753 & 6.0511  & 54.5112 & 40 & 0.0824 &  &  &  &  &  &  &  & \\
        \hline
        \hline
            Promedio  & 19.1906 & 10.4836 & 86.7853 & 40.0 & 0.0842 &  &  &  &  &  &  &  & \\
            \cline{1-6}
            Mejor & 20.5477 & 10.9682  & 135.1091 & 40 & 0.0831 & 10 & 0.0 & 0.0 & 1.0 & 0.8 & 0.5 & 1.4 & 0.5\\
            \cline{1-6}
            Peor & 18.3753 & 6.0511  & 54.5112 & 40 & 0.0858 &  &  &  &  &  &  &  & \\
        \hline
        \hline
            Promedio  & 19.2069 & 11.2355 & 86.3337 & 40.0 & 0.0838 &  &  &  &  &  &  &  & \\
            \cline{1-6}
            Mejor & 20.5477 & 10.9682  & 135.1091 & 40 & 0.086 & 10 & 0.0 & 0.0 & 1.0 & 0.8 & 0.5 & 1.1 & 0.9\\
            \cline{1-6}
            Peor & 18.3753 & 6.0511  & 54.5112 & 40 & 0.0884 &  &  &  &  &  &  &  & \\
        \hline
        \hline
            Promedio  & 19.6959 & 10.2546 & 91.4527 & 20.0 & 0.0419 &  &  &  &  &  &  &  & \\
            \cline{1-6}
            Mejor & 21.6526 & 6.9407  & 80.7228 & 20 & 0.0444 & 5 & 0.0 & 0.0 & 1.0 & 1.1 & 1.1 & 1.4 & 0.7\\
            \cline{1-6}
            Peor & 17.7947 & 11.4616  & 99.8348 & 20 & 0.0457 &  &  &  &  &  &  &  & \\
        \hline
        \hline
            Promedio  & 19.6861 & 10.2943 & 90.7865 & 20.0 & 0.0423 &  &  &  &  &  &  &  & \\
            \cline{1-6}
            Mejor & 21.6526 & 6.9407  & 80.7228 & 20 & 0.0391 & 5 & 0.0 & 0.0 & 1.0 & 1.1 & 1.1 & 1.4 & 0.5\\
            \cline{1-6}
            Peor & 17.7947 & 11.4616  & 99.8348 & 20 & 0.0398 &  &  &  &  &  &  &  & \\
        \hline
        \hline
            Promedio  & 20.6196 & 10.9362 & 97.0209 & 120.0 & 0.2531 &  &  &  &  &  &  &  & \\
            \cline{1-6}
            Mejor & 21.759 & 30.7571  & 114.119 & 120 & 0.2741 & 30 & 0.2 & 0.0 & 0.8 & 0.5 & 0.5 & 0.5 & 0.9\\
            \cline{1-6}
            Peor & 19.4754 & 15.9213  & 64.5499 & 120 & 0.2661 &  &  &  &  &  &  &  & \\
        \hline
        \hline
            Promedio  & 20.5489 & 9.5802 & 97.3755 & 120.0 & 0.2511 &  &  &  &  &  &  &  & \\
            \cline{1-6}
            Mejor & 21.759 & 30.7571  & 114.119 & 120 & 0.2589 & 30 & 0.2 & 0.0 & 0.8 & 0.5 & 0.5 & 2.0 & 0.9\\
            \cline{1-6}
            Peor & 19.0486 & 7.903  & 92.5078 & 120 & 0.2734 &  &  &  &  &  &  &  & \\
        \hline
        \hline
            Promedio  & 19.6407 & 9.9371 & 89.8618 & 20.0 & 0.0429 &  &  &  &  &  &  &  & \\
            \cline{1-6}
            Mejor & 21.9845 & 9.2314  & 99.3713 & 20 & 0.0408 & 5 & 0.0 & 0.0 & 1.0 & 1.1 & 1.1 & 1.1 & 0.9\\
            \cline{1-6}
            Peor & 17.7947 & 11.4616  & 99.8348 & 20 & 0.0441 &  &  &  &  &  &  &  & \\
        \hline
        \end{tabular}
        \caption{Resultados de las mejores corridas de \emph{WPSO} no hibridado para {\bf Lenna}}
        \label{tb:tablewpsoalgimg}
    \end{center}
\end{table}


\begin{table}[h!]
    \footnotesize
    \begin{center}
        \begin{tabular}{|c|c|c|c|c|c|c|c|c|c|c|c|c|c|}
        \hline
            & {\bf FO} & {\bf DB} & $J_e$ & {\bf E} & {\bf T} & $I$ & $w_1$ & $w_2$ & $w_3$ & $W$ & $c_1$ & $c_2$ & $vmx$ \\
        \hline
        \hline
            Promedio  & 23.7574 & 11.0542 & 98.3247 & 120.0 & 0.2499 &  &  &  &  &  &  &  & \\
            \cline{1-6}
            Mejor & 25.6389 & 7.2992  & 82.8866 & 120 & 0.2489 & 30 & 0.6 & 0.0 & 0.4 & 0.5 & 2.0 & 0.8 & 0.7\\
            \cline{1-6}
            Peor & 21.6399 & 5.6573  & 57.2955 & 120 & 0.2734 &  &  &  &  &  &  &  & \\
        \hline
        \hline
            Promedio  & 24.9175 & 9.2355 & 99.2847 & 80.0 & 0.1673 &  &  &  &  &  &  &  & \\
            \cline{1-6}
            Mejor & 26.9061 & 19.3391  & 160.7946 & 80 & 0.1808 & 20 & 0.8 & 0.0 & 0.2 & 1.1 & 1.1 & 0.8 & 0.7\\
            \cline{1-6}
            Peor & 22.5786 & 6.6858  & 79.8526 & 80 & 0.1608 &  &  &  &  &  &  &  & \\
        \hline
        \hline
            Promedio  & 24.4597 & 11.2179 & 93.2569 & 160.0 & 0.3337 &  &  &  &  &  &  &  & \\
            \cline{1-6}
            Mejor & 27.122 & 56.5882  & 105.6645 & 160 & 0.3238 & 40 & 0.8 & 0.0 & 0.2 & 0.8 & 0.8 & 2.0 & 0.7\\
            \cline{1-6}
            Peor & 21.5691 & 5.0126  & 69.9673 & 160 & 0.3266 &  &  &  &  &  &  &  & \\
        \hline
        \hline
            Promedio  & 24.8838 & 10.6832 & 98.9887 & 80.0 & 0.1684 &  &  &  &  &  &  &  & \\
            \cline{1-6}
            Mejor & 28.5793 & 11.072  & 111.4132 & 80 & 0.1754 & 20 & 0.8 & 0.0 & 0.2 & 1.1 & 1.1 & 0.8 & 0.5\\
            \cline{1-6}
            Peor & 22.1546 & 6.9085  & 90.1169 & 80 & 0.1665 &  &  &  &  &  &  &  & \\
        \hline
        \hline
            Promedio  & 25.1094 & 8.5629 & 93.4832 & 100.0 & 0.2084 &  &  &  &  &  &  &  & \\
            \cline{1-6}
            Mejor & 29.5269 & 10.0367  & 102.8227 & 100 & 0.2122 & 25 & 0.8 & 0.0 & 0.2 & 0.8 & 1.7 & 1.7 & 0.7\\
            \cline{1-6}
            Peor & 22.7291 & 5.6759  & 105.358 & 100 & 0.2155 &  &  &  &  &  &  &  & \\
        \hline
        \hline
            Promedio  & 100.195 & 6.9474 & 104.1164 & 82.0 & 0.172 &  &  &  &  &  &  &  & \\
            \cline{1-6}
            Mejor & 101.642 & 12.8066  & 138.6754 & 80 & 0.1639 & 20 & 0.0 & 0.2 & 0.8 & 0.5 & 1.1 & 1.7 & 0.7\\
            \cline{1-6}
            Peor & 97.6743 & 6.2551  & 104.6887 & 80 & 0.1758 &  &  &  &  &  &  &  & \\
        \hline
        \hline
            Promedio  & 101.8446 & 6.0322 & 91.6344 & 125.0 & 0.2599 &  &  &  &  &  &  &  & \\
            \cline{1-6}
            Mejor & 103.7253 & 3.5907  & 57.0002 & 120 & 0.2478 & 30 & 0.2 & 0.2 & 0.6 & 0.8 & 1.7 & 0.5 & 0.5\\
            \cline{1-6}
            Peor & 100.1234 & 4.8655  & 65.1067 & 120 & 0.2669 &  &  &  &  &  &  &  & \\
        \hline
        \hline
            Promedio  & 101.5294 & 6.58 & 95.1895 & 104.1667 & 0.2176 &  &  &  &  &  &  &  & \\
            \cline{1-6}
            Mejor & 104.3527 & 6.3038  & 121.035 & 100 & 0.2118 & 25 & 0.2 & 0.2 & 0.6 & 0.8 & 1.1 & 0.5 & 0.9\\
            \cline{1-6}
            Peor & 98.331 & 3.5639  & 108.776 & 125 & 0.2481 &  &  &  &  &  &  &  & \\
        \hline
        \hline
            Promedio  & 240.4078 & 3.4226 & 114.0829 & 253.1667 & 0.5014 &  &  &  &  &  &  &  & \\
            \cline{1-6}
            Mejor & 246.8582 & 4.8549  & 145.1414 & 175 & 0.3607 & 35 & 0.0 & 0.6 & 0.4 & 0.5 & 0.5 & 0.5 & 0.9\\
            \cline{1-6}
            Peor & 236.6754 & 3.2675  & 109.4744 & 210 & 0.4156 &  &  &  &  &  &  &  & \\
        \hline
        \hline
            Promedio  & 373.2204 & 3.5303 & 124.5132 & 264.8333 & 0.5242 &  &  &  &  &  &  &  & \\
            \cline{1-6}
            Mejor & 386.1302 & 3.4384  & 104.443 & 175 & 0.3567 & 35 & 0.0 & 1.0 & 0.0 & 0.5 & 2.0 & 0.5 & 0.5\\
            \cline{1-6}
            Peor & 361.038 & 3.3777  & 139.0754 & 245 & 0.4858 &  &  &  &  &  &  &  & \\
        \hline
        \end{tabular}
        \caption{Continuacion resultados de las mejores corridas de \emph{WPSO} no hibridado para {\bf Lenna}}
        \label{tb:tablewpsoalgimgc}
    \end{center}
\end{table}

        
        \begin{table}[h!]
    \footnotesize
    \begin{center}
        \begin{tabular}{|c|c|c|c|c|c|c|c|c|c|c|c|c|c|c|c|}
        \hline
            & {\bf FO} & {\bf DB} & $J_e$ & {\bf E} & {\bf T} & {\bf KE} & {\bf KO} & $I$ & $w_1$ & $w_2$ & $w_3$ & $W$ & $c_1$ & $c_2$ & $vmx$ \\
        \hline
        \hline
            Promedio  & 1.2005 & 0.8348 & 17.7634 & 275.8333 & 0.0137 & 9.0 & $[5-10]$ &  &  &  &  &  &  &  & \\
            \cline{1-8}
            Mejor & 1.3039 & 0.7669  & 16.9839 & 183 & 0.01 & 9 & $[5-10]$ & 35 & 0.0 & 1.0 & 0.0 & 0.5 & 2.0 & 0.5 & 0.5\\
            \cline{1-8}
            Peor & 1.1074 & 0.903  & 17.6885 & 219 & 0.0112 & 9 & $[5-10]$ &  &  &  &  &  &  &  & \\
        \hline
        \hline
            Promedio  & 1.1945 & 0.843 & 16.9607 & 96.2667 & 0.0213 & 9.0 & $[5-10]$ &  &  &  &  &  &  &  & \\
            \cline{1-8}
            Mejor & 1.2965 & 0.7713  & 16.8186 & 96 & 0.0206 & 9 & $[5-10]$ & 20 & 0.0 & 0.0 & 1.0 & 0.5 & 0.8 & 1.7 & 0.9\\
            \cline{1-8}
            Peor & 0.9229 & 1.0835  & 17.3661 & 85 & 0.0057 & 9 & $[5-10]$ &  &  &  &  &  &  &  & \\
        \hline
        \hline
            Promedio  & 1.2037 & 0.8331 & 17.2391 & 144.4 & 0.0256 & 9.0 & $[5-10]$ &  &  &  &  &  &  &  & \\
            \cline{1-8}
            Mejor & 1.2953 & 0.772  & 16.7712 & 127 & 0.0084 & 9 & $[5-10]$ & 30 & 0.2 & 0.2 & 0.6 & 0.8 & 1.7 & 0.5 & 0.5\\
            \cline{1-8}
            Peor & 1.0378 & 0.9636  & 17.629 & 125 & 0.0058 & 9 & $[5-10]$ &  &  &  &  &  &  &  & \\
        \hline
        \hline
            Promedio  & 1.1928 & 0.8413 & 17.0955 & 35.1333 & 0.0207 & 9.0 & $[5-10]$ &  &  &  &  &  &  &  & \\
            \cline{1-8}
            Mejor & 1.2932 & 0.7733  & 16.9661 & 30 & 0.0129 & 9 & $[5-10]$ & 5 & 0.0 & 0.0 & 1.0 & 1.1 & 1.1 & 1.4 & 0.5\\
            \cline{1-8}
            Peor & 1.0367 & 0.9646  & 18.2838 & 26 & 0.007 & 9 & $[5-10]$ &  &  &  &  &  &  &  & \\
        \hline
        \hline
            Promedio  & 1.1945 & 0.8401 & 17.0464 & 35.1333 & 0.0199 & 9.0 & $[5-10]$ &  &  &  &  &  &  &  & \\
            \cline{1-8}
            Mejor & 1.2932 & 0.7733  & 16.9661 & 30 & 0.0129 & 9 & $[5-10]$ & 5 & 0.0 & 0.0 & 1.0 & 1.1 & 1.1 & 1.1 & 0.9\\
            \cline{1-8}
            Peor & 1.0367 & 0.9646  & 18.2838 & 26 & 0.0068 & 9 & $[5-10]$ &  &  &  &  &  &  &  & \\
        \hline
        \hline
            Promedio  & 1.1934 & 0.8425 & 16.9685 & 54.2 & 0.0186 & 9.0 & $[5-10]$ &  &  &  &  &  &  &  & \\
            \cline{1-8}
            Mejor & 1.2932 & 0.7733  & 16.9661 & 50 & 0.0125 & 9 & $[5-10]$ & 10 & 0.0 & 0.0 & 1.0 & 0.8 & 0.5 & 1.4 & 0.5\\
            \cline{1-8}
            Peor & 0.9866 & 1.0136  & 17.0783 & 50 & 0.0128 & 9 & $[5-10]$ &  &  &  &  &  &  &  & \\
        \hline
        \hline
            Promedio  & 1.1847 & 0.849 & 16.9272 & 54.4333 & 0.0195 & 9.0 & $[5-10]$ &  &  &  &  &  &  &  & \\
            \cline{1-8}
            Mejor & 1.2932 & 0.7733  & 16.9661 & 50 & 0.0127 & 9 & $[5-10]$ & 10 & 0.0 & 0.0 & 1.0 & 0.8 & 0.5 & 1.1 & 0.9\\
            \cline{1-8}
            Peor & 0.9866 & 1.0136  & 17.0783 & 50 & 0.0125 & 9 & $[5-10]$ &  &  &  &  &  &  &  & \\
        \hline
        \hline
            Promedio  & 1.1909 & 0.8442 & 16.9116 & 54.2667 & 0.0203 & 9.0 & $[5-10]$ &  &  &  &  &  &  &  & \\
            \cline{1-8}
            Mejor & 1.2932 & 0.7733  & 16.9661 & 50 & 0.0126 & 9 & $[5-10]$ & 10 & 0.0 & 0.0 & 1.0 & 0.8 & 0.5 & 1.1 & 0.7\\
            \cline{1-8}
            Peor & 0.9866 & 1.0136  & 17.0783 & 50 & 0.0125 & 9 & $[5-10]$ &  &  &  &  &  &  &  & \\
        \hline
        \hline
            Promedio  & 1.179 & 0.8517 & 17.1651 & 116.8333 & 0.0228 & 9.0 & $[5-10]$ &  &  &  &  &  &  &  & \\
            \cline{1-8}
            Mejor & 1.29 & 0.7752  & 16.7011 & 118 & 0.0239 & 9 & $[5-10]$ & 25 & 0.8 & 0.0 & 0.2 & 0.8 & 1.7 & 1.7 & 0.7\\
            \cline{1-8}
            Peor & 1.0575 & 0.9457  & 17.2586 & 105 & 0.0056 & 9 & $[5-10]$ &  &  &  &  &  &  &  & \\
        \hline
        \hline
            Promedio  & 1.1897 & 0.8431 & 17.271 & 119.3667 & 0.02 & 9.0 & $[5-10]$ &  &  &  &  &  &  &  & \\
            \cline{1-8}
            Mejor & 1.2887 & 0.776  & 16.7719 & 109 & 0.0114 & 9 & $[5-10]$ & 25 & 0.2 & 0.2 & 0.6 & 0.8 & 1.1 & 0.5 & 0.9\\
            \cline{1-8}
            Peor & 1.0624 & 0.9412  & 17.2029 & 107 & 0.0084 & 9 & $[5-10]$ &  &  &  &  &  &  &  & \\
        \hline
        \end{tabular}
        \caption{Resultados de las mejores corridas de \emph{WPSO} hibridado para {\bf Lenna}}
        \label{tb:tablewpsohibimg}
    \end{center}
\end{table}


\begin{table}[h!]
    \footnotesize
    \begin{center}
        \begin{tabular}{|c|c|c|c|c|c|c|c|c|c|c|c|c|c|c|c|}
        \hline
            & {\bf FO} & {\bf DB} & $J_e$ & {\bf E} & {\bf T} & {\bf KE} & {\bf KO} & $I$ & $w_1$ & $w_2$ & $w_3$ & $W$ & $c_1$ & $c_2$ & $vmx$ \\
        \hline
        \hline
            Promedio  & 1.1807 & 0.8521 & 17.0803 & 135.0333 & 0.0195 & 9.0 & $[5-10]$ &  &  &  &  &  &  &  & \\
            \cline{1-8}
            Mejor & 1.2864 & 0.7774  & 16.7391 & 145 & 0.0337 & 9 & $[5-10]$ & 30 & 0.2 & 0.0 & 0.8 & 0.5 & 0.5 & 2.0 & 0.9\\
            \cline{1-8}
            Peor & 0.9915 & 1.0086  & 17.5165 & 127 & 0.0081 & 9 & $[5-10]$ &  &  &  &  &  &  &  & \\
        \hline
        \hline
            Promedio  & 1.1905 & 0.8444 & 16.8737 & 135.8 & 0.0207 & 9.0 & $[5-10]$ &  &  &  &  &  &  &  & \\
            \cline{1-8}
            Mejor & 1.2833 & 0.7792  & 16.787 & 139 & 0.0255 & 9 & $[5-10]$ & 30 & 0.0 & 0.0 & 1.0 & 1.1 & 1.7 & 1.4 & 0.9\\
            \cline{1-8}
            Peor & 1.0181 & 0.9822  & 16.4166 & 126 & 0.0069 & 9 & $[5-10]$ &  &  &  &  &  &  &  & \\
        \hline
        \hline
            Promedio  & 1.195 & 0.8399 & 17.2609 & 95.5667 & 0.0177 & 9.0 & $[5-10]$ &  &  &  &  &  &  &  & \\
            \cline{1-8}
            Mejor & 1.2833 & 0.7792  & 16.787 & 99 & 0.0255 & 9 & $[5-10]$ & 20 & 0.0 & 0.2 & 0.8 & 0.5 & 1.1 & 1.7 & 0.7\\
            \cline{1-8}
            Peor & 1.0465 & 0.9555  & 16.9116 & 86 & 0.0072 & 9 & $[5-10]$ &  &  &  &  &  &  &  & \\
        \hline
        \hline
            Promedio  & 1.1661 & 0.8613 & 17.2109 & 175.6667 & 0.0205 & 9.0 & $[5-10]$ &  &  &  &  &  &  &  & \\
            \cline{1-8}
            Mejor & 1.2831 & 0.7794  & 16.7151 & 172 & 0.0156 & 9 & $[5-10]$ & 40 & 0.8 & 0.0 & 0.2 & 0.8 & 0.8 & 2.0 & 0.7\\
            \cline{1-8}
            Peor & 1.0482 & 0.9541  & 16.772 & 166 & 0.007 & 9 & $[5-10]$ &  &  &  &  &  &  &  & \\
        \hline
        \hline
            Promedio  & 1.1916 & 0.842 & 17.1025 & 35.0333 & 0.0201 & 9.0 & $[5-10]$ &  &  &  &  &  &  &  & \\
            \cline{1-8}
            Mejor & 1.2816 & 0.7803  & 16.7862 & 36 & 0.0211 & 9 & $[5-10]$ & 5 & 0.0 & 0.0 & 1.0 & 1.1 & 1.1 & 1.4 & 0.7\\
            \cline{1-8}
            Peor & 1.0367 & 0.9646  & 18.2838 & 26 & 0.0068 & 9 & $[5-10]$ &  &  &  &  &  &  &  & \\
        \hline
        \hline
            Promedio  & 1.1758 & 0.855 & 17.178 & 137.5333 & 0.0231 & 9.0 & $[5-10]$ &  &  &  &  &  &  &  & \\
            \cline{1-8}
            Mejor & 1.2806 & 0.7809  & 16.8202 & 131 & 0.0194 & 9 & $[5-10]$ & 30 & 0.6 & 0.0 & 0.4 & 0.5 & 2.0 & 0.8 & 0.7\\
            \cline{1-8}
            Peor & 0.9045 & 1.1056  & 16.8258 & 125 & 0.0055 & 9 & $[5-10]$ &  &  &  &  &  &  &  & \\
        \hline
        \hline
            Promedio  & 1.1648 & 0.863 & 17.0646 & 133.9 & 0.0182 & 9.0 & $[5-10]$ &  &  &  &  &  &  &  & \\
            \cline{1-8}
            Mejor & 1.2804 & 0.781  & 16.7901 & 133 & 0.0237 & 9 & $[5-10]$ & 30 & 0.2 & 0.0 & 0.8 & 0.5 & 0.5 & 0.5 & 0.9\\
            \cline{1-8}
            Peor & 0.9915 & 1.0086  & 17.5165 & 127 & 0.0099 & 9 & $[5-10]$ &  &  &  &  &  &  &  & \\
        \hline
        \hline
            Promedio  & 1.1508 & 0.8741 & 17.3317 & 97.3333 & 0.0228 & 9.0 & $[5-10]$ &  &  &  &  &  &  &  & \\
            \cline{1-8}
            Mejor & 1.2602 & 0.7935  & 16.8007 & 92 & 0.0212 & 9 & $[5-10]$ & 20 & 0.8 & 0.0 & 0.2 & 1.1 & 1.1 & 0.8 & 0.7\\
            \cline{1-8}
            Peor & 0.9229 & 1.0835  & 17.3661 & 85 & 0.0057 & 9 & $[5-10]$ &  &  &  &  &  &  &  & \\
        \hline
        \hline
            Promedio  & 1.161 & 0.8661 & 17.2909 & 97.8333 & 0.0236 & 9.0 & $[5-10]$ &  &  &  &  &  &  &  & \\
            \cline{1-8}
            Mejor & 1.2602 & 0.7935  & 16.8007 & 92 & 0.0153 & 9 & $[5-10]$ & 20 & 0.8 & 0.0 & 0.2 & 1.1 & 1.1 & 0.8 & 0.5\\
            \cline{1-8}
            Peor & 0.9229 & 1.0835  & 17.3661 & 85 & 0.0055 & 9 & $[5-10]$ &  &  &  &  &  &  &  & \\
        \hline
        \hline
            Promedio  & 1.1935 & 0.8395 & 17.6411 & 266.4333 & 0.0167 & 9.0 & $[5-10]$ &  &  &  &  &  &  &  & \\
            \cline{1-8}
            Mejor & 1.2482 & 0.8012  & 17.6257 & 267 & 0.0284 & 9 & $[5-10]$ & 35 & 0.0 & 0.6 & 0.4 & 0.5 & 0.5 & 0.5 & 0.9\\
            \cline{1-8}
            Peor & 1.0262 & 0.9745  & 19.2937 & 215 & 0.0056 & 9 & $[5-10]$ &  &  &  &  &  &  &  & \\
        \hline
        \end{tabular}
        \caption{Continuacion resultados de las mejores corridas de \emph{WPSO} hibridado para {\bf Lenna}}
        \label{tb:tablewpsohibimgc}
    \end{center}
\end{table}


        \end{landscape}

        \begin{table}[h!]
    \footnotesize
    \begin{center}
        \begin{tabular}{|c|c|c|c|c|c|c|c|c|c|c|c|c|c|}
        \hline
            & {\bf FO} & {\bf DB} & $J_e$ & {\bf E} & {\bf T} & $I$ & $w_1$ & $w_2$ & $w_3$ & $W$ & $c_1$ & $c_2$ & $vmx$ \\
        \hline
        \hline
            Promedio  & 1.822 & 1.8998 & 2.8918 & 54.0 & 0.0012 &  &  &  &  &  &  &  & \\
            \cline{1-6}
            Mejor & 1.9414 & 2.1848  & 4.113 & 100 & 0.0028 & 10 & 0.8 & 0.2 & 0.0 & 0.8 & 1.1 & 1.1 & 0.5\\
            \cline{1-6}
            Peor & 1.7643 & 1.4449  & 1.8597 & 40 & 0.0009 &  &  &  &  &  &  &  & \\
        \hline
        \hline
            Promedio  & 1.822 & 1.8998 & 2.8918 & 54.0 & 0.0012 &  &  &  &  &  &  &  & \\
            \cline{1-6}
            Mejor & 1.9414 & 2.1848  & 4.113 & 100 & 0.0028 & 10 & 0.8 & 0.2 & 0.0 & 0.8 & 1.1 & 0.8 & 0.9\\
            \cline{1-6}
            Peor & 1.7643 & 1.4449  & 1.8597 & 40 & 0.0008 &  &  &  &  &  &  &  & \\
        \hline
        \hline
            Promedio  & 1.822 & 1.8998 & 2.8918 & 54.0 & 0.0012 &  &  &  &  &  &  &  & \\
            \cline{1-6}
            Mejor & 1.9414 & 2.1848  & 4.113 & 100 & 0.0027 & 10 & 0.8 & 0.2 & 0.0 & 0.8 & 1.1 & 0.8 & 0.7\\
            \cline{1-6}
            Peor & 1.7643 & 1.4449  & 1.8597 & 40 & 0.0008 &  &  &  &  &  &  &  & \\
        \hline
        \hline
            Promedio  & 1.822 & 1.8998 & 2.8918 & 54.0 & 0.0012 &  &  &  &  &  &  &  & \\
            \cline{1-6}
            Mejor & 1.9414 & 2.1848  & 4.113 & 100 & 0.0027 & 10 & 0.8 & 0.2 & 0.0 & 0.8 & 1.1 & 0.8 & 0.5\\
            \cline{1-6}
            Peor & 1.7643 & 1.4449  & 1.8597 & 40 & 0.0008 &  &  &  &  &  &  &  & \\
        \hline
        \hline
            Promedio  & 1.822 & 1.8998 & 2.8918 & 54.0 & 0.0011 &  &  &  &  &  &  &  & \\
            \cline{1-6}
            Mejor & 1.9414 & 2.1848  & 4.113 & 100 & 0.0028 & 10 & 0.8 & 0.2 & 0.0 & 0.8 & 1.1 & 0.5 & 0.9\\
            \cline{1-6}
            Peor & 1.7643 & 1.4449  & 1.8597 & 40 & 0.0008 &  &  &  &  &  &  &  & \\
        \hline
        \hline
            Promedio  & 1.8249 & 1.8846 & 2.9012 & 56.0 & 0.0012 &  &  &  &  &  &  &  & \\
            \cline{1-6}
            Mejor & 1.9414 & 2.1848  & 4.113 & 100 & 0.0027 & 10 & 0.8 & 0.2 & 0.0 & 0.8 & 1.1 & 0.5 & 0.7\\
            \cline{1-6}
            Peor & 1.7643 & 1.4449  & 1.8597 & 40 & 0.0008 &  &  &  &  &  &  &  & \\
        \hline
        \hline
            Promedio  & 1.8249 & 1.8846 & 2.9012 & 56.0 & 0.0012 &  &  &  &  &  &  &  & \\
            \cline{1-6}
            Mejor & 1.9414 & 2.1848  & 4.113 & 100 & 0.0028 & 10 & 0.8 & 0.2 & 0.0 & 0.8 & 1.1 & 0.5 & 0.5\\
            \cline{1-6}
            Peor & 1.7643 & 1.4449  & 1.8597 & 40 & 0.0008 &  &  &  &  &  &  &  & \\
        \hline
        \hline
            Promedio  & 1.8249 & 1.8846 & 2.9012 & 56.0 & 0.0011 &  &  &  &  &  &  &  & \\
            \cline{1-6}
            Mejor & 1.9414 & 2.1848  & 4.113 & 100 & 0.0027 & 10 & 0.8 & 0.2 & 0.0 & 0.8 & 0.8 & 2.0 & 0.9\\
            \cline{1-6}
            Peor & 1.7643 & 1.4449  & 1.8597 & 40 & 0.0009 &  &  &  &  &  &  &  & \\
        \hline
        \hline
            Promedio  & 1.8249 & 1.8846 & 2.9012 & 56.0 & 0.0011 &  &  &  &  &  &  &  & \\
            \cline{1-6}
            Mejor & 1.9414 & 2.1848  & 4.113 & 100 & 0.0029 & 10 & 0.8 & 0.2 & 0.0 & 0.8 & 0.8 & 2.0 & 0.7\\
            \cline{1-6}
            Peor & 1.7643 & 1.4449  & 1.8597 & 40 & 0.0006 &  &  &  &  &  &  &  & \\
        \hline
        \hline
            Promedio  & 1.8249 & 1.8846 & 2.9012 & 56.0 & 0.0011 &  &  &  &  &  &  &  & \\
            \cline{1-6}
            Mejor & 1.9414 & 2.1848  & 4.113 & 100 & 0.0027 & 10 & 0.8 & 0.2 & 0.0 & 0.8 & 0.8 & 2.0 & 0.5\\
            \cline{1-6}
            Peor & 1.7643 & 1.4449  & 1.8597 & 40 & 0.0006 &  &  &  &  &  &  &  & \\
        \hline
        \hline
            Promedio  & 1.8249 & 1.8845 & 2.9025 & 56.0 & 0.0011 &  &  &  &  &  &  &  & \\
            \cline{1-6}
            Mejor & 1.9414 & 2.1848  & 4.113 & 100 & 0.0028 & 10 & 0.8 & 0.2 & 0.0 & 0.8 & 0.8 & 1.7 & 0.9\\
            \cline{1-6}
            Peor & 1.7643 & 1.4449  & 1.8597 & 40 & 0.0006 &  &  &  &  &  &  &  & \\
        \hline
        \hline
            Promedio  & 1.8249 & 1.8845 & 2.9025 & 56.0 & 0.001 &  &  &  &  &  &  &  & \\
            \cline{1-6}
            Mejor & 1.9414 & 2.1848  & 4.113 & 100 & 0.0027 & 10 & 0.8 & 0.2 & 0.0 & 0.8 & 0.8 & 1.7 & 0.7\\
            \cline{1-6}
            Peor & 1.7643 & 1.4449  & 1.8597 & 40 & 0.0006 &  &  &  &  &  &  &  & \\
        \hline
        \hline
            Promedio  & 1.8249 & 1.8845 & 2.9025 & 56.0 & 0.001 &  &  &  &  &  &  &  & \\
            \cline{1-6}
            Mejor & 1.9414 & 2.1848  & 4.113 & 100 & 0.0028 & 10 & 0.8 & 0.2 & 0.0 & 0.8 & 0.8 & 1.7 & 0.5\\
            \cline{1-6}
            Peor & 1.7643 & 1.4449  & 1.8597 & 40 & 0.0005 &  &  &  &  &  &  &  & \\
        \hline
        \hline
            Promedio  & 1.8249 & 1.8845 & 2.9025 & 56.0 & 0.0008 &  &  &  &  &  &  &  & \\
            \cline{1-6}
            Mejor & 1.9414 & 2.1848  & 4.113 & 100 & 0.0014 & 10 & 0.8 & 0.2 & 0.0 & 0.8 & 0.8 & 1.4 & 0.9\\
            \cline{1-6}
            Peor & 1.7643 & 1.4449  & 1.8597 & 40 & 0.0005 &  &  &  &  &  &  &  & \\
        \hline
        \hline
            Promedio  & 1.8249 & 1.8845 & 2.9025 & 56.0 & 0.0008 &  &  &  &  &  &  &  & \\
            \cline{1-6}
            Mejor & 1.9414 & 2.1848  & 4.113 & 100 & 0.0012 & 10 & 0.8 & 0.2 & 0.0 & 0.8 & 0.8 & 1.4 & 0.7\\
            \cline{1-6}
            Peor & 1.7643 & 1.4449  & 1.8597 & 40 & 0.0005 &  &  &  &  &  &  &  & \\
        \hline
        \end{tabular}
        \caption{Resultados de las mejores corridas de \emph{WPSO} no hibridado para {\bf Iris}}
        \label{tb:tablewpsoalgcsv}
    \end{center}
\end{table}


\begin{table}[h!]
    \footnotesize
    \begin{center}
        \begin{tabular}{|c|c|c|c|c|c|c|c|c|c|c|c|c|c|}
        \hline
            & {\bf FO} & {\bf DB} & $J_e$ & {\bf E} & {\bf T} & $I$ & $w_1$ & $w_2$ & $w_3$ & $W$ & $c_1$ & $c_2$ & $vmx$ \\
        \hline
        \hline
            Promedio  & 1.8249 & 1.8845 & 2.9025 & 56.0 & 0.0007 &  &  &  &  &  &  &  & \\
            \cline{1-6}
            Mejor & 1.9414 & 2.1848  & 4.113 & 100 & 0.0012 & 10 & 0.8 & 0.2 & 0.0 & 0.8 & 0.8 & 1.4 & 0.5\\
            \cline{1-6}
            Peor & 1.7643 & 1.4449  & 1.8597 & 40 & 0.0005 &  &  &  &  &  &  &  & \\
        \hline
        \hline
            Promedio  & 1.819 & 1.8598 & 2.8274 & 54.0 & 0.0007 &  &  &  &  &  &  &  & \\
            \cline{1-6}
            Mejor & 1.9414 & 2.1848  & 4.113 & 100 & 0.0012 & 10 & 0.8 & 0.2 & 0.0 & 0.8 & 0.8 & 1.1 & 0.9\\
            \cline{1-6}
            Peor & 1.7643 & 1.4449  & 1.8597 & 40 & 0.0004 &  &  &  &  &  &  &  & \\
        \hline
        \hline
            Promedio  & 1.819 & 1.8598 & 2.8274 & 54.0 & 0.0006 &  &  &  &  &  &  &  & \\
            \cline{1-6}
            Mejor & 1.9414 & 2.1848  & 4.113 & 100 & 0.0013 & 10 & 0.8 & 0.2 & 0.0 & 0.8 & 0.8 & 1.1 & 0.7\\
            \cline{1-6}
            Peor & 1.7643 & 1.4449  & 1.8597 & 40 & 0.0004 &  &  &  &  &  &  &  & \\
        \hline
        \hline
            Promedio  & 1.819 & 1.8598 & 2.8274 & 54.0 & 0.0006 &  &  &  &  &  &  &  & \\
            \cline{1-6}
            Mejor & 1.9414 & 2.1848  & 4.113 & 100 & 0.001 & 10 & 0.8 & 0.2 & 0.0 & 0.8 & 0.8 & 1.1 & 0.5\\
            \cline{1-6}
            Peor & 1.7643 & 1.4449  & 1.8597 & 40 & 0.0004 &  &  &  &  &  &  &  & \\
        \hline
        \hline
            Promedio  & 1.819 & 1.8598 & 2.8274 & 54.0 & 0.0006 &  &  &  &  &  &  &  & \\
            \cline{1-6}
            Mejor & 1.9414 & 2.1848  & 4.113 & 100 & 0.001 & 10 & 0.8 & 0.2 & 0.0 & 0.8 & 0.8 & 0.8 & 0.9\\
            \cline{1-6}
            Peor & 1.7643 & 1.4449  & 1.8597 & 40 & 0.0004 &  &  &  &  &  &  &  & \\
        \hline
        \end{tabular}
        \caption{Continuacion resultados de las mejores corridas de \emph{WPSO} no hibridado para {\bf Iris}}
        \label{tb:tablewpsoalgcsvc}
    \end{center}
\end{table}

        
        \begin{landscape}

        \begin{table}[h!]
    \footnotesize
    \begin{center}
        \begin{tabular}{|c|c|c|c|c|c|c|c|c|c|c|c|c|c|c|c|}
        \hline
            & {\bf FO} & {\bf DB} & $J_e$ & {\bf E} & {\bf T} & {\bf KE} & {\bf KO} & $I$ & $w_1$ & $w_2$ & $w_3$ & $W$ & $c_1$ & $c_2$ & $vmx$ \\
        \hline
        \hline
            Promedio  & 1.5694 & 0.6385 & 0.661 & 59.0 & 0.0001 & 3.0 & 3 &  &  &  &  &  &  &  & \\
            \cline{1-8}
            Mejor & 1.6992 & 0.5885  & 0.6709 & 45 & 0.0001 & 3 & 3 & 10 & 0.8 & 0.2 & 0.0 & 0.8 & 1.1 & 1.1 & 0.5\\
            \cline{1-8}
            Peor & 1.5165 & 0.6594  & 0.6466 & 45 & 0.0001 & 3 & 3 &  &  &  &  &  &  &  & \\
        \hline
        \hline
            Promedio  & 1.5694 & 0.6385 & 0.661 & 59.0 & 0.0001 & 3.0 & 3 &  &  &  &  &  &  &  & \\
            \cline{1-8}
            Mejor & 1.6992 & 0.5885  & 0.6709 & 45 & 0.0001 & 3 & 3 & 10 & 0.8 & 0.2 & 0.0 & 0.8 & 1.1 & 0.8 & 0.9\\
            \cline{1-8}
            Peor & 1.5165 & 0.6594  & 0.6466 & 45 & 0.0001 & 3 & 3 &  &  &  &  &  &  &  & \\
        \hline
        \hline
            Promedio  & 1.5694 & 0.6385 & 0.661 & 59.0 & 0.0001 & 3.0 & 3 &  &  &  &  &  &  &  & \\
            \cline{1-8}
            Mejor & 1.6992 & 0.5885  & 0.6709 & 45 & 0.0001 & 3 & 3 & 10 & 0.8 & 0.2 & 0.0 & 0.8 & 1.1 & 0.8 & 0.7\\
            \cline{1-8}
            Peor & 1.5165 & 0.6594  & 0.6466 & 45 & 0.0001 & 3 & 3 &  &  &  &  &  &  &  & \\
        \hline
        \hline
            Promedio  & 1.5694 & 0.6385 & 0.661 & 59.0 & 0.0001 & 3.0 & 3 &  &  &  &  &  &  &  & \\
            \cline{1-8}
            Mejor & 1.6992 & 0.5885  & 0.6709 & 45 & 0.0001 & 3 & 3 & 10 & 0.8 & 0.2 & 0.0 & 0.8 & 1.1 & 0.8 & 0.5\\
            \cline{1-8}
            Peor & 1.5165 & 0.6594  & 0.6466 & 45 & 0.0001 & 3 & 3 &  &  &  &  &  &  &  & \\
        \hline
        \hline
            Promedio  & 1.5694 & 0.6385 & 0.661 & 59.0 & 0.0001 & 3.0 & 3 &  &  &  &  &  &  &  & \\
            \cline{1-8}
            Mejor & 1.6992 & 0.5885  & 0.6709 & 45 & 0.0001 & 3 & 3 & 10 & 0.8 & 0.2 & 0.0 & 0.8 & 1.1 & 0.5 & 0.9\\
            \cline{1-8}
            Peor & 1.5165 & 0.6594  & 0.6466 & 45 & 0.0001 & 3 & 3 &  &  &  &  &  &  &  & \\
        \hline
        \hline
            Promedio  & 1.5692 & 0.6385 & 0.6609 & 61.0 & 0.0001 & 3.0 & 3 &  &  &  &  &  &  &  & \\
            \cline{1-8}
            Mejor & 1.6992 & 0.5885  & 0.6709 & 45 & 0.0001 & 3 & 3 & 10 & 0.8 & 0.2 & 0.0 & 0.8 & 1.1 & 0.5 & 0.7\\
            \cline{1-8}
            Peor & 1.5165 & 0.6594  & 0.6466 & 45 & 0.0001 & 3 & 3 &  &  &  &  &  &  &  & \\
        \hline
        \hline
            Promedio  & 1.5692 & 0.6385 & 0.6609 & 61.0 & 0.0001 & 3.0 & 3 &  &  &  &  &  &  &  & \\
            \cline{1-8}
            Mejor & 1.6992 & 0.5885  & 0.6709 & 45 & 0.0001 & 3 & 3 & 10 & 0.8 & 0.2 & 0.0 & 0.8 & 1.1 & 0.5 & 0.5\\
            \cline{1-8}
            Peor & 1.5165 & 0.6594  & 0.6466 & 45 & 0.0001 & 3 & 3 &  &  &  &  &  &  &  & \\
        \hline
        \hline
            Promedio  & 1.5692 & 0.6385 & 0.6609 & 61.0 & 0.0001 & 3.0 & 3 &  &  &  &  &  &  &  & \\
            \cline{1-8}
            Mejor & 1.6992 & 0.5885  & 0.6709 & 45 & 0.0001 & 3 & 3 & 10 & 0.8 & 0.2 & 0.0 & 0.8 & 0.8 & 2.0 & 0.9\\
            \cline{1-8}
            Peor & 1.5165 & 0.6594  & 0.6466 & 45 & 0.0001 & 3 & 3 &  &  &  &  &  &  &  & \\
        \hline
        \hline
            Promedio  & 1.5692 & 0.6385 & 0.6609 & 61.0 & 0.0001 & 3.0 & 3 &  &  &  &  &  &  &  & \\
            \cline{1-8}
            Mejor & 1.6992 & 0.5885  & 0.6709 & 45 & 0.0001 & 3 & 3 & 10 & 0.8 & 0.2 & 0.0 & 0.8 & 0.8 & 2.0 & 0.7\\
            \cline{1-8}
            Peor & 1.5165 & 0.6594  & 0.6466 & 45 & 0.0001 & 3 & 3 &  &  &  &  &  &  &  & \\
        \hline
        \hline
            Promedio  & 1.5692 & 0.6385 & 0.6609 & 61.0 & 0.0001 & 3.0 & 3 &  &  &  &  &  &  &  & \\
            \cline{1-8}
            Mejor & 1.6992 & 0.5885  & 0.6709 & 45 & 0.0001 & 3 & 3 & 10 & 0.8 & 0.2 & 0.0 & 0.8 & 0.8 & 2.0 & 0.5\\
            \cline{1-8}
            Peor & 1.5165 & 0.6594  & 0.6466 & 45 & 0.0001 & 3 & 3 &  &  &  &  &  &  &  & \\
        \hline
        \end{tabular}
        \caption{Resultados de las mejores corridas de \emph{WPSO} hibridado para {\bf Iris}}
        \label{tb:tablewpsohibcsv}
    \end{center}
\end{table}


\begin{table}[h!]
    \footnotesize
    \begin{center}
        \begin{tabular}{|c|c|c|c|c|c|c|c|c|c|c|c|c|c|c|c|}
        \hline
            & {\bf FO} & {\bf DB} & $J_e$ & {\bf E} & {\bf T} & {\bf KE} & {\bf KO} & $I$ & $w_1$ & $w_2$ & $w_3$ & $W$ & $c_1$ & $c_2$ & $vmx$ \\
        \hline
        \hline
            Promedio  & 1.5749 & 0.6364 & 0.6613 & 61.0 & 0.0001 & 3.0 & 3 &  &  &  &  &  &  &  & \\
            \cline{1-8}
            Mejor & 1.6992 & 0.5885  & 0.6709 & 45 & 0.0001 & 3 & 3 & 10 & 0.8 & 0.2 & 0.0 & 0.8 & 0.8 & 1.7 & 0.9\\
            \cline{1-8}
            Peor & 1.5165 & 0.6594  & 0.6466 & 45 & 0.0001 & 3 & 3 &  &  &  &  &  &  &  & \\
        \hline
        \hline
            Promedio  & 1.5749 & 0.6364 & 0.6613 & 61.0 & 0.0001 & 3.0 & 3 &  &  &  &  &  &  &  & \\
            \cline{1-8}
            Mejor & 1.6992 & 0.5885  & 0.6709 & 45 & 0.0001 & 3 & 3 & 10 & 0.8 & 0.2 & 0.0 & 0.8 & 0.8 & 1.7 & 0.7\\
            \cline{1-8}
            Peor & 1.5165 & 0.6594  & 0.6466 & 45 & 0.0001 & 3 & 3 &  &  &  &  &  &  &  & \\
        \hline
        \hline
            Promedio  & 1.5749 & 0.6364 & 0.6613 & 61.0 & 0.0 & 3.0 & 3 &  &  &  &  &  &  &  & \\
            \cline{1-8}
            Mejor & 1.6992 & 0.5885  & 0.6709 & 45 & 0.0001 & 3 & 3 & 10 & 0.8 & 0.2 & 0.0 & 0.8 & 0.8 & 1.7 & 0.5\\
            \cline{1-8}
            Peor & 1.5165 & 0.6594  & 0.6466 & 45 & 0.0001 & 3 & 3 &  &  &  &  &  &  &  & \\
        \hline
        \hline
            Promedio  & 1.5749 & 0.6364 & 0.6613 & 61.0 & 0.0 & 3.0 & 3 &  &  &  &  &  &  &  & \\
            \cline{1-8}
            Mejor & 1.6992 & 0.5885  & 0.6709 & 45 & 0.0 & 3 & 3 & 10 & 0.8 & 0.2 & 0.0 & 0.8 & 0.8 & 1.4 & 0.9\\
            \cline{1-8}
            Peor & 1.5165 & 0.6594  & 0.6466 & 45 & 0.0001 & 3 & 3 &  &  &  &  &  &  &  & \\
        \hline
        \hline
            Promedio  & 1.5749 & 0.6364 & 0.6613 & 61.0 & 0.0 & 3.0 & 3 &  &  &  &  &  &  &  & \\
            \cline{1-8}
            Mejor & 1.6992 & 0.5885  & 0.6709 & 45 & 0.0001 & 3 & 3 & 10 & 0.8 & 0.2 & 0.0 & 0.8 & 0.8 & 1.4 & 0.7\\
            \cline{1-8}
            Peor & 1.5165 & 0.6594  & 0.6466 & 45 & 0.0001 & 3 & 3 &  &  &  &  &  &  &  & \\
        \hline
        \hline
            Promedio  & 1.5749 & 0.6364 & 0.6613 & 61.0 & 0.0 & 3.0 & 3 &  &  &  &  &  &  &  & \\
            \cline{1-8}
            Mejor & 1.6992 & 0.5885  & 0.6709 & 45 & 0.0 & 3 & 3 & 10 & 0.8 & 0.2 & 0.0 & 0.8 & 0.8 & 1.4 & 0.5\\
            \cline{1-8}
            Peor & 1.5165 & 0.6594  & 0.6466 & 45 & 0.0001 & 3 & 3 &  &  &  &  &  &  &  & \\
        \hline
        \hline
            Promedio  & 1.5748 & 0.6364 & 0.6613 & 59.0 & 0.0 & 3.0 & 3 &  &  &  &  &  &  &  & \\
            \cline{1-8}
            Mejor & 1.6992 & 0.5885  & 0.6709 & 45 & 0.0 & 3 & 3 & 10 & 0.8 & 0.2 & 0.0 & 0.8 & 0.8 & 1.1 & 0.9\\
            \cline{1-8}
            Peor & 1.5165 & 0.6594  & 0.6466 & 45 & 0.0002 & 3 & 3 &  &  &  &  &  &  &  & \\
        \hline
        \hline
            Promedio  & 1.5748 & 0.6364 & 0.6613 & 59.0 & 0.0 & 3.0 & 3 &  &  &  &  &  &  &  & \\
            \cline{1-8}
            Mejor & 1.6992 & 0.5885  & 0.6709 & 45 & 0.0 & 3 & 3 & 10 & 0.8 & 0.2 & 0.0 & 0.8 & 0.8 & 1.1 & 0.7\\
            \cline{1-8}
            Peor & 1.5165 & 0.6594  & 0.6466 & 45 & 0.0001 & 3 & 3 &  &  &  &  &  &  &  & \\
        \hline
        \hline
            Promedio  & 1.5748 & 0.6364 & 0.6613 & 59.0 & 0.0 & 3.0 & 3 &  &  &  &  &  &  &  & \\
            \cline{1-8}
            Mejor & 1.6992 & 0.5885  & 0.6709 & 45 & 0.0 & 3 & 3 & 10 & 0.8 & 0.2 & 0.0 & 0.8 & 0.8 & 1.1 & 0.5\\
            \cline{1-8}
            Peor & 1.5165 & 0.6594  & 0.6466 & 45 & 0.0001 & 3 & 3 &  &  &  &  &  &  &  & \\
        \hline
        \hline
            Promedio  & 1.5748 & 0.6364 & 0.6613 & 59.0 & 0.0 & 3.0 & 3 &  &  &  &  &  &  &  & \\
            \cline{1-8}
            Mejor & 1.6992 & 0.5885  & 0.6709 & 45 & 0.0 & 3 & 3 & 10 & 0.8 & 0.2 & 0.0 & 0.8 & 0.8 & 0.8 & 0.9\\
            \cline{1-8}
            Peor & 1.5165 & 0.6594  & 0.6466 & 45 & 0.0001 & 3 & 3 &  &  &  &  &  &  &  & \\
        \hline
        \end{tabular}
        \caption{Continuacion resultados de las mejores corridas de \emph{WPSO} hibridado para {\bf Iris}}
        \label{tb:tablewpsohibcsvc}
    \end{center}
\end{table}

        
        \end{landscape}

\section{Evolución Diferencial (DEH)}

    \subsection{\emph{NDEH}}\label{sect:ande}

        Las variables del \emph{NDEH}(\ref{sect:ide}) son las siguientes:
        \begin{itemize}
            \item $I$: tamaño de la población. Se varió su valor en el rango
        $[5, 10, \cdots, 40]$.
            \item $w_1$: peso de la distancia intracluster. Se varió en el rango
        $[0.0; 0.1; \cdots; 1.0]$.
            \item $w_2$: peso de distancia intercluster. Se varió en el rango
        $[0.0; 0.1; \cdots; 1.0]$.
            \item $w_3$: peso del error. Se varió en el rango
        $[0.0; 0.1; \cdots; 1.0]$.
        \end{itemize}

        \begin{table}[h!]
    \footnotesize
    \begin{center}
        \begin{tabular}{|c|c|c|c|c|c|c|c|c|c|}
        \hline
            & {\bf FO} & {\bf DB} & $J_e$ & {\bf E} & {\bf T} & $I$ & $w_1$ & $w_2$ & $w_3$ \\
        \hline
        \hline
            Promedio  & 25.6163 & 2.8267 & 23.6834 & 55.0 & 0.1471 &  &  &  & \\
            \cline{1-6}
            Mejor & 29.0553 & 0.915  & 28.4868 & 55 & 0.1384 & 5 & 0.1 & 0.0 & 0.9\\
            \cline{1-6}
            Peor & 22.0208 & 3.1691  & 21.1421 & 55 & 0.1462 &  &  &  & \\
        \hline
        \hline
            Promedio  & 27.4497 & 2.6187 & 27.4497 & 440.0 & 1.1398 &  &  &  & \\
            \cline{1-6}
            Mejor & 29.6483 & 5.1912  & 29.6483 & 440 & 1.1326 & 40 & 0.0 & 0.0 & 1.0\\
            \cline{1-6}
            Peor & 24.5947 & 1.636  & 24.5947 & 440 & 1.1297 &  &  &  & \\
        \hline
        \hline
            Promedio  & 26.5344 & 3.703 & 26.5344 & 220.0 & 0.5725 &  &  &  & \\
            \cline{1-6}
            Mejor & 29.7575 & 2.1092  & 29.7575 & 220 & 0.5693 & 20 & 0.0 & 0.0 & 1.0\\
            \cline{1-6}
            Peor & 24.044 & 1.4744  & 24.044 & 220 & 0.5621 &  &  &  & \\
        \hline
        \hline
            Promedio  & 33.4356 & 1.5385 & 26.7568 & 110.0 & 0.2972 &  &  &  & \\
            \cline{1-6}
            Mejor & 37.1324 & 0.8728  & 29.6328 & 110 & 0.2955 & 10 & 0.5 & 0.0 & 0.5\\
            \cline{1-6}
            Peor & 29.533 & 0.8129  & 26.242 & 110 & 0.2948 &  &  &  & \\
        \hline
        \hline
            Promedio  & 35.2605 & 1.7041 & 28.6247 & 440.0 & 1.1317 &  &  &  & \\
            \cline{1-6}
            Mejor & 38.6722 & 9.3892  & 23.7827 & 440 & 1.1262 & 40 & 0.5 & 0.0 & 0.5\\
            \cline{1-6}
            Peor & 32.8247 & 0.9447  & 30.6038 & 440 & 1.1328 &  &  &  & \\
        \hline
        \hline
            Promedio  & 38.9542 & 1.2159 & 30.5176 & 275.0 & 0.7076 &  &  &  & \\
            \cline{1-6}
            Mejor & 43.2151 & 1.5835  & 21.0709 & 275 & 0.698 & 25 & 0.9 & 0.0 & 0.1\\
            \cline{1-6}
            Peor & 35.8332 & 0.9218  & 29.3292 & 275 & 0.7059 &  &  &  & \\
        \hline
        \hline
            Promedio  & 37.9684 & 1.4636 & 29.6664 & 330.0 & 0.856 &  &  &  & \\
            \cline{1-6}
            Mejor & 43.4044 & 0.9798  & 36.2999 & 330 & 0.8617 & 30 & 0.7 & 0.0 & 0.3\\
            \cline{1-6}
            Peor & 34.9885 & 0.8769  & 29.1156 & 330 & 0.8387 &  &  &  & \\
        \hline
        \hline
            Promedio  & 66.9492 & 1.5745 & 28.701 & 330.0 & 0.8544 &  &  &  & \\
            \cline{1-6}
            Mejor & 70.4104 & 6.811  & 24.7788 & 330 & 0.8533 & 30 & 0.1 & 0.1 & 0.8\\
            \cline{1-6}
            Peor & 65.4498 & 0.8788  & 29.5316 & 330 & 0.8674 &  &  &  & \\
        \hline
        \hline
            Promedio  & 69.1227 & 1.2955 & 28.7276 & 110.0 & 0.2964 &  &  &  & \\
            \cline{1-6}
            Mejor & 73.5159 & 0.9612  & 34.2066 & 110 & 0.2859 & 10 & 0.4 & 0.1 & 0.5\\
            \cline{1-6}
            Peor & 67.0885 & 0.9121  & 29.5655 & 110 & 0.305 &  &  &  & \\
        \hline
        \hline
            Promedio  & 70.1804 & 1.1929 & 29.3415 & 275.0 & 0.7182 &  &  &  & \\
            \cline{1-6}
            Mejor & 76.408 & 2.3852  & 24.2057 & 275 & 0.7119 & 25 & 0.4 & 0.1 & 0.5\\
            \cline{1-6}
            Peor & 67.7243 & 0.9351  & 30.3787 & 275 & 0.7227 &  &  &  & \\
        \hline
        \hline
            Promedio  & 75.4037 & 1.3603 & 30.1216 & 330.0 & 0.8533 &  &  &  & \\
            \cline{1-6}
            Mejor & 81.3028 & 1.1408  & 38.1054 & 330 & 0.8404 & 30 & 0.9 & 0.1 & 0.0\\
            \cline{1-6}
            Peor & 72.7969 & 0.9318  & 31.6591 & 330 & 0.8517 &  &  &  & \\
        \hline
        \hline
            Promedio  & 103.0155 & 0.9822 & 31.9016 & 385.0 & 0.9929 &  &  &  & \\
            \cline{1-6}
            Mejor & 105.6171 & 0.9819  & 38.3743 & 385 & 0.9954 & 35 & 0.1 & 0.2 & 0.7\\
            \cline{1-6}
            Peor & 101.4806 & 0.9038  & 30.1676 & 385 & 0.9698 &  &  &  & \\
        \hline
        \hline
            Promedio  & 105.2711 & 1.0829 & 30.6635 & 440.0 & 1.1401 &  &  &  & \\
            \cline{1-6}
            Mejor & 106.974 & 1.738  & 22.815 & 440 & 1.1177 & 40 & 0.3 & 0.2 & 0.5\\
            \cline{1-6}
            Peor & 103.3265 & 0.9352  & 30.4908 & 440 & 1.1184 &  &  &  & \\
        \hline
        \hline
            Promedio  & 136.6439 & 0.9425 & 32.149 & 385.0 & 0.9958 &  &  &  & \\
            \cline{1-6}
            Mejor & 139.0367 & 1.0063  & 40.6607 & 385 & 0.9951 & 35 & 0.0 & 0.3 & 0.7\\
            \cline{1-6}
            Peor & 135.6574 & 0.9351  & 30.9094 & 385 & 0.9923 &  &  &  & \\
        \hline
        \hline
            Promedio  & 144.2327 & 0.9457 & 32.7844 & 385.0 & 0.9958 &  &  &  & \\
            \cline{1-6}
            Mejor & 147.1217 & 1.042  & 35.2635 & 385 & 1.0027 & 35 & 0.7 & 0.3 & 0.0\\
            \cline{1-6}
            Peor & 141.9999 & 0.9709  & 34.7711 & 385 & 1.004 &  &  &  & \\
        \hline
        \end{tabular}
        \caption{Resultados de las mejores corridas de \emph{NDE} no hibridado para {\bf Lenna}}
        \label{tb:tabledealgimg}
    \end{center}
\end{table}


\begin{table}[h!]
    \footnotesize
    \begin{center}
        \begin{tabular}{|c|c|c|c|c|c|c|c|c|c|}
        \hline
            & {\bf FO} & {\bf DB} & $J_e$ & {\bf E} & {\bf T} & $I$ & $w_1$ & $w_2$ & $w_3$ \\
        \hline
        \hline
            Promedio  & 172.3278 & 0.9242 & 31.6471 & 385.0 & 0.9949 &  &  &  & \\
            \cline{1-6}
            Mejor & 173.2183 & 0.9916  & 35.9413 & 385 & 0.995 & 35 & 0.1 & 0.4 & 0.5\\
            \cline{1-6}
            Peor & 171.6772 & 0.8264  & 28.1707 & 385 & 1.0022 &  &  &  & \\
        \hline
        \hline
            Promedio  & 207.092 & 0.9377 & 33.2412 & 55.0 & 0.1488 &  &  &  & \\
            \cline{1-6}
            Mejor & 209.4944 & 1.0265  & 34.9178 & 55 & 0.1589 & 5 & 0.2 & 0.5 & 0.3\\
            \cline{1-6}
            Peor & 205.4 & 0.952  & 33.3773 & 55 & 0.1518 &  &  &  & \\
        \hline
        \hline
            Promedio  & 241.2163 & 0.9139 & 31.5335 & 55.0 & 0.1506 &  &  &  & \\
            \cline{1-6}
            Mejor & 243.0256 & 0.9758  & 32.092 & 55 & 0.1443 & 5 & 0.2 & 0.6 & 0.2\\
            \cline{1-6}
            Peor & 238.8899 & 0.8127  & 27.0657 & 55 & 0.1593 &  &  &  & \\
        \hline
        \hline
            Promedio  & 241.2897 & 0.9544 & 34.9897 & 275.0 & 0.7123 &  &  &  & \\
            \cline{1-6}
            Mejor & 243.0642 & 0.961  & 31.7659 & 275 & 0.7206 & 25 & 0.1 & 0.6 & 0.3\\
            \cline{1-6}
            Peor & 240.2025 & 1.0866  & 39.8763 & 275 & 0.7145 &  &  &  & \\
        \hline
        \hline
            Promedio  & 243.6882 & 0.927 & 31.7937 & 385.0 & 0.9904 &  &  &  & \\
            \cline{1-6}
            Mejor & 245.1328 & 1.0059  & 35.13 & 385 & 0.9933 & 35 & 0.3 & 0.6 & 0.1\\
            \cline{1-6}
            Peor & 242.7763 & 0.9723  & 35.6122 & 385 & 0.9979 &  &  &  & \\
        \hline
        \end{tabular}
        \caption{Continuacion resultados de las mejores corridas de \emph{NDE} no hibridado para {\bf Lenna}}
        \label{tb:tabledealgimgc}
    \end{center}
\end{table}


        \begin{table}[h!]
    \footnotesize
    \begin{center}
        \begin{tabular}{|c|c|c|c|c|c|c|c|c|c|c|c|}
        \hline
            & {\bf FO} & {\bf DB} & $J_e$ & {\bf E} & {\bf T} & {\bf KE} & {\bf KO} & $I$ & $w_1$ & $w_2$ & $w_3$ \\
        \hline
        \hline
            Promedio  & 1.2132 & 0.8256 & 19.9024 & 452.2333 & 0.0155 & 9.0 & $[5-10]$ &  &  &  & \\
            \cline{1-8}
            Mejor & 1.3346 & 0.7493  & 16.589 & 447 & 0.0084 & 9 & $[5-10]$ & 40 & 0.0 & 0.0 & 1.0\\
            \cline{1-8}
            Peor & 1.1097 & 0.9011  & 17.6334 & 447 & 0.0081 & 9 & $[5-10]$ &  &  &  & \\
        \hline
        \hline
            Promedio  & 1.1621 & 0.8635 & 23.5036 & 394.9333 & 0.0122 & 9.0 & $[5-10]$ &  &  &  & \\
            \cline{1-8}
            Mejor & 1.3117 & 0.7623  & 18.0484 & 390 & 0.0054 & 9 & $[5-10]$ & 35 & 0.7 & 0.3 & 0.0\\
            \cline{1-8}
            Peor & 1.03 & 0.9709  & 34.7711 & 389 & 0.0041 & 9 & $[5-10]$ &  &  &  & \\
        \hline
        \hline
            Promedio  & 1.1945 & 0.839 & 20.4497 & 122.3 & 0.0154 & 9.0 & $[5-10]$ &  &  &  & \\
            \cline{1-8}
            Mejor & 1.3095 & 0.7637  & 17.4098 & 117 & 0.0083 & 9 & $[5-10]$ & 10 & 0.5 & 0.0 & 0.5\\
            \cline{1-8}
            Peor & 1.0834 & 0.923  & 18.2255 & 117 & 0.0082 & 9 & $[5-10]$ &  &  &  & \\
        \hline
        \hline
            Promedio  & 1.184 & 0.8461 & 19.7641 & 452.6667 & 0.0159 & 9.0 & $[5-10]$ &  &  &  & \\
            \cline{1-8}
            Mejor & 1.2845 & 0.7785  & 16.785 & 463 & 0.0306 & 9 & $[5-10]$ & 40 & 0.5 & 0.0 & 0.5\\
            \cline{1-8}
            Peor & 1.0812 & 0.9249  & 16.977 & 446 & 0.0071 & 9 & $[5-10]$ &  &  &  & \\
        \hline
        \hline
            Promedio  & 1.1804 & 0.8493 & 21.0635 & 123.3667 & 0.0169 & 9.0 & $[5-10]$ &  &  &  & \\
            \cline{1-8}
            Mejor & 1.2834 & 0.7792  & 16.782 & 135 & 0.0333 & 9 & $[5-10]$ & 10 & 0.4 & 0.1 & 0.5\\
            \cline{1-8}
            Peor & 1.0707 & 0.934  & 31.4136 & 114 & 0.0041 & 9 & $[5-10]$ &  &  &  & \\
        \hline
        \hline
            Promedio  & 1.1527 & 0.8702 & 22.0432 & 284.9 & 0.0121 & 9.0 & $[5-10]$ &  &  &  & \\
            \cline{1-8}
            Mejor & 1.2823 & 0.7799  & 16.7893 & 291 & 0.021 & 9 & $[5-10]$ & 25 & 0.9 & 0.0 & 0.1\\
            \cline{1-8}
            Peor & 1.0355 & 0.9657  & 34.7351 & 279 & 0.0042 & 9 & $[5-10]$ &  &  &  & \\
        \hline
        \hline
            Promedio  & 1.1967 & 0.8375 & 18.5967 & 234.4333 & 0.019 & 9.0 & $[5-10]$ &  &  &  & \\
            \cline{1-8}
            Mejor & 1.2817 & 0.7802  & 17.0026 & 228 & 0.0097 & 9 & $[5-10]$ & 20 & 0.0 & 0.0 & 1.0\\
            \cline{1-8}
            Peor & 1.0493 & 0.953  & 17.8524 & 231 & 0.0136 & 9 & $[5-10]$ &  &  &  & \\
        \hline
        \hline
            Promedio  & 1.1769 & 0.8531 & 21.0599 & 339.7667 & 0.0123 & 9.0 & $[5-10]$ &  &  &  & \\
            \cline{1-8}
            Mejor & 1.2799 & 0.7813  & 17.2809 & 344 & 0.0177 & 9 & $[5-10]$ & 30 & 0.1 & 0.1 & 0.8\\
            \cline{1-8}
            Peor & 1.0214 & 0.9791  & 31.9754 & 334 & 0.0042 & 9 & $[5-10]$ &  &  &  & \\
        \hline
        \hline
            Promedio  & 1.1572 & 0.8671 & 21.4641 & 285.0 & 0.0122 & 9.0 & $[5-10]$ &  &  &  & \\
            \cline{1-8}
            Mejor & 1.2775 & 0.7828  & 16.8505 & 290 & 0.0197 & 9 & $[5-10]$ & 25 & 0.4 & 0.1 & 0.5\\
            \cline{1-8}
            Peor & 1.0185 & 0.9819  & 32.8367 & 279 & 0.0043 & 9 & $[5-10]$ &  &  &  & \\
        \hline
        \hline
            Promedio  & 1.1879 & 0.8438 & 21.1817 & 65.6667 & 0.0131 & 9.0 & $[5-10]$ &  &  &  & \\
            \cline{1-8}
            Mejor & 1.2731 & 0.7855  & 25.2148 & 59 & 0.004 & 9 & $[5-10]$ & 5 & 0.2 & 0.6 & 0.2\\
            \cline{1-8}
            Peor & 1.0645 & 0.9394  & 19.2058 & 60 & 0.0059 & 9 & $[5-10]$ &  &  &  & \\
        \hline
        \hline
            Promedio  & 1.1799 & 0.8506 & 20.1871 & 341.8667 & 0.0148 & 9.0 & $[5-10]$ &  &  &  & \\
            \cline{1-8}
            Mejor & 1.2692 & 0.7879  & 16.8357 & 347 & 0.0223 & 9 & $[5-10]$ & 30 & 0.9 & 0.1 & 0.0\\
            \cline{1-8}
            Peor & 0.9391 & 1.0649  & 17.1959 & 335 & 0.0056 & 9 & $[5-10]$ &  &  &  & \\
        \hline
        \hline
            Promedio  & 1.1614 & 0.8641 & 20.3254 & 340.3333 & 0.0128 & 9.0 & $[5-10]$ &  &  &  & \\
            \cline{1-8}
            Mejor & 1.2685 & 0.7883  & 18.3452 & 335 & 0.0053 & 9 & $[5-10]$ & 30 & 0.7 & 0.0 & 0.3\\
            \cline{1-8}
            Peor & 1.0154 & 0.9848  & 17.2159 & 339 & 0.0115 & 9 & $[5-10]$ &  &  &  & \\
        \hline
        \hline
            Promedio  & 1.1946 & 0.8389 & 17.7597 & 68.8333 & 0.018 & 9.0 & $[5-10]$ &  &  &  & \\
            \cline{1-8}
            Mejor & 1.2682 & 0.7885  & 16.7318 & 70 & 0.0196 & 9 & $[5-10]$ & 5 & 0.1 & 0.0 & 0.9\\
            \cline{1-8}
            Peor & 1.0679 & 0.9364  & 17.2664 & 63 & 0.0096 & 9 & $[5-10]$ &  &  &  & \\
        \hline
        \hline
            Promedio  & 1.1651 & 0.8605 & 19.6724 & 393.4667 & 0.0101 & 9.0 & $[5-10]$ &  &  &  & \\
            \cline{1-8}
            Mejor & 1.2604 & 0.7934  & 17.5164 & 400 & 0.0188 & 9 & $[5-10]$ & 35 & 0.0 & 0.3 & 0.7\\
            \cline{1-8}
            Peor & 1.01 & 0.9901  & 31.6365 & 389 & 0.004 & 9 & $[5-10]$ &  &  &  & \\
        \hline
        \hline
            Promedio  & 1.1589 & 0.8655 & 22.7628 & 394.1667 & 0.0112 & 9.0 & $[5-10]$ &  &  &  & \\
            \cline{1-8}
            Mejor & 1.2599 & 0.7937  & 17.0567 & 395 & 0.0124 & 9 & $[5-10]$ & 35 & 0.3 & 0.6 & 0.1\\
            \cline{1-8}
            Peor & 1.0285 & 0.9723  & 35.6122 & 389 & 0.0043 & 9 & $[5-10]$ &  &  &  & \\
        \hline
        \end{tabular}
        \caption{Resultados de las mejores corridas de \emph{NDE} hibridado para {\bf Lenna}}
        \label{tb:tabledehibimg}
    \end{center}
\end{table}


\begin{table}[h!]
    \footnotesize
    \begin{center}
        \begin{tabular}{|c|c|c|c|c|c|c|c|c|c|c|c|}
        \hline
            & {\bf FO} & {\bf DB} & $J_e$ & {\bf E} & {\bf T} & {\bf KE} & {\bf KO} & $I$ & $w_1$ & $w_2$ & $w_3$ \\
        \hline
        \hline
            Promedio  & 1.1575 & 0.8662 & 22.5359 & 394.6 & 0.0117 & 9.0 & $[5-10]$ &  &  &  & \\
            \cline{1-8}
            Mejor & 1.2576 & 0.7952  & 17.0407 & 404 & 0.0245 & 9 & $[5-10]$ & 35 & 0.1 & 0.2 & 0.7\\
            \cline{1-8}
            Peor & 1.0587 & 0.9446  & 18.5563 & 390 & 0.0053 & 9 & $[5-10]$ &  &  &  & \\
        \hline
        \hline
            Promedio  & 1.1616 & 0.8647 & 20.9612 & 448.9 & 0.0109 & 9.0 & $[5-10]$ &  &  &  & \\
            \cline{1-8}
            Mejor & 1.2543 & 0.7973  & 17.0045 & 463 & 0.0303 & 9 & $[5-10]$ & 40 & 0.3 & 0.2 & 0.5\\
            \cline{1-8}
            Peor & 1.008 & 0.992  & 29.0253 & 444 & 0.0041 & 9 & $[5-10]$ &  &  &  & \\
        \hline
        \hline
            Promedio  & 1.1757 & 0.8523 & 21.2353 & 66.1333 & 0.0139 & 9.0 & $[5-10]$ &  &  &  & \\
            \cline{1-8}
            Mejor & 1.2537 & 0.7976  & 17.1086 & 83 & 0.0366 & 9 & $[5-10]$ & 5 & 0.2 & 0.5 & 0.3\\
            \cline{1-8}
            Peor & 1.0848 & 0.9218  & 17.9777 & 61 & 0.0067 & 9 & $[5-10]$ &  &  &  & \\
        \hline
        \hline
            Promedio  & 1.1803 & 0.8489 & 20.5585 & 285.0 & 0.0122 & 9.0 & $[5-10]$ &  &  &  & \\
            \cline{1-8}
            Mejor & 1.253 & 0.7981  & 17.6771 & 281 & 0.0068 & 9 & $[5-10]$ & 25 & 0.1 & 0.6 & 0.3\\
            \cline{1-8}
            Peor & 1.0746 & 0.9306  & 19.218 & 280 & 0.0055 & 9 & $[5-10]$ &  &  &  & \\
        \hline
        \hline
            Promedio  & 1.1746 & 0.8527 & 21.8228 & 394.7 & 0.0118 & 9.0 & $[5-10]$ &  &  &  & \\
            \cline{1-8}
            Mejor & 1.2519 & 0.7988  & 16.9593 & 398 & 0.0166 & 9 & $[5-10]$ & 35 & 0.1 & 0.4 & 0.5\\
            \cline{1-8}
            Peor & 1.0524 & 0.9502  & 29.8301 & 389 & 0.0041 & 9 & $[5-10]$ &  &  &  & \\
        \hline
        \end{tabular}
        \caption{Continuacion resultados de las mejores corridas de \emph{NDE} hibridado para {\bf Lenna}}
        \label{tb:tabledehibimgc}
    \end{center}
\end{table}


        \begin{table}[h!]
    \footnotesize
    \begin{center}
        \begin{tabular}{|c|c|c|c|c|c|c|c|c|c|}
        \hline
            & {\bf FO} & {\bf DB} & $J_e$ & {\bf E} & {\bf T} & $I$ & $w_1$ & $w_2$ & $w_3$ \\
        \hline
        \hline
            Promedio  & 1.3686 & 1.9751 & 1.3196 & 220.0 & 0.0025 &  &  &  & \\
            \cline{1-6}
            Mejor & 1.548 & 1.5597  & 1.4866 & 220 & 0.0018 & 20 & 0.1 & 0.0 & 0.9\\
            \cline{1-6}
            Peor & 1.251 & 2.9174  & 1.2188 & 220 & 0.0023 &  &  &  & \\
        \hline
        \hline
            Promedio  & 1.5653 & 2.0878 & 1.3291 & 220.0 & 0.003 &  &  &  & \\
            \cline{1-6}
            Mejor & 1.6851 & 1.5748  & 1.556 & 220 & 0.0027 & 20 & 0.4 & 0.0 & 0.6\\
            \cline{1-6}
            Peor & 1.4997 & 2.1788  & 1.124 & 220 & 0.002 &  &  &  & \\
        \hline
        \hline
            Promedio  & 1.7674 & 1.8494 & 1.4592 & 385.0 & 0.0049 &  &  &  & \\
            \cline{1-6}
            Mejor & 1.9107 & 1.4334  & 1.5167 & 385 & 0.0045 & 35 & 0.5 & 0.0 & 0.5\\
            \cline{1-6}
            Peor & 1.7194 & 2.4495  & 1.4299 & 385 & 0.0046 &  &  &  & \\
        \hline
        \hline
            Promedio  & 1.8375 & 1.7808 & 1.5569 & 330.0 & 0.0043 &  &  &  & \\
            \cline{1-6}
            Mejor & 2.0586 & 8.6972  & 1.2974 & 330 & 0.0037 & 30 & 0.7 & 0.0 & 0.3\\
            \cline{1-6}
            Peor & 1.735 & 1.9965  & 1.1342 & 330 & 0.0039 &  &  &  & \\
        \hline
        \hline
            Promedio  & 2.2334 & 1.3086 & 1.7088 & 385.0 & 0.0048 &  &  &  & \\
            \cline{1-6}
            Mejor & 2.3861 & 1.6036  & 1.9633 & 385 & 0.0037 & 35 & 0.8 & 0.1 & 0.1\\
            \cline{1-6}
            Peor & 2.1452 & 0.9462  & 1.6568 & 385 & 0.0039 &  &  &  & \\
        \hline
        \hline
            Promedio  & 2.2568 & 1.3086 & 1.7088 & 385.0 & 0.0046 &  &  &  & \\
            \cline{1-6}
            Mejor & 2.3948 & 1.6036  & 1.9633 & 385 & 0.0042 & 35 & 0.9 & 0.1 & 0.0\\
            \cline{1-6}
            Peor & 2.1709 & 0.9462  & 1.6568 & 385 & 0.0041 &  &  &  & \\
        \hline
        \hline
            Promedio  & 2.6959 & 1.1832 & 1.8426 & 440.0 & 0.0057 &  &  &  & \\
            \cline{1-6}
            Mejor & 2.8083 & 1.2968  & 1.639 & 440 & 0.0054 & 40 & 0.0 & 0.3 & 0.7\\
            \cline{1-6}
            Peor & 2.6407 & 1.0657  & 2.0503 & 440 & 0.0055 &  &  &  & \\
        \hline
        \hline
            Promedio  & 2.5764 & 1.0815 & 1.4964 & 110.0 & 0.0013 &  &  &  & \\
            \cline{1-6}
            Mejor & 2.8517 & 1.3148  & 1.8682 & 110 & 0.0027 & 10 & 0.0 & 0.3 & 0.7\\
            \cline{1-6}
            Peor & 2.4996 & 1.0323  & 1.4185 & 110 & 0.0011 &  &  &  & \\
        \hline
        \hline
            Promedio  & 2.617 & 1.0815 & 1.4964 & 110.0 & 0.0012 &  &  &  & \\
            \cline{1-6}
            Mejor & 2.8712 & 1.3148  & 1.8682 & 110 & 0.0009 & 10 & 0.1 & 0.3 & 0.6\\
            \cline{1-6}
            Peor & 2.559 & 1.0622  & 1.5155 & 110 & 0.0011 &  &  &  & \\
        \hline
        \hline
            Promedio  & 2.9949 & 1.1241 & 1.6599 & 110.0 & 0.0011 &  &  &  & \\
            \cline{1-6}
            Mejor & 3.2184 & 1.3148  & 1.8682 & 110 & 0.0026 & 10 & 0.2 & 0.4 & 0.4\\
            \cline{1-6}
            Peor & 2.9032 & 1.0306  & 1.6751 & 110 & 0.0009 &  &  &  & \\
        \hline
        \hline
            Promedio  & 3.0227 & 1.1179 & 1.7259 & 110.0 & 0.0011 &  &  &  & \\
            \cline{1-6}
            Mejor & 3.2378 & 1.3148  & 1.8682 & 110 & 0.0027 & 10 & 0.3 & 0.4 & 0.3\\
            \cline{1-6}
            Peor & 2.9463 & 1.0397  & 2.0003 & 110 & 0.0009 &  &  &  & \\
        \hline
        \hline
            Promedio  & 3.0644 & 1.0797 & 1.4338 & 110.0 & 0.0013 &  &  &  & \\
            \cline{1-6}
            Mejor & 3.2572 & 1.3148  & 1.8682 & 110 & 0.0027 & 10 & 0.4 & 0.4 & 0.2\\
            \cline{1-6}
            Peor & 2.9912 & 1.0366  & 1.4243 & 110 & 0.001 &  &  &  & \\
        \hline
        \hline
            Promedio  & 3.0859 & 1.0791 & 1.4551 & 110.0 & 0.0011 &  &  &  & \\
            \cline{1-6}
            Mejor & 3.2766 & 1.3148  & 1.8682 & 110 & 0.0025 & 10 & 0.5 & 0.4 & 0.1\\
            \cline{1-6}
            Peor & 2.9986 & 1.0366  & 1.4243 & 110 & 0.001 &  &  &  & \\
        \hline
        \hline
            Promedio  & 3.4134 & 1.1023 & 1.5085 & 110.0 & 0.0011 &  &  &  & \\
            \cline{1-6}
            Mejor & 3.5851 & 1.3148  & 1.8682 & 110 & 0.0027 & 10 & 0.4 & 0.5 & 0.1\\
            \cline{1-6}
            Peor & 3.3778 & 1.0368  & 1.5958 & 110 & 0.0009 &  &  &  & \\
        \hline
        \hline
            Promedio  & 3.4418 & 1.0994 & 1.5444 & 110.0 & 0.0012 &  &  &  & \\
            \cline{1-6}
            Mejor & 3.6045 & 1.3148  & 1.8682 & 110 & 0.0027 & 10 & 0.5 & 0.5 & 0.0\\
            \cline{1-6}
            Peor & 3.4046 & 1.0368  & 1.5958 & 110 & 0.0009 &  &  &  & \\
        \hline
        \end{tabular}
        \caption{Resultados de las mejores corridas de \emph{NDE} no hibridado para {\bf Iris}}
        \label{tb:tabledealgcsv}
    \end{center}
\end{table}


\begin{table}[h!]
    \footnotesize
    \begin{center}
        \begin{tabular}{|c|c|c|c|c|c|c|c|c|c|}
        \hline
            & {\bf FO} & {\bf DB} & $J_e$ & {\bf E} & {\bf T} & $I$ & $w_1$ & $w_2$ & $w_3$ \\
        \hline
        \hline
            Promedio  & 3.9341 & 1.2823 & 1.6793 & 165.0 & 0.0021 &  &  &  & \\
            \cline{1-6}
            Mejor & 4.0658 & 1.5149  & 1.9933 & 165 & 0.0019 & 15 & 0.4 & 0.6 & 0.0\\
            \cline{1-6}
            Peor & 3.7721 & 1.2482  & 1.8069 & 165 & 0.0013 &  &  &  & \\
        \hline
        \hline
            Promedio  & 4.204 & 1.4093 & 1.9482 & 165.0 & 0.0021 &  &  &  & \\
            \cline{1-6}
            Mejor & 4.3878 & 1.8565  & 2.2687 & 165 & 0.0021 & 15 & 0.2 & 0.7 & 0.1\\
            \cline{1-6}
            Peor & 3.9798 & 1.3428  & 2.0975 & 165 & 0.002 &  &  &  & \\
        \hline
        \hline
            Promedio  & 4.4479 & 1.5681 & 2.1235 & 165.0 & 0.0021 &  &  &  & \\
            \cline{1-6}
            Mejor & 4.655 & 1.8565  & 2.2687 & 165 & 0.0018 & 15 & 0.1 & 0.8 & 0.1\\
            \cline{1-6}
            Peor & 4.2105 & 1.3428  & 2.0975 & 165 & 0.002 &  &  &  & \\
        \hline
        \hline
            Promedio  & 4.4938 & 1.5681 & 2.1235 & 165.0 & 0.0021 &  &  &  & \\
            \cline{1-6}
            Mejor & 4.6826 & 1.8565  & 2.2687 & 165 & 0.0018 & 15 & 0.2 & 0.8 & 0.0\\
            \cline{1-6}
            Peor & 4.2402 & 1.3428  & 2.0975 & 165 & 0.0019 &  &  &  & \\
        \hline
        \hline
            Promedio  & 4.7327 & 1.5681 & 2.1235 & 165.0 & 0.0021 &  &  &  & \\
            \cline{1-6}
            Mejor & 4.9498 & 1.8565  & 2.2687 & 165 & 0.0019 & 15 & 0.1 & 0.9 & 0.0\\
            \cline{1-6}
            Peor & 4.4709 & 1.3428  & 2.0975 & 165 & 0.002 &  &  &  & \\
        \hline
        \end{tabular}
        \caption{Continuacion resultados de las mejores corridas de \emph{NDE} no hibridado para {\bf Iris}}
        \label{tb:tabledealgcsvc}
    \end{center}
\end{table}

        
        \begin{table}[h!]
    \footnotesize
    \begin{center}
        \begin{tabular}{|c|c|c|c|c|c|c|c|c|c|c|c|}
        \hline
            & {\bf FO} & {\bf DB} & $J_e$ & {\bf E} & {\bf T} & {\bf KE} & {\bf KO} & $I$ & $w_1$ & $w_2$ & $w_3$ \\
        \hline
        \hline
            Promedio  & 1.5638 & 0.6403 & 0.6367 & 172.4 & 0.0001 & 3.0 & 3 &  &  &  & \\
            \cline{1-8}
            Mejor & 1.7153 & 0.583  & 0.6586 & 170 & 0.0001 & 3 & 3 & 15 & 0.2 & 0.7 & 0.1\\
            \cline{1-8}
            Peor & 1.5006 & 0.6664  & 0.649 & 176 & 0.0001 & 3 & 3 &  &  &  & \\
        \hline
        \hline
            Promedio  & 1.5559 & 0.6441 & 0.6437 & 173.8667 & 0.0001 & 3.0 & 3 &  &  &  & \\
            \cline{1-8}
            Mejor & 1.6975 & 0.5891  & 0.6339 & 173 & 0.0001 & 3 & 3 & 15 & 0.4 & 0.6 & 0.0\\
            \cline{1-8}
            Peor & 1.5006 & 0.6664  & 0.649 & 174 & 0.0002 & 3 & 3 &  &  &  & \\
        \hline
        \hline
            Promedio  & 1.5749 & 0.6359 & 0.671 & 336.5333 & 0.0001 & 3.0 & 3 &  &  &  & \\
            \cline{1-8}
            Mejor & 1.6622 & 0.6016  & 0.6343 & 337 & 0.0001 & 3 & 3 & 30 & 0.7 & 0.0 & 0.3\\
            \cline{1-8}
            Peor & 1.5006 & 0.6664  & 0.649 & 341 & 0.0002 & 3 & 3 &  &  &  & \\
        \hline
        \hline
            Promedio  & 1.5395 & 0.6502 & 0.6451 & 228.0333 & 0.0001 & 3.0 & 3 &  &  &  & \\
            \cline{1-8}
            Mejor & 1.6382 & 0.6104  & 0.6269 & 226 & 0.0001 & 3 & 3 & 20 & 0.4 & 0.0 & 0.6\\
            \cline{1-8}
            Peor & 1.5006 & 0.6664  & 0.649 & 231 & 0.0001 & 3 & 3 &  &  &  & \\
        \hline
        \hline
            Promedio  & 1.5312 & 0.6539 & 0.6617 & 226.9667 & 0.0001 & 3.0 & 3 &  &  &  & \\
            \cline{1-8}
            Mejor & 1.6242 & 0.6157  & 0.7124 & 226 & 0.0001 & 3 & 3 & 20 & 0.1 & 0.0 & 0.9\\
            \cline{1-8}
            Peor & 1.4738 & 0.6785  & 0.6408 & 226 & 0.0001 & 3 & 3 &  &  &  & \\
        \hline
        \hline
            Promedio  & 1.5526 & 0.6444 & 0.6356 & 172.0667 & 0.0001 & 3.0 & 3 &  &  &  & \\
            \cline{1-8}
            Mejor & 1.5967 & 0.6263  & 0.624 & 171 & 0.0001 & 3 & 3 & 15 & 0.1 & 0.9 & 0.0\\
            \cline{1-8}
            Peor & 1.5006 & 0.6664  & 0.649 & 177 & 0.0001 & 3 & 3 &  &  &  & \\
        \hline
        \hline
            Promedio  & 1.5526 & 0.6444 & 0.6356 & 172.0667 & 0.0001 & 3.0 & 3 &  &  &  & \\
            \cline{1-8}
            Mejor & 1.5967 & 0.6263  & 0.624 & 171 & 0.0001 & 3 & 3 & 15 & 0.2 & 0.8 & 0.0\\
            \cline{1-8}
            Peor & 1.5006 & 0.6664  & 0.649 & 177 & 0.0001 & 3 & 3 &  &  &  & \\
        \hline
        \hline
            Promedio  & 1.5526 & 0.6444 & 0.6356 & 172.0667 & 0.0001 & 3.0 & 3 &  &  &  & \\
            \cline{1-8}
            Mejor & 1.5967 & 0.6263  & 0.624 & 171 & 0.0001 & 3 & 3 & 15 & 0.1 & 0.8 & 0.1\\
            \cline{1-8}
            Peor & 1.5006 & 0.6664  & 0.649 & 177 & 0.0001 & 3 & 3 &  &  &  & \\
        \hline
        \hline
            Promedio  & 1.5372 & 0.6509 & 0.6349 & 116.7333 & 0.0 & 3.0 & 3 &  &  &  & \\
            \cline{1-8}
            Mejor & 1.586 & 0.6305  & 0.6254 & 115 & 0.0 & 3 & 3 & 10 & 0.5 & 0.5 & 0.0\\
            \cline{1-8}
            Peor & 1.5006 & 0.6664  & 0.649 & 120 & 0.0001 & 3 & 3 &  &  &  & \\
        \hline
        \hline
            Promedio  & 1.5273 & 0.6551 & 0.639 & 117.0667 & 0.0001 & 3.0 & 3 &  &  &  & \\
            \cline{1-8}
            Mejor & 1.586 & 0.6305  & 0.6254 & 115 & 0.0 & 3 & 3 & 10 & 0.4 & 0.5 & 0.1\\
            \cline{1-8}
            Peor & 1.5006 & 0.6664  & 0.649 & 120 & 0.0001 & 3 & 3 &  &  &  & \\
        \hline
        \hline
            Promedio  & 1.506 & 0.6643 & 0.6494 & 446.5 & 0.0001 & 3.0 & 3 &  &  &  & \\
            \cline{1-8}
            Mejor & 1.5639 & 0.6394  & 0.692 & 445 & 0.0001 & 3 & 3 & 40 & 0.0 & 0.3 & 0.7\\
            \cline{1-8}
            Peor & 1.4591 & 0.6854  & 0.6438 & 446 & 0.0001 & 3 & 3 &  &  &  & \\
        \hline
        \hline
            Promedio  & 1.5244 & 0.6561 & 0.6403 & 116.1 & 0.0001 & 3.0 & 3 &  &  &  & \\
            \cline{1-8}
            Mejor & 1.5609 & 0.6407  & 0.627 & 116 & 0.0 & 3 & 3 & 10 & 0.4 & 0.4 & 0.2\\
            \cline{1-8}
            Peor & 1.5056 & 0.6642  & 0.635 & 116 & 0.0001 & 3 & 3 &  &  &  & \\
        \hline
        \hline
            Promedio  & 1.525 & 0.6558 & 0.6402 & 116.1 & 0.0 & 3.0 & 3 &  &  &  & \\
            \cline{1-8}
            Mejor & 1.5609 & 0.6407  & 0.627 & 116 & 0.0 & 3 & 3 & 10 & 0.5 & 0.4 & 0.1\\
            \cline{1-8}
            Peor & 1.5056 & 0.6642  & 0.635 & 116 & 0.0 & 3 & 3 &  &  &  & \\
        \hline
        \hline
            Promedio  & 1.5241 & 0.6563 & 0.6401 & 116.3333 & 0.0 & 3.0 & 3 &  &  &  & \\
            \cline{1-8}
            Mejor & 1.5609 & 0.6407  & 0.627 & 116 & 0.0 & 3 & 3 & 10 & 0.3 & 0.4 & 0.3\\
            \cline{1-8}
            Peor & 1.5056 & 0.6642  & 0.635 & 116 & 0.0 & 3 & 3 &  &  &  & \\
        \hline
        \hline
            Promedio  & 1.5258 & 0.6555 & 0.6394 & 116.5333 & 0.0 & 3.0 & 3 &  &  &  & \\
            \cline{1-8}
            Mejor & 1.5609 & 0.6407  & 0.627 & 116 & 0.0 & 3 & 3 & 10 & 0.2 & 0.4 & 0.4\\
            \cline{1-8}
            Peor & 1.5056 & 0.6642  & 0.635 & 116 & 0.0 & 3 & 3 &  &  &  & \\
        \hline
        \end{tabular}
        \caption{Resultados de las mejores corridas de \emph{NDE} hibridado para {\bf Iris}}
        \label{tb:tabledehibcsv}
    \end{center}
\end{table}


\begin{table}[h!]
    \footnotesize
    \begin{center}
        \begin{tabular}{|c|c|c|c|c|c|c|c|c|c|c|c|}
        \hline
            & {\bf FO} & {\bf DB} & $J_e$ & {\bf E} & {\bf T} & {\bf KE} & {\bf KO} & $I$ & $w_1$ & $w_2$ & $w_3$ \\
        \hline
        \hline
            Promedio  & 1.5183 & 0.6587 & 0.6403 & 116.4333 & 0.0001 & 3.0 & 3 &  &  &  & \\
            \cline{1-8}
            Mejor & 1.5517 & 0.6445  & 0.6292 & 115 & 0.0 & 3 & 3 & 10 & 0.1 & 0.3 & 0.6\\
            \cline{1-8}
            Peor & 1.5006 & 0.6664  & 0.649 & 121 & 0.0001 & 3 & 3 &  &  &  & \\
        \hline
        \hline
            Promedio  & 1.5183 & 0.6587 & 0.6403 & 116.4333 & 0.0001 & 3.0 & 3 &  &  &  & \\
            \cline{1-8}
            Mejor & 1.5517 & 0.6445  & 0.6292 & 115 & 0.0 & 3 & 3 & 10 & 0.0 & 0.3 & 0.7\\
            \cline{1-8}
            Peor & 1.5006 & 0.6664  & 0.649 & 121 & 0.0001 & 3 & 3 &  &  &  & \\
        \hline
        \hline
            Promedio  & 1.5192 & 0.6582 & 0.6468 & 392.7333 & 0.0001 & 3.0 & 3 &  &  &  & \\
            \cline{1-8}
            Mejor & 1.5286 & 0.6542  & 0.6462 & 393 & 0.0001 & 3 & 3 & 35 & 0.5 & 0.0 & 0.5\\
            \cline{1-8}
            Peor & 1.5098 & 0.6623  & 0.6474 & 393 & 0.0001 & 3 & 3 &  &  &  & \\
        \hline
        \hline
            Promedio  & 1.3969 & 0.728 & 0.6485 & 392.2 & 0.0001 & 3.0 & 3 &  &  &  & \\
            \cline{1-8}
            Mejor & 1.5098 & 0.6623  & 0.6474 & 391 & 0.0001 & 3 & 3 & 35 & 0.9 & 0.1 & 0.0\\
            \cline{1-8}
            Peor & 1.0961 & 0.9123  & 0.6538 & 391 & 0.0001 & 3 & 3 &  &  &  & \\
        \hline
        \hline
            Promedio  & 1.3969 & 0.728 & 0.6485 & 392.2 & 0.0001 & 3.0 & 3 &  &  &  & \\
            \cline{1-8}
            Mejor & 1.5098 & 0.6623  & 0.6474 & 391 & 0.0001 & 3 & 3 & 35 & 0.8 & 0.1 & 0.1\\
            \cline{1-8}
            Peor & 1.0961 & 0.9123  & 0.6538 & 391 & 0.0 & 3 & 3 &  &  &  & \\
        \hline
        \end{tabular}
        \caption{Continuacion resultados de las mejores corridas de \emph{NDE} hibridado para {\bf Iris}}
        \label{tb:tabledehibcsvc}
    \end{center}
\end{table}


    \subsection{\emph{SDEH}}\label{sect:asde}

        Las variables del \emph{SDEH}(\ref{sect:ide}) son las siguientes:
        \begin{itemize}
            \item $I$: tamaño de la población. Se varió su valor en el rango
        $[10, 20, \cdots, 40]$.
            \item $w_1$: peso de la distancia intracluster. Se varió en el rango
        $[0.0; 0.1; \cdots; 1.0]$.
            \item $w_2$: peso de distancia intercluster. Se varió en el rango
        $[0.0; 0.1; \cdots; 1.0]$.
            \item $w_3$: peso del error. Se varió en el rango
        $[0.0; 0.1; \cdots; 1.0]$.
	        \item $\gamma$: valor de escalamiento de los vectores. Se varió en el rango
        $[0.5; 0.6; \cdots; 1.0]$.
	        \item $Cr$: probabilidad de cruce. Se varió en el rango
        $[0.1; 0.3; \cdots; 0.9]$.
        \end{itemize}

        \begin{table}[h!]
    \footnotesize
    \begin{center}
        \begin{tabular}{|c|c|c|c|c|c|c|c|c|c|c|c|}
        \hline
            & {\bf FO} & {\bf DB} & $J_e$ & {\bf E} & {\bf T} & $I$ & $w_1$ & $w_2$ & $w_3$ & $\gamma$ & $Cr$ \\
        \hline
        \hline
            Promedio  & 26.5314 & 2.8737 & 26.5314 & 275.0 & 0.7076 &  &  &  &  &  & \\
            \cline{1-6}
            Mejor & 30.3187 & 3.889  & 30.3187 & 275 & 0.6974 & 25 & 0.0 & 0.0 & 1.0 & 0.7 & 0.3\\
            \cline{1-6}
            Peor & 24.097 & 5.5354  & 24.097 & 275 & 0.7 &  &  &  &  &  & \\
        \hline
        \hline
            Promedio  & 30.6351 & 1.6464 & 27.3 & 165.0 & 0.4322 &  &  &  &  &  & \\
            \cline{1-6}
            Mejor & 33.5746 & 0.9474  & 30.862 & 165 & 0.4157 & 15 & 0.2 & 0.0 & 0.8 & 0.7 & 0.1\\
            \cline{1-6}
            Peor & 28.357 & 1.8338  & 24.5403 & 165 & 0.4217 &  &  &  &  &  & \\
        \hline
        \hline
            Promedio  & 31.1139 & 1.6541 & 26.1145 & 55.0 & 0.147 &  &  &  &  &  & \\
            \cline{1-6}
            Mejor & 34.3745 & 0.8953  & 30.3162 & 55 & 0.1551 & 5 & 0.4 & 0.0 & 0.6 & 0.7 & 0.1\\
            \cline{1-6}
            Peor & 28.4035 & 0.8448  & 27.4003 & 55 & 0.1444 &  &  &  &  &  & \\
        \hline
        \hline
            Promedio  & 31.9587 & 1.4372 & 27.3184 & 110.0 & 0.295 &  &  &  &  &  & \\
            \cline{1-6}
            Mejor & 35.871 & 0.9102  & 30.6337 & 110 & 0.29 & 10 & 0.4 & 0.0 & 0.6 & 0.5 & 0.1\\
            \cline{1-6}
            Peor & 29.4821 & 2.0784  & 21.4586 & 110 & 0.303 &  &  &  &  &  & \\
        \hline
        \hline
            Promedio  & 32.5814 & 1.5854 & 28.4268 & 330.0 & 0.8533 &  &  &  &  &  & \\
            \cline{1-6}
            Mejor & 37.1772 & 0.887  & 33.3669 & 330 & 0.83 & 30 & 0.3 & 0.0 & 0.7 & 0.9 & 0.1\\
            \cline{1-6}
            Peor & 30.022 & 0.8634  & 28.5639 & 330 & 0.8485 &  &  &  &  &  & \\
        \hline
        \hline
            Promedio  & 33.4593 & 1.7747 & 28.1594 & 330.0 & 0.8552 &  &  &  &  &  & \\
            \cline{1-6}
            Mejor & 38.6834 & 0.8813  & 33.9962 & 330 & 0.8518 & 30 & 0.4 & 0.0 & 0.6 & 0.9 & 0.9\\
            \cline{1-6}
            Peor & 31.4655 & 0.882  & 30.1177 & 330 & 0.8429 &  &  &  &  &  & \\
        \hline
        \hline
            Promedio  & 36.1714 & 1.8688 & 28.8049 & 385.0 & 0.9987 &  &  &  &  &  & \\
            \cline{1-6}
            Mejor & 39.74 & 1.0525  & 35.4739 & 385 & 0.9987 & 35 & 0.6 & 0.0 & 0.4 & 0.8 & 0.7\\
            \cline{1-6}
            Peor & 33.8242 & 0.9007  & 30.1466 & 385 & 1.0029 &  &  &  &  &  & \\
        \hline
        \hline
            Promedio  & 37.7046 & 1.4104 & 30.0509 & 440.0 & 1.1422 &  &  &  &  &  & \\
            \cline{1-6}
            Mejor & 40.7954 & 0.9046  & 31.4319 & 440 & 1.1301 & 40 & 0.7 & 0.0 & 0.3 & 0.6 & 0.3\\
            \cline{1-6}
            Peor & 34.759 & 0.9038  & 29.8346 & 440 & 1.1328 &  &  &  &  &  & \\
        \hline
        \hline
            Promedio  & 38.9513 & 1.5178 & 29.6 & 440.0 & 1.1465 &  &  &  &  &  & \\
            \cline{1-6}
            Mejor & 44.6176 & 0.9989  & 33.6233 & 440 & 1.1366 & 40 & 0.8 & 0.0 & 0.2 & 0.6 & 0.5\\
            \cline{1-6}
            Peor & 36.2197 & 1.9888  & 20.8988 & 440 & 1.1451 &  &  &  &  &  & \\
        \hline
        \hline
            Promedio  & 37.7991 & 1.3513 & 29.1032 & 275.0 & 0.7153 &  &  &  &  &  & \\
            \cline{1-6}
            Mejor & 46.9009 & 1.0263  & 32.9961 & 275 & 0.6864 & 25 & 0.8 & 0.0 & 0.2 & 0.9 & 0.1\\
            \cline{1-6}
            Peor & 33.8044 & 1.4348  & 22.3942 & 275 & 0.7319 &  &  &  &  &  & \\
        \hline
        \hline
            Promedio  & 71.2872 & 1.4501 & 29.6462 & 385.0 & 0.9962 &  &  &  &  &  & \\
            \cline{1-6}
            Mejor & 76.852 & 0.8803  & 35.3065 & 385 & 1.0024 & 35 & 0.5 & 0.1 & 0.4 & 0.7 & 0.3\\
            \cline{1-6}
            Peor & 69.0102 & 0.8446  & 27.3762 & 385 & 0.9911 &  &  &  &  &  & \\
        \hline
        \hline
            Promedio  & 102.9121 & 0.9596 & 32.0691 & 330.0 & 0.8496 &  &  &  &  &  & \\
            \cline{1-6}
            Mejor & 104.8186 & 1.7779  & 22.882 & 330 & 0.8541 & 30 & 0.1 & 0.2 & 0.7 & 0.8 & 0.7\\
            \cline{1-6}
            Peor & 101.4495 & 0.9654  & 30.3986 & 330 & 0.8538 &  &  &  &  &  & \\
        \hline
        \hline
            Promedio  & 105.1269 & 1.0437 & 30.9904 & 330.0 & 0.849 &  &  &  &  &  & \\
            \cline{1-6}
            Mejor & 107.4062 & 0.9258  & 34.7111 & 330 & 0.8413 & 30 & 0.3 & 0.2 & 0.5 & 0.8 & 0.5\\
            \cline{1-6}
            Peor & 103.6107 & 1.5007  & 21.3574 & 330 & 0.8403 &  &  &  &  &  & \\
        \hline
        \hline
            Promedio  & 175.7211 & 0.9584 & 34.1445 & 440.0 & 1.1375 &  &  &  &  &  & \\
            \cline{1-6}
            Mejor & 178.351 & 1.0087  & 43.9358 & 440 & 1.1201 & 40 & 0.4 & 0.4 & 0.2 & 0.6 & 0.1\\
            \cline{1-6}
            Peor & 174.5555 & 0.9776  & 40.6584 & 440 & 1.1281 &  &  &  &  &  & \\
        \hline
        \hline
            Promedio  & 206.8234 & 0.9642 & 34.005 & 385.0 & 0.9958 &  &  &  &  &  & \\
            \cline{1-6}
            Mejor & 208.1884 & 0.9324  & 32.4767 & 385 & 0.9785 & 35 & 0.1 & 0.5 & 0.4 & 0.8 & 0.1\\
            \cline{1-6}
            Peor & 205.9809 & 0.9182  & 31.3782 & 385 & 1.0069 &  &  &  &  &  & \\
        \hline
        \end{tabular}
        \caption{Resultados de las mejores corridas de \emph{SDE} no hibridado para {\bf Lenna}}
        \label{tb:tablesdealgimg}
    \end{center}
\end{table}


\begin{table}[h!]
    \footnotesize
    \begin{center}
        \begin{tabular}{|c|c|c|c|c|c|c|c|c|c|c|c|}
        \hline
            & {\bf FO} & {\bf DB} & $J_e$ & {\bf E} & {\bf T} & $I$ & $w_1$ & $w_2$ & $w_3$ & $\gamma$ & $Cr$ \\
        \hline
        \hline
            Promedio  & 207.9119 & 0.9375 & 32.855 & 275.0 & 0.7081 &  &  &  &  &  & \\
            \cline{1-6}
            Mejor & 209.3262 & 0.9735  & 32.8026 & 275 & 0.7008 & 25 & 0.2 & 0.5 & 0.3 & 0.6 & 0.7\\
            \cline{1-6}
            Peor & 206.7237 & 0.7915  & 26.5073 & 275 & 0.721 &  &  &  &  &  & \\
        \hline
        \hline
            Promedio  & 209.3613 & 0.9441 & 32.7395 & 110.0 & 0.2965 &  &  &  &  &  & \\
            \cline{1-6}
            Mejor & 212.0993 & 1.0052  & 39.0413 & 110 & 0.2947 & 10 & 0.4 & 0.5 & 0.1 & 0.5 & 0.7\\
            \cline{1-6}
            Peor & 207.0519 & 0.8631  & 26.7751 & 110 & 0.3096 &  &  &  &  &  & \\
        \hline
        \hline
            Promedio  & 240.9729 & 0.9532 & 34.0421 & 220.0 & 0.571 &  &  &  &  &  & \\
            \cline{1-6}
            Mejor & 241.981 & 0.9833  & 36.7868 & 220 & 0.5734 & 20 & 0.1 & 0.6 & 0.3 & 0.6 & 0.3\\
            \cline{1-6}
            Peor & 240.1862 & 0.905  & 29.5821 & 220 & 0.5662 &  &  &  &  &  & \\
        \hline
        \hline
            Promedio  & 241.1899 & 0.9504 & 33.4036 & 220.0 & 0.5759 &  &  &  &  &  & \\
            \cline{1-6}
            Mejor & 242.9883 & 0.8468  & 27.5457 & 220 & 0.6295 & 20 & 0.1 & 0.6 & 0.3 & 0.6 & 0.5\\
            \cline{1-6}
            Peor & 239.9857 & 1.0908  & 44.9694 & 220 & 0.5754 &  &  &  &  &  & \\
        \hline
        \hline
            Promedio  & 310.1062 & 0.9331 & 32.4915 & 275.0 & 0.7151 &  &  &  &  &  & \\
            \cline{1-6}
            Mejor & 312.5989 & 0.9153  & 29.6134 & 275 & 0.7108 & 25 & 0.1 & 0.8 & 0.1 & 0.8 & 0.5\\
            \cline{1-6}
            Peor & 308.7057 & 0.8806  & 30.7712 & 275 & 0.7254 &  &  &  &  &  & \\
        \hline
        \end{tabular}
        \caption{Continuacion resultados de las mejores corridas de \emph{SDE} no hibridado para {\bf Lenna}}
        \label{tb:tablesdealgimgc}
    \end{center}
\end{table}

        
        \begin{landscape}
        
        \begin{table}[h!]
    \footnotesize
    \begin{center}
        \begin{tabular}{|c|c|c|c|c|c|c|c|c|c|c|c|c|c|}
        \hline
            & {\bf FO} & {\bf DB} & $J_e$ & {\bf E} & {\bf T} & {\bf KE} & {\bf KO} & $I$ & $w_1$ & $w_2$ & $w_3$ & $\gamma$ & $Cr$ \\
        \hline
        \hline
            Promedio  & 1.1813 & 0.8491 & 19.5178 & 120.5667 & 0.0131 & 9.0 & $[5-10]$ &  &  &  &  &  & \\
            \cline{1-8}
            Mejor & 1.3195 & 0.7578  & 16.9559 & 115 & 0.0055 & 9 & $[5-10]$ & 10 & 0.4 & 0.5 & 0.1 & 0.5 & 0.7\\
            \cline{1-8}
            Peor & 0.9948 & 1.0052  & 39.0413 & 114 & 0.0042 & 9 & $[5-10]$ &  &  &  &  &  & \\
        \hline
        \hline
            Promedio  & 1.2098 & 0.8276 & 19.5192 & 67.2 & 0.0157 & 9.0 & $[5-10]$ &  &  &  &  &  & \\
            \cline{1-8}
            Mejor & 1.2943 & 0.7726  & 16.9606 & 66 & 0.0136 & 9 & $[5-10]$ & 5 & 0.4 & 0.0 & 0.6 & 0.7 & 0.1\\
            \cline{1-8}
            Peor & 1.145 & 0.8734  & 31.5717 & 59 & 0.0043 & 9 & $[5-10]$ &  &  &  &  &  & \\
        \hline
        \hline
            Promedio  & 1.1734 & 0.8549 & 20.1315 & 396.5333 & 0.0144 & 9.0 & $[5-10]$ &  &  &  &  &  & \\
            \cline{1-8}
            Mejor & 1.2912 & 0.7745  & 17.2302 & 394 & 0.011 & 9 & $[5-10]$ & 35 & 0.6 & 0.0 & 0.4 & 0.8 & 0.7\\
            \cline{1-8}
            Peor & 0.995 & 1.0051  & 31.9457 & 389 & 0.0041 & 9 & $[5-10]$ &  &  &  &  &  & \\
        \hline
        \hline
            Promedio  & 1.1778 & 0.8515 & 21.111 & 120.7333 & 0.0136 & 9.0 & $[5-10]$ &  &  &  &  &  & \\
            \cline{1-8}
            Mejor & 1.2907 & 0.7748  & 16.7368 & 126 & 0.0207 & 9 & $[5-10]$ & 10 & 0.4 & 0.0 & 0.6 & 0.5 & 0.1\\
            \cline{1-8}
            Peor & 1.0611 & 0.9425  & 18.1377 & 115 & 0.0055 & 9 & $[5-10]$ &  &  &  &  &  & \\
        \hline
        \hline
            Promedio  & 1.1759 & 0.8535 & 21.8932 & 452.4667 & 0.0156 & 9.0 & $[5-10]$ &  &  &  &  &  & \\
            \cline{1-8}
            Mejor & 1.2807 & 0.7808  & 16.8144 & 455 & 0.0195 & 9 & $[5-10]$ & 40 & 0.8 & 0.0 & 0.2 & 0.6 & 0.5\\
            \cline{1-8}
            Peor & 1.0447 & 0.9572  & 34.0983 & 444 & 0.0041 & 9 & $[5-10]$ &  &  &  &  &  & \\
        \hline
        \hline
            Promedio  & 1.1858 & 0.8452 & 19.0253 & 450.0333 & 0.0123 & 9.0 & $[5-10]$ &  &  &  &  &  & \\
            \cline{1-8}
            Mejor & 1.2806 & 0.7809  & 17.244 & 451 & 0.0138 & 9 & $[5-10]$ & 40 & 0.4 & 0.4 & 0.2 & 0.6 & 0.1\\
            \cline{1-8}
            Peor & 1.0444 & 0.9575  & 18.7914 & 445 & 0.0056 & 9 & $[5-10]$ &  &  &  &  &  & \\
        \hline
        \hline
            Promedio  & 1.1777 & 0.8511 & 18.9748 & 398.1667 & 0.0166 & 9.0 & $[5-10]$ &  &  &  &  &  & \\
            \cline{1-8}
            Mejor & 1.2797 & 0.7814  & 16.9623 & 399 & 0.0179 & 9 & $[5-10]$ & 35 & 0.5 & 0.1 & 0.4 & 0.7 & 0.3\\
            \cline{1-8}
            Peor & 1.056 & 0.9469  & 17.114 & 391 & 0.0068 & 9 & $[5-10]$ &  &  &  &  &  & \\
        \hline
        \hline
            Promedio  & 1.1774 & 0.851 & 20.1944 & 229.7 & 0.0118 & 9.0 & $[5-10]$ &  &  &  &  &  & \\
            \cline{1-8}
            Mejor & 1.2795 & 0.7815  & 16.9606 & 229 & 0.0107 & 9 & $[5-10]$ & 20 & 0.1 & 0.6 & 0.3 & 0.6 & 0.3\\
            \cline{1-8}
            Peor & 1.073 & 0.932  & 18.1933 & 225 & 0.0054 & 9 & $[5-10]$ &  &  &  &  &  & \\
        \hline
        \hline
            Promedio  & 1.1879 & 0.8436 & 20.302 & 178.2333 & 0.0166 & 9.0 & $[5-10]$ &  &  &  &  &  & \\
            \cline{1-8}
            Mejor & 1.2789 & 0.7819  & 17.0424 & 174 & 0.011 & 9 & $[5-10]$ & 15 & 0.2 & 0.0 & 0.8 & 0.7 & 0.1\\
            \cline{1-8}
            Peor & 1.0782 & 0.9274  & 30.4108 & 169 & 0.0041 & 9 & $[5-10]$ &  &  &  &  &  & \\
        \hline
        \hline
            Promedio  & 1.1597 & 0.8646 & 23.0706 & 228.5 & 0.0102 & 9.0 & $[5-10]$ &  &  &  &  &  & \\
            \cline{1-8}
            Mejor & 1.2753 & 0.7841  & 17.1144 & 225 & 0.0057 & 9 & $[5-10]$ & 20 & 0.1 & 0.6 & 0.3 & 0.6 & 0.5\\
            \cline{1-8}
            Peor & 1.0511 & 0.9513  & 34.4047 & 224 & 0.0041 & 9 & $[5-10]$ &  &  &  &  &  & \\
        \hline
        \end{tabular}
        \caption{Resultados de las mejores corridas de \emph{SDE} hibridado para {\bf Lenna}}
        \label{tb:tablesdehibimg}
    \end{center}
\end{table}


\begin{table}[h!]
    \footnotesize
    \begin{center}
        \begin{tabular}{|c|c|c|c|c|c|c|c|c|c|c|c|c|c|}
        \hline
            & {\bf FO} & {\bf DB} & $J_e$ & {\bf E} & {\bf T} & {\bf KE} & {\bf KO} & $I$ & $w_1$ & $w_2$ & $w_3$ & $\gamma$ & $Cr$ \\
        \hline
        \hline
            Promedio  & 1.1975 & 0.836 & 18.9567 & 287.6667 & 0.0158 & 9.0 & $[5-10]$ &  &  &  &  &  & \\
            \cline{1-8}
            Mejor & 1.2735 & 0.7852  & 16.9051 & 284 & 0.0112 & 9 & $[5-10]$ & 25 & 0.0 & 0.0 & 1.0 & 0.7 & 0.3\\
            \cline{1-8}
            Peor & 1.1145 & 0.8972  & 17.069 & 285 & 0.0123 & 9 & $[5-10]$ &  &  &  &  &  & \\
        \hline
        \hline
            Promedio  & 1.1912 & 0.842 & 19.949 & 286.4667 & 0.0141 & 9.0 & $[5-10]$ &  &  &  &  &  & \\
            \cline{1-8}
            Mejor & 1.2718 & 0.7863  & 18.3354 & 280 & 0.0054 & 9 & $[5-10]$ & 25 & 0.2 & 0.5 & 0.3 & 0.6 & 0.7\\
            \cline{1-8}
            Peor & 1.0475 & 0.9546  & 34.1146 & 279 & 0.0042 & 9 & $[5-10]$ &  &  &  &  &  & \\
        \hline
        \hline
            Promedio  & 1.1848 & 0.8459 & 20.8694 & 284.2 & 0.0112 & 9.0 & $[5-10]$ &  &  &  &  &  & \\
            \cline{1-8}
            Mejor & 1.2701 & 0.7874  & 16.8232 & 293 & 0.0236 & 9 & $[5-10]$ & 25 & 0.8 & 0.0 & 0.2 & 0.9 & 0.1\\
            \cline{1-8}
            Peor & 1.0874 & 0.9196  & 17.7045 & 293 & 0.0232 & 9 & $[5-10]$ &  &  &  &  &  & \\
        \hline
        \hline
            Promedio  & 1.1914 & 0.8411 & 20.8233 & 341.2667 & 0.014 & 9.0 & $[5-10]$ &  &  &  &  &  & \\
            \cline{1-8}
            Mejor & 1.2676 & 0.7889  & 16.8504 & 344 & 0.0182 & 9 & $[5-10]$ & 30 & 0.3 & 0.0 & 0.7 & 0.9 & 0.1\\
            \cline{1-8}
            Peor & 1.0849 & 0.9217  & 32.2337 & 334 & 0.0041 & 9 & $[5-10]$ &  &  &  &  &  & \\
        \hline
        \hline
            Promedio  & 1.1661 & 0.86 & 22.3663 & 341.4333 & 0.0145 & 9.0 & $[5-10]$ &  &  &  &  &  & \\
            \cline{1-8}
            Mejor & 1.2638 & 0.7912  & 16.8954 & 361 & 0.0416 & 9 & $[5-10]$ & 30 & 0.1 & 0.2 & 0.7 & 0.8 & 0.7\\
            \cline{1-8}
            Peor & 1.0422 & 0.9595  & 32.2244 & 334 & 0.004 & 9 & $[5-10]$ &  &  &  &  &  & \\
        \hline
        \hline
            Promedio  & 1.1887 & 0.8431 & 19.9226 & 341.7667 & 0.0147 & 9.0 & $[5-10]$ &  &  &  &  &  & \\
            \cline{1-8}
            Mejor & 1.2629 & 0.7918  & 16.9003 & 363 & 0.044 & 9 & $[5-10]$ & 30 & 0.3 & 0.2 & 0.5 & 0.8 & 0.5\\
            \cline{1-8}
            Peor & 1.0735 & 0.9315  & 32.1406 & 334 & 0.0041 & 9 & $[5-10]$ &  &  &  &  &  & \\
        \hline
        \hline
            Promedio  & 1.1933 & 0.8392 & 20.1672 & 343.2 & 0.0166 & 9.0 & $[5-10]$ &  &  &  &  &  & \\
            \cline{1-8}
            Mejor & 1.2612 & 0.7929  & 16.8972 & 341 & 0.014 & 9 & $[5-10]$ & 30 & 0.4 & 0.0 & 0.6 & 0.9 & 0.9\\
            \cline{1-8}
            Peor & 1.0975 & 0.9111  & 29.955 & 334 & 0.0041 & 9 & $[5-10]$ &  &  &  &  &  & \\
        \hline
        \hline
            Promedio  & 1.1713 & 0.8564 & 20.377 & 284.9667 & 0.0123 & 9.0 & $[5-10]$ &  &  &  &  &  & \\
            \cline{1-8}
            Mejor & 1.2585 & 0.7946  & 17.7476 & 280 & 0.0054 & 9 & $[5-10]$ & 25 & 0.1 & 0.8 & 0.1 & 0.8 & 0.5\\
            \cline{1-8}
            Peor & 1.0252 & 0.9754  & 18.7737 & 280 & 0.0054 & 9 & $[5-10]$ &  &  &  &  &  & \\
        \hline
        \hline
            Promedio  & 1.1635 & 0.8608 & 20.7577 & 448.1667 & 0.0098 & 9.0 & $[5-10]$ &  &  &  &  &  & \\
            \cline{1-8}
            Mejor & 1.2474 & 0.8016  & 17.0901 & 459 & 0.0252 & 9 & $[5-10]$ & 40 & 0.7 & 0.0 & 0.3 & 0.6 & 0.3\\
            \cline{1-8}
            Peor & 1.0802 & 0.9258  & 17.6532 & 448 & 0.0095 & 9 & $[5-10]$ &  &  &  &  &  & \\
        \hline
        \hline
            Promedio  & 1.1307 & 0.8872 & 22.7265 & 392.7333 & 0.0091 & 9.0 & $[5-10]$ &  &  &  &  &  & \\
            \cline{1-8}
            Mejor & 1.2338 & 0.8105  & 18.4953 & 390 & 0.0054 & 9 & $[5-10]$ & 35 & 0.1 & 0.5 & 0.4 & 0.8 & 0.1\\
            \cline{1-8}
            Peor & 1.0037 & 0.9963  & 32.3333 & 389 & 0.0042 & 9 & $[5-10]$ &  &  &  &  &  & \\
        \hline
        \end{tabular}
        \caption{Continuacion resultados de las mejores corridas de \emph{SDE} hibridado para {\bf Lenna}}
        \label{tb:tablesdehibimgc}
    \end{center}
\end{table}

        
        \end{landscape}

        \begin{table}[h!]
    \footnotesize
    \begin{center}
        \begin{tabular}{|c|c|c|c|c|c|c|c|c|c|c|c|}
        \hline
            & {\bf FO} & {\bf DB} & $J_e$ & {\bf E} & {\bf T} & $I$ & $w_1$ & $w_2$ & $w_3$ & $\gamma$ & $Cr$ \\
        \hline
        \hline
            Promedio  & 1.927 & 1.1284 & 1.4216 & 165.0 & 0.0028 &  &  &  &  &  & \\
            \cline{1-6}
            Mejor & 1.9561 & 1.1339  & 1.4304 & 165 & 0.003 & 15 & 0.5 & 0.1 & 0.4 & 0.8 & 0.9\\
            \cline{1-6}
            Peor & 1.8878 & 0.9216  & 1.5186 & 165 & 0.004 &  &  &  &  &  & \\
        \hline
        \hline
            Promedio  & 1.927 & 1.1284 & 1.4216 & 165.0 & 0.0028 &  &  &  &  &  & \\
            \cline{1-6}
            Mejor & 1.9561 & 1.1339  & 1.4304 & 165 & 0.0024 & 15 & 0.5 & 0.1 & 0.4 & 0.8 & 0.7\\
            \cline{1-6}
            Peor & 1.8878 & 0.9216  & 1.5186 & 165 & 0.0041 &  &  &  &  &  & \\
        \hline
        \hline
            Promedio  & 1.927 & 1.1284 & 1.4216 & 165.0 & 0.0027 &  &  &  &  &  & \\
            \cline{1-6}
            Mejor & 1.9561 & 1.1339  & 1.4304 & 165 & 0.0024 & 15 & 0.5 & 0.1 & 0.4 & 0.8 & 0.5\\
            \cline{1-6}
            Peor & 1.8878 & 0.9216  & 1.5186 & 165 & 0.004 &  &  &  &  &  & \\
        \hline
        \hline
            Promedio  & 1.927 & 1.1284 & 1.4216 & 165.0 & 0.0027 &  &  &  &  &  & \\
            \cline{1-6}
            Mejor & 1.9561 & 1.1339  & 1.4304 & 165 & 0.0023 & 15 & 0.5 & 0.1 & 0.4 & 0.8 & 0.3\\
            \cline{1-6}
            Peor & 1.8878 & 0.9216  & 1.5186 & 165 & 0.004 &  &  &  &  &  & \\
        \hline
        \hline
            Promedio  & 1.927 & 1.1284 & 1.4216 & 165.0 & 0.0026 &  &  &  &  &  & \\
            \cline{1-6}
            Mejor & 1.9561 & 1.1339  & 1.4304 & 165 & 0.0022 & 15 & 0.5 & 0.1 & 0.4 & 0.8 & 0.1\\
            \cline{1-6}
            Peor & 1.8878 & 0.9216  & 1.5186 & 165 & 0.0039 &  &  &  &  &  & \\
        \hline
        \hline
            Promedio  & 1.927 & 1.1284 & 1.4216 & 165.0 & 0.0026 &  &  &  &  &  & \\
            \cline{1-6}
            Mejor & 1.9561 & 1.1339  & 1.4304 & 165 & 0.0021 & 15 & 0.5 & 0.1 & 0.4 & 0.7 & 0.9\\
            \cline{1-6}
            Peor & 1.8878 & 0.9216  & 1.5186 & 165 & 0.004 &  &  &  &  &  & \\
        \hline
        \hline
            Promedio  & 1.927 & 1.1284 & 1.4216 & 165.0 & 0.0025 &  &  &  &  &  & \\
            \cline{1-6}
            Mejor & 1.9561 & 1.1339  & 1.4304 & 165 & 0.0022 & 15 & 0.5 & 0.1 & 0.4 & 0.7 & 0.7\\
            \cline{1-6}
            Peor & 1.8878 & 0.9216  & 1.5186 & 165 & 0.0041 &  &  &  &  &  & \\
        \hline
        \hline
            Promedio  & 1.927 & 1.1284 & 1.4216 & 165.0 & 0.0024 &  &  &  &  &  & \\
            \cline{1-6}
            Mejor & 1.9561 & 1.1339  & 1.4304 & 165 & 0.002 & 15 & 0.5 & 0.1 & 0.4 & 0.7 & 0.5\\
            \cline{1-6}
            Peor & 1.8878 & 0.9216  & 1.5186 & 165 & 0.004 &  &  &  &  &  & \\
        \hline
        \hline
            Promedio  & 1.927 & 1.1284 & 1.4216 & 165.0 & 0.0023 &  &  &  &  &  & \\
            \cline{1-6}
            Mejor & 1.9561 & 1.1339  & 1.4304 & 165 & 0.002 & 15 & 0.5 & 0.1 & 0.4 & 0.7 & 0.3\\
            \cline{1-6}
            Peor & 1.8878 & 0.9216  & 1.5186 & 165 & 0.0039 &  &  &  &  &  & \\
        \hline
        \hline
            Promedio  & 1.927 & 1.1284 & 1.4216 & 165.0 & 0.0022 &  &  &  &  &  & \\
            \cline{1-6}
            Mejor & 1.9561 & 1.1339  & 1.4304 & 165 & 0.0019 & 15 & 0.5 & 0.1 & 0.4 & 0.7 & 0.1\\
            \cline{1-6}
            Peor & 1.8878 & 0.9216  & 1.5186 & 165 & 0.0039 &  &  &  &  &  & \\
        \hline
        \hline
            Promedio  & 1.9287 & 1.1307 & 1.421 & 165.0 & 0.0021 &  &  &  &  &  & \\
            \cline{1-6}
            Mejor & 1.9561 & 1.1339  & 1.4304 & 165 & 0.0019 & 15 & 0.5 & 0.1 & 0.4 & 0.6 & 0.9\\
            \cline{1-6}
            Peor & 1.8878 & 0.9216  & 1.5186 & 165 & 0.004 &  &  &  &  &  & \\
        \hline
        \hline
            Promedio  & 1.9287 & 1.1307 & 1.421 & 165.0 & 0.002 &  &  &  &  &  & \\
            \cline{1-6}
            Mejor & 1.9561 & 1.1339  & 1.4304 & 165 & 0.0016 & 15 & 0.5 & 0.1 & 0.4 & 0.6 & 0.7\\
            \cline{1-6}
            Peor & 1.8878 & 0.9216  & 1.5186 & 165 & 0.004 &  &  &  &  &  & \\
        \hline
        \hline
            Promedio  & 1.9293 & 1.143 & 1.4158 & 165.0 & 0.002 &  &  &  &  &  & \\
            \cline{1-6}
            Mejor & 1.9561 & 1.1339  & 1.4304 & 165 & 0.0016 & 15 & 0.5 & 0.1 & 0.4 & 0.6 & 0.5\\
            \cline{1-6}
            Peor & 1.8878 & 0.9216  & 1.5186 & 165 & 0.0017 &  &  &  &  &  & \\
        \hline
        \hline
            Promedio  & 1.9293 & 1.143 & 1.4158 & 165.0 & 0.0019 &  &  &  &  &  & \\
            \cline{1-6}
            Mejor & 1.9561 & 1.1339  & 1.4304 & 165 & 0.0015 & 15 & 0.5 & 0.1 & 0.4 & 0.6 & 0.3\\
            \cline{1-6}
            Peor & 1.8878 & 0.9216  & 1.5186 & 165 & 0.0017 &  &  &  &  &  & \\
        \hline
        \hline
            Promedio  & 1.9293 & 1.143 & 1.4158 & 165.0 & 0.0019 &  &  &  &  &  & \\
            \cline{1-6}
            Mejor & 1.9561 & 1.1339  & 1.4304 & 165 & 0.0015 & 15 & 0.5 & 0.1 & 0.4 & 0.6 & 0.1\\
            \cline{1-6}
            Peor & 1.8878 & 0.9216  & 1.5186 & 165 & 0.0017 &  &  &  &  &  & \\
        \hline
        \end{tabular}
        \caption{Resultados de las mejores corridas de \emph{SDE} no hibridado para {\bf Iris}}
        \label{tb:tablesdealgcsv}
    \end{center}
\end{table}


\begin{table}[h!]
    \footnotesize
    \begin{center}
        \begin{tabular}{|c|c|c|c|c|c|c|c|c|c|c|c|}
        \hline
            & {\bf FO} & {\bf DB} & $J_e$ & {\bf E} & {\bf T} & $I$ & $w_1$ & $w_2$ & $w_3$ & $\gamma$ & $Cr$ \\
        \hline
        \hline
            Promedio  & 1.929 & 1.1367 & 1.4164 & 165.0 & 0.0019 &  &  &  &  &  & \\
            \cline{1-6}
            Mejor & 1.9561 & 1.1339  & 1.4304 & 165 & 0.0016 & 15 & 0.5 & 0.1 & 0.4 & 0.5 & 0.9\\
            \cline{1-6}
            Peor & 1.8878 & 0.9216  & 1.5186 & 165 & 0.0016 &  &  &  &  &  & \\
        \hline
        \hline
            Promedio  & 1.929 & 1.1367 & 1.4164 & 165.0 & 0.0018 &  &  &  &  &  & \\
            \cline{1-6}
            Mejor & 1.9561 & 1.1339  & 1.4304 & 165 & 0.0015 & 15 & 0.5 & 0.1 & 0.4 & 0.5 & 0.7\\
            \cline{1-6}
            Peor & 1.8878 & 0.9216  & 1.5186 & 165 & 0.0015 &  &  &  &  &  & \\
        \hline
        \hline
            Promedio  & 1.927 & 1.136 & 1.4188 & 165.0 & 0.0017 &  &  &  &  &  & \\
            \cline{1-6}
            Mejor & 1.9561 & 1.1339  & 1.4304 & 165 & 0.0016 & 15 & 0.5 & 0.1 & 0.4 & 0.5 & 0.5\\
            \cline{1-6}
            Peor & 1.8878 & 0.9216  & 1.5186 & 165 & 0.0014 &  &  &  &  &  & \\
        \hline
        \hline
            Promedio  & 1.927 & 1.136 & 1.4188 & 165.0 & 0.0016 &  &  &  &  &  & \\
            \cline{1-6}
            Mejor & 1.9561 & 1.1339  & 1.4304 & 165 & 0.0014 & 15 & 0.5 & 0.1 & 0.4 & 0.5 & 0.3\\
            \cline{1-6}
            Peor & 1.8878 & 0.9216  & 1.5186 & 165 & 0.0014 &  &  &  &  &  & \\
        \hline
        \hline
            Promedio  & 1.927 & 1.136 & 1.4188 & 165.0 & 0.0015 &  &  &  &  &  & \\
            \cline{1-6}
            Mejor & 1.9561 & 1.1339  & 1.4304 & 165 & 0.0015 & 15 & 0.5 & 0.1 & 0.4 & 0.5 & 0.1\\
            \cline{1-6}
            Peor & 1.8878 & 0.9216  & 1.5186 & 165 & 0.0014 &  &  &  &  &  & \\
        \hline
        \end{tabular}
        \caption{Continuacion resultados de las mejores corridas de \emph{SDE} no hibridado para {\bf Iris}}
        \label{tb:tablesdealgcsvc}
    \end{center}
\end{table}

        
        \begin{table}[h!]
    \footnotesize
    \begin{center}
        \begin{tabular}{|c|c|c|c|c|c|c|c|c|c|c|c|c|c|}
        \hline
            & {\bf FO} & {\bf DB} & $J_e$ & {\bf E} & {\bf T} & {\bf KE} & {\bf KO} & $I$ & $w_1$ & $w_2$ & $w_3$ & $\gamma$ & $Cr$ \\
        \hline
        \hline
            Promedio  & 1.5312 & 0.6533 & 0.6448 & 172.3667 & 0.0001 & 3.0 & 3 &  &  &  &  &  & \\
            \cline{1-8}
            Mejor & 1.5778 & 0.6338  & 0.625 & 171 & 0.0001 & 3 & 3 & 15 & 0.5 & 0.1 & 0.4 & 0.8 & 0.9\\
            \cline{1-8}
            Peor & 1.5006 & 0.6664  & 0.649 & 176 & 0.0001 & 3 & 3 &  &  &  &  &  & \\
        \hline
        \hline
            Promedio  & 1.5312 & 0.6533 & 0.6448 & 172.3667 & 0.0001 & 3.0 & 3 &  &  &  &  &  & \\
            \cline{1-8}
            Mejor & 1.5778 & 0.6338  & 0.625 & 171 & 0.0001 & 3 & 3 & 15 & 0.5 & 0.1 & 0.4 & 0.8 & 0.7\\
            \cline{1-8}
            Peor & 1.5006 & 0.6664  & 0.649 & 176 & 0.0002 & 3 & 3 &  &  &  &  &  & \\
        \hline
        \hline
            Promedio  & 1.5312 & 0.6533 & 0.6448 & 172.3667 & 0.0001 & 3.0 & 3 &  &  &  &  &  & \\
            \cline{1-8}
            Mejor & 1.5778 & 0.6338  & 0.625 & 171 & 0.0001 & 3 & 3 & 15 & 0.5 & 0.1 & 0.4 & 0.8 & 0.5\\
            \cline{1-8}
            Peor & 1.5006 & 0.6664  & 0.649 & 176 & 0.0001 & 3 & 3 &  &  &  &  &  & \\
        \hline
        \hline
            Promedio  & 1.5312 & 0.6533 & 0.6448 & 172.3667 & 0.0001 & 3.0 & 3 &  &  &  &  &  & \\
            \cline{1-8}
            Mejor & 1.5778 & 0.6338  & 0.625 & 171 & 0.0001 & 3 & 3 & 15 & 0.5 & 0.1 & 0.4 & 0.8 & 0.3\\
            \cline{1-8}
            Peor & 1.5006 & 0.6664  & 0.649 & 176 & 0.0001 & 3 & 3 &  &  &  &  &  & \\
        \hline
        \hline
            Promedio  & 1.5312 & 0.6533 & 0.6448 & 172.3667 & 0.0001 & 3.0 & 3 &  &  &  &  &  & \\
            \cline{1-8}
            Mejor & 1.5778 & 0.6338  & 0.625 & 171 & 0.0001 & 3 & 3 & 15 & 0.5 & 0.1 & 0.4 & 0.8 & 0.1\\
            \cline{1-8}
            Peor & 1.5006 & 0.6664  & 0.649 & 176 & 0.0002 & 3 & 3 &  &  &  &  &  & \\
        \hline
        \hline
            Promedio  & 1.5312 & 0.6533 & 0.6448 & 172.3667 & 0.0001 & 3.0 & 3 &  &  &  &  &  & \\
            \cline{1-8}
            Mejor & 1.5778 & 0.6338  & 0.625 & 171 & 0.0001 & 3 & 3 & 15 & 0.5 & 0.1 & 0.4 & 0.7 & 0.9\\
            \cline{1-8}
            Peor & 1.5006 & 0.6664  & 0.649 & 176 & 0.0001 & 3 & 3 &  &  &  &  &  & \\
        \hline
        \hline
            Promedio  & 1.5312 & 0.6533 & 0.6448 & 172.3667 & 0.0001 & 3.0 & 3 &  &  &  &  &  & \\
            \cline{1-8}
            Mejor & 1.5778 & 0.6338  & 0.625 & 171 & 0.0001 & 3 & 3 & 15 & 0.5 & 0.1 & 0.4 & 0.7 & 0.7\\
            \cline{1-8}
            Peor & 1.5006 & 0.6664  & 0.649 & 176 & 0.0001 & 3 & 3 &  &  &  &  &  & \\
        \hline
        \hline
            Promedio  & 1.5312 & 0.6533 & 0.6448 & 172.3667 & 0.0001 & 3.0 & 3 &  &  &  &  &  & \\
            \cline{1-8}
            Mejor & 1.5778 & 0.6338  & 0.625 & 171 & 0.0001 & 3 & 3 & 15 & 0.5 & 0.1 & 0.4 & 0.7 & 0.5\\
            \cline{1-8}
            Peor & 1.5006 & 0.6664  & 0.649 & 176 & 0.0002 & 3 & 3 &  &  &  &  &  & \\
        \hline
        \hline
            Promedio  & 1.5312 & 0.6533 & 0.6448 & 172.3667 & 0.0001 & 3.0 & 3 &  &  &  &  &  & \\
            \cline{1-8}
            Mejor & 1.5778 & 0.6338  & 0.625 & 171 & 0.0001 & 3 & 3 & 15 & 0.5 & 0.1 & 0.4 & 0.7 & 0.3\\
            \cline{1-8}
            Peor & 1.5006 & 0.6664  & 0.649 & 176 & 0.0001 & 3 & 3 &  &  &  &  &  & \\
        \hline
        \hline
            Promedio  & 1.5312 & 0.6533 & 0.6448 & 172.3667 & 0.0001 & 3.0 & 3 &  &  &  &  &  & \\
            \cline{1-8}
            Mejor & 1.5778 & 0.6338  & 0.625 & 171 & 0.0001 & 3 & 3 & 15 & 0.5 & 0.1 & 0.4 & 0.7 & 0.1\\
            \cline{1-8}
            Peor & 1.5006 & 0.6664  & 0.649 & 176 & 0.0001 & 3 & 3 &  &  &  &  &  & \\
        \hline
        \hline
            Promedio  & 1.5337 & 0.6522 & 0.644 & 172.2 & 0.0001 & 3.0 & 3 &  &  &  &  &  & \\
            \cline{1-8}
            Mejor & 1.5778 & 0.6338  & 0.625 & 171 & 0.0001 & 3 & 3 & 15 & 0.5 & 0.1 & 0.4 & 0.6 & 0.9\\
            \cline{1-8}
            Peor & 1.5006 & 0.6664  & 0.649 & 176 & 0.0001 & 3 & 3 &  &  &  &  &  & \\
        \hline
        \hline
            Promedio  & 1.5337 & 0.6522 & 0.644 & 172.2 & 0.0001 & 3.0 & 3 &  &  &  &  &  & \\
            \cline{1-8}
            Mejor & 1.5778 & 0.6338  & 0.625 & 171 & 0.0001 & 3 & 3 & 15 & 0.5 & 0.1 & 0.4 & 0.6 & 0.7\\
            \cline{1-8}
            Peor & 1.5006 & 0.6664  & 0.649 & 176 & 0.0001 & 3 & 3 &  &  &  &  &  & \\
        \hline
        \hline
            Promedio  & 1.533 & 0.6525 & 0.6439 & 172.3667 & 0.0001 & 3.0 & 3 &  &  &  &  &  & \\
            \cline{1-8}
            Mejor & 1.5778 & 0.6338  & 0.625 & 171 & 0.0001 & 3 & 3 & 15 & 0.5 & 0.1 & 0.4 & 0.6 & 0.5\\
            \cline{1-8}
            Peor & 1.5006 & 0.6664  & 0.649 & 176 & 0.0001 & 3 & 3 &  &  &  &  &  & \\
        \hline
        \hline
            Promedio  & 1.533 & 0.6525 & 0.6439 & 172.3667 & 0.0001 & 3.0 & 3 &  &  &  &  &  & \\
            \cline{1-8}
            Mejor & 1.5778 & 0.6338  & 0.625 & 171 & 0.0001 & 3 & 3 & 15 & 0.5 & 0.1 & 0.4 & 0.6 & 0.3\\
            \cline{1-8}
            Peor & 1.5006 & 0.6664  & 0.649 & 176 & 0.0001 & 3 & 3 &  &  &  &  &  & \\
        \hline
        \hline
            Promedio  & 1.533 & 0.6525 & 0.6439 & 172.3667 & 0.0001 & 3.0 & 3 &  &  &  &  &  & \\
            \cline{1-8}
            Mejor & 1.5778 & 0.6338  & 0.625 & 171 & 0.0001 & 3 & 3 & 15 & 0.5 & 0.1 & 0.4 & 0.6 & 0.1\\
            \cline{1-8}
            Peor & 1.5006 & 0.6664  & 0.649 & 176 & 0.0001 & 3 & 3 &  &  &  &  &  & \\
        \hline
        \end{tabular}
        \caption{Resultados de las mejores corridas de \emph{SDE} hibridado para {\bf Iris}}
        \label{tb:tablesdehibcsv}
    \end{center}
\end{table}


\begin{table}[h!]
    \footnotesize
    \begin{center}
        \begin{tabular}{|c|c|c|c|c|c|c|c|c|c|c|c|c|c|}
        \hline
            & {\bf FO} & {\bf DB} & $J_e$ & {\bf E} & {\bf T} & {\bf KE} & {\bf KO} & $I$ & $w_1$ & $w_2$ & $w_3$ & $\gamma$ & $Cr$ \\
        \hline
        \hline
            Promedio  & 1.5329 & 0.6526 & 0.6438 & 172.4 & 0.0001 & 3.0 & 3 &  &  &  &  &  & \\
            \cline{1-8}
            Mejor & 1.5778 & 0.6338  & 0.625 & 171 & 0.0001 & 3 & 3 & 15 & 0.5 & 0.1 & 0.4 & 0.5 & 0.9\\
            \cline{1-8}
            Peor & 1.5006 & 0.6664  & 0.649 & 176 & 0.0001 & 3 & 3 &  &  &  &  &  & \\
        \hline
        \hline
            Promedio  & 1.5329 & 0.6526 & 0.6438 & 172.4 & 0.0001 & 3.0 & 3 &  &  &  &  &  & \\
            \cline{1-8}
            Mejor & 1.5778 & 0.6338  & 0.625 & 171 & 0.0001 & 3 & 3 & 15 & 0.5 & 0.1 & 0.4 & 0.5 & 0.7\\
            \cline{1-8}
            Peor & 1.5006 & 0.6664  & 0.649 & 176 & 0.0001 & 3 & 3 &  &  &  &  &  & \\
        \hline
        \hline
            Promedio  & 1.5327 & 0.6527 & 0.644 & 172.3667 & 0.0001 & 3.0 & 3 &  &  &  &  &  & \\
            \cline{1-8}
            Mejor & 1.5778 & 0.6338  & 0.625 & 171 & 0.0001 & 3 & 3 & 15 & 0.5 & 0.1 & 0.4 & 0.5 & 0.5\\
            \cline{1-8}
            Peor & 1.5006 & 0.6664  & 0.649 & 176 & 0.0001 & 3 & 3 &  &  &  &  &  & \\
        \hline
        \hline
            Promedio  & 1.5327 & 0.6527 & 0.644 & 172.3667 & 0.0001 & 3.0 & 3 &  &  &  &  &  & \\
            \cline{1-8}
            Mejor & 1.5778 & 0.6338  & 0.625 & 171 & 0.0001 & 3 & 3 & 15 & 0.5 & 0.1 & 0.4 & 0.5 & 0.3\\
            \cline{1-8}
            Peor & 1.5006 & 0.6664  & 0.649 & 176 & 0.0001 & 3 & 3 &  &  &  &  &  & \\
        \hline
        \hline
            Promedio  & 1.5327 & 0.6527 & 0.644 & 172.3667 & 0.0 & 3.0 & 3 &  &  &  &  &  & \\
            \cline{1-8}
            Mejor & 1.5778 & 0.6338  & 0.625 & 171 & 0.0001 & 3 & 3 & 15 & 0.5 & 0.1 & 0.4 & 0.5 & 0.1\\
            \cline{1-8}
            Peor & 1.5006 & 0.6664  & 0.649 & 176 & 0.0001 & 3 & 3 &  &  &  &  &  & \\
        \hline
        \end{tabular}
        \caption{Continuacion resultados de las mejores corridas de \emph{SDE} hibridado para {\bf Iris}}
        \label{tb:tablesdehibcsvc}
    \end{center}
\end{table}


\section{Algoritmo de Abeja (BeeH)}\label{sect:abee}

    Las variables del \emph{BeeH} (\ref{sect:ibee}) son las siguientes:
    \begin{itemize}
        \item $I$: cantidad de abejas. Se varió su valor en el rango
    $[5, 10, \cdots, 40]$.
        \item $m$: cantidad de lugares de flores. Se varió su valor en el rango
    $[5, 6, \cdots, 15]$.
        \item $e$: cantidad de sitios élite de flores. Se varió su valor en el rango
    $[1, 2, \cdots, 15]$.
        \item $eb$: cantidad de abejas élite. Se varió su valor en el rango
    $[5, 6, \cdots, 15]$.
	    \item $ob$: cantidad de abejas exploradoras. Se varió su valor en el rango
    $[1, 2, \cdots, 15]$.
    \end{itemize}

    \begin{table}[h!]
    \footnotesize
    \begin{center}
        \begin{tabular}{|c|c|c|c|c|c|c|c|c|c|c|}
        \hline
            & {\bf FO} & {\bf DB} & $J_e$ & {\bf E} & {\bf T} & $I$ & $m$ & $e$ & $eb$ & $ob$ \\
        \hline
        \hline
            Promedio  & 1.3052 & 0.7664 & 16.924 & 338.3333 & 0.4738 &  &  &  &  & \\
            \cline{1-6}
            Mejor & 1.3624 & 0.734  & 16.9922 & 280 & 0.3785 & 35 & 15 & 2 & 14 & 6\\
            \cline{1-6}
            Peor & 1.2586 & 0.7945  & 16.7638 & 175 & 0.2384 &  &  &  &  & \\
        \hline
        \hline
            Promedio  & 1.2994 & 0.77 & 17.0356 & 429.3333 & 0.596 &  &  &  &  & \\
            \cline{1-6}
            Mejor & 1.3595 & 0.7355  & 16.8857 & 360 & 0.4887 & 40 & 14 & 1 & 15 & 7\\
            \cline{1-6}
            Peor & 1.2142 & 0.8236  & 17.0579 & 200 & 0.279 &  &  &  &  & \\
        \hline
        \hline
            Promedio  & 1.3065 & 0.7656 & 16.8663 & 394.3333 & 0.5524 &  &  &  &  & \\
            \cline{1-6}
            Mejor & 1.3572 & 0.7368  & 17.0624 & 315 & 0.4454 & 35 & 15 & 2 & 14 & 7\\
            \cline{1-6}
            Peor & 1.2519 & 0.7988  & 16.6099 & 210 & 0.3091 &  &  &  &  & \\
        \hline
        \hline
            Promedio  & 1.3018 & 0.7684 & 16.8983 & 374.5 & 0.5227 &  &  &  &  & \\
            \cline{1-6}
            Mejor & 1.3565 & 0.7372  & 17.3008 & 210 & 0.2867 & 35 & 14 & 5 & 15 & 2\\
            \cline{1-6}
            Peor & 1.26 & 0.7937  & 16.7771 & 245 & 0.3614 &  &  &  &  & \\
        \hline
        \hline
            Promedio  & 1.2864 & 0.7779 & 16.9763 & 200.6667 & 0.284 &  &  &  &  & \\
            \cline{1-6}
            Mejor & 1.3553 & 0.7379  & 16.6018 & 220 & 0.3026 & 20 & 11 & 1 & 11 & 4\\
            \cline{1-6}
            Peor & 1.2214 & 0.8187  & 16.8586 & 100 & 0.1374 &  &  &  &  & \\
        \hline
        \hline
            Promedio  & 1.2982 & 0.7705 & 16.9716 & 364.0 & 0.5033 &  &  &  &  & \\
            \cline{1-6}
            Mejor & 1.3535 & 0.7388  & 16.9898 & 280 & 0.3917 & 40 & 15 & 2 & 10 & 7\\
            \cline{1-6}
            Peor & 1.248 & 0.8013  & 17.4557 & 240 & 0.3337 &  &  &  &  & \\
        \hline
        \hline
            Promedio  & 1.297 & 0.7715 & 16.9614 & 416.0 & 0.5858 &  &  &  &  & \\
            \cline{1-6}
            Mejor & 1.3533 & 0.7389  & 17.2768 & 240 & 0.3297 & 40 & 12 & 1 & 12 & 15\\
            \cline{1-6}
            Peor & 1.2125 & 0.8247  & 16.8693 & 280 & 0.4161 &  &  &  &  & \\
        \hline
        \hline
            Promedio  & 1.3017 & 0.7686 & 16.9351 & 348.0 & 0.4851 &  &  &  &  & \\
            \cline{1-6}
            Mejor & 1.3512 & 0.7401  & 16.8872 & 280 & 0.3945 & 40 & 10 & 1 & 15 & 7\\
            \cline{1-6}
            Peor & 1.2357 & 0.8093  & 17.0133 & 240 & 0.3313 &  &  &  &  & \\
        \hline
        \hline
            Promedio  & 1.307 & 0.7654 & 16.9329 & 428.0 & 0.5942 &  &  &  &  & \\
            \cline{1-6}
            Mejor & 1.3496 & 0.741  & 16.5407 & 200 & 0.2856 & 40 & 13 & 1 & 12 & 5\\
            \cline{1-6}
            Peor & 1.2416 & 0.8054  & 17.6686 & 360 & 0.494 &  &  &  &  & \\
        \hline
        \hline
            Promedio  & 1.3053 & 0.7664 & 16.9986 & 374.6667 & 0.5212 &  &  &  &  & \\
            \cline{1-6}
            Mejor & 1.3494 & 0.741  & 16.931 & 360 & 0.5052 & 40 & 13 & 1 & 15 & 7\\
            \cline{1-6}
            Peor & 1.2482 & 0.8011  & 17.1707 & 400 & 0.5461 &  &  &  &  & \\
        \hline
        \hline
            Promedio  & 1.3046 & 0.7667 & 16.9127 & 291.0 & 0.4072 &  &  &  &  & \\
            \cline{1-6}
            Mejor & 1.3469 & 0.7424  & 17.0975 & 240 & 0.3246 & 30 & 15 & 4 & 10 & 1\\
            \cline{1-6}
            Peor & 1.259 & 0.7943  & 17.0465 & 270 & 0.3842 &  &  &  &  & \\
        \hline
        \hline
            Promedio  & 1.3 & 0.7694 & 16.9167 & 350.0 & 0.4873 &  &  &  &  & \\
            \cline{1-6}
            Mejor & 1.3444 & 0.7438  & 16.6935 & 455 & 0.6452 & 35 & 11 & 4 & 12 & 4\\
            \cline{1-6}
            Peor & 1.2584 & 0.7946  & 17.6382 & 210 & 0.3007 &  &  &  &  & \\
        \hline
        \hline
            Promedio  & 1.3073 & 0.7651 & 16.9697 & 333.6667 & 0.463 &  &  &  &  & \\
            \cline{1-6}
            Mejor & 1.3423 & 0.745  & 16.7189 & 315 & 0.4242 & 35 & 13 & 1 & 11 & 2\\
            \cline{1-6}
            Peor & 1.266 & 0.7899  & 16.9588 & 175 & 0.2571 &  &  &  &  & \\
        \hline
        \hline
            Promedio  & 1.2937 & 0.7733 & 16.9564 & 305.6667 & 0.4291 &  &  &  &  & \\
            \cline{1-6}
            Mejor & 1.3417 & 0.7453  & 16.5588 & 420 & 0.5914 & 35 & 9 & 1 & 7 & 13\\
            \cline{1-6}
            Peor & 1.2458 & 0.8027  & 16.8434 & 280 & 0.3798 &  &  &  &  & \\
        \hline
        \hline
            Promedio  & 1.302 & 0.7683 & 16.9919 & 381.5 & 0.5318 &  &  &  &  & \\
            \cline{1-6}
            Mejor & 1.3396 & 0.7465  & 16.6402 & 525 & 0.7477 & 35 & 15 & 4 & 14 & 5\\
            \cline{1-6}
            Peor & 1.2292 & 0.8136  & 17.34 & 175 & 0.245 &  &  &  &  & \\
        \hline
        \end{tabular}
        \caption{Resultados de las mejores corridas de \emph{BEE} no hibridado para {\bf Lenna}}
        \label{tb:tablebeealgimg}
    \end{center}
\end{table}


\begin{table}[h!]
    \footnotesize
    \begin{center}
        \begin{tabular}{|c|c|c|c|c|c|c|c|c|c|c|}
        \hline
            & {\bf FO} & {\bf DB} & $J_e$ & {\bf E} & {\bf T} & $I$ & $m$ & $e$ & $eb$ & $ob$ \\
        \hline
        \hline
            Promedio  & 1.293 & 0.7737 & 16.9901 & 317.3333 & 0.4447 &  &  &  &  & \\
            \cline{1-6}
            Mejor & 1.3364 & 0.7483  & 16.7379 & 350 & 0.4785 & 35 & 9 & 1 & 15 & 1\\
            \cline{1-6}
            Peor & 1.2243 & 0.8168  & 16.2465 & 245 & 0.3265 &  &  &  &  & \\
        \hline
        \hline
            Promedio  & 1.3016 & 0.7687 & 16.9795 & 373.3333 & 0.5183 &  &  &  &  & \\
            \cline{1-6}
            Mejor & 1.3333 & 0.75  & 16.6586 & 280 & 0.394 & 40 & 15 & 1 & 12 & 12\\
            \cline{1-6}
            Peor & 1.1774 & 0.8493  & 16.521 & 240 & 0.3295 &  &  &  &  & \\
        \hline
        \hline
            Promedio  & 1.2815 & 0.7808 & 16.9242 & 264.1667 & 0.3675 &  &  &  &  & \\
            \cline{1-6}
            Mejor & 1.3317 & 0.7509  & 16.6688 & 375 & 0.5221 & 25 & 11 & 3 & 8 & 9\\
            \cline{1-6}
            Peor & 1.198 & 0.8347  & 17.3952 & 150 & 0.2043 &  &  &  &  & \\
        \hline
        \hline
            Promedio  & 1.2995 & 0.7698 & 16.8846 & 289.0 & 0.4031 &  &  &  &  & \\
            \cline{1-6}
            Mejor & 1.3313 & 0.7511  & 16.9553 & 420 & 0.5828 & 30 & 12 & 2 & 6 & 8\\
            \cline{1-6}
            Peor & 1.233 & 0.811  & 17.3257 & 300 & 0.4063 &  &  &  &  & \\
        \hline
        \hline
            Promedio  & 1.3016 & 0.7684 & 16.934 & 422.6667 & 0.5866 &  &  &  &  & \\
            \cline{1-6}
            Mejor & 1.3264 & 0.7539  & 16.6326 & 480 & 0.6633 & 40 & 7 & 1 & 8 & 4\\
            \cline{1-6}
            Peor & 1.2654 & 0.7902  & 16.9853 & 280 & 0.3931 &  &  &  &  & \\
        \hline
        \end{tabular}
        \caption{Continuacion resultados de las mejores corridas de \emph{BEE} no hibridado para {\bf Lenna}}
        \label{tb:tablebeealgimg}
    \end{center}
\end{table}

    
    \begin{table}[h!]
    \footnotesize
    \begin{center}
        \begin{tabular}{|c|c|c|c|c|c|c|c|c|c|c|c|c|}
        \hline
            & {\bf FO} & {\bf DB} & $J_e$ & {\bf E} & {\bf T} & {\bf KE} & {\bf KO} & $I$ & $m$ & $e$ & $eb$ & $ob$ \\
        \hline
        \hline
            Promedio  & 1.3052 & 0.7664 & 16.924 & 342.3333 & 0.0042 & 9.0 & $[5-10]$ &  &  &  &  & \\
            \cline{1-8}
            Mejor & 1.3624 & 0.734  & 16.9922 & 284 & 0.0041 & 9 & $[5-10]$ & 35 & 15 & 2 & 14 & 6\\
            \cline{1-8}
            Peor & 1.2586 & 0.7945  & 16.7638 & 179 & 0.0041 & 9 & $[5-10]$ &  &  &  &  & \\
        \hline
        \hline
            Promedio  & 1.3004 & 0.7694 & 17.0312 & 433.3667 & 0.0043 & 9.0 & $[5-10]$ &  &  &  &  & \\
            \cline{1-8}
            Mejor & 1.3595 & 0.7355  & 16.8857 & 364 & 0.0041 & 9 & $[5-10]$ & 40 & 14 & 1 & 15 & 7\\
            \cline{1-8}
            Peor & 1.2142 & 0.8236  & 17.0579 & 204 & 0.0043 & 9 & $[5-10]$ &  &  &  &  & \\
        \hline
        \hline
            Promedio  & 1.3079 & 0.7648 & 16.8691 & 398.4333 & 0.0044 & 9.0 & $[5-10]$ &  &  &  &  & \\
            \cline{1-8}
            Mejor & 1.3572 & 0.7368  & 17.0624 & 319 & 0.0042 & 9 & $[5-10]$ & 35 & 15 & 2 & 14 & 7\\
            \cline{1-8}
            Peor & 1.2672 & 0.7891  & 17.1792 & 319 & 0.0041 & 9 & $[5-10]$ &  &  &  &  & \\
        \hline
        \hline
            Promedio  & 1.302 & 0.7683 & 16.8974 & 378.5333 & 0.0043 & 9.0 & $[5-10]$ &  &  &  &  & \\
            \cline{1-8}
            Mejor & 1.3565 & 0.7372  & 17.3008 & 214 & 0.0041 & 9 & $[5-10]$ & 35 & 14 & 5 & 15 & 2\\
            \cline{1-8}
            Peor & 1.26 & 0.7937  & 16.7771 & 249 & 0.004 & 9 & $[5-10]$ &  &  &  &  & \\
        \hline
        \hline
            Promedio  & 1.289 & 0.7763 & 16.9652 & 204.8 & 0.0045 & 9.0 & $[5-10]$ &  &  &  &  & \\
            \cline{1-8}
            Mejor & 1.3553 & 0.7379  & 16.6018 & 224 & 0.0041 & 9 & $[5-10]$ & 20 & 11 & 1 & 11 & 4\\
            \cline{1-8}
            Peor & 1.2315 & 0.812  & 17.6028 & 144 & 0.0043 & 9 & $[5-10]$ &  &  &  &  & \\
        \hline
        \hline
            Promedio  & 1.2989 & 0.7701 & 16.9685 & 368.1 & 0.0043 & 9.0 & $[5-10]$ &  &  &  &  & \\
            \cline{1-8}
            Mejor & 1.3535 & 0.7388  & 16.9898 & 284 & 0.0041 & 9 & $[5-10]$ & 40 & 15 & 2 & 10 & 7\\
            \cline{1-8}
            Peor & 1.248 & 0.8013  & 17.4557 & 244 & 0.0042 & 9 & $[5-10]$ &  &  &  &  & \\
        \hline
        \hline
            Promedio  & 1.2985 & 0.7706 & 16.9554 & 420.1667 & 0.0046 & 9.0 & $[5-10]$ &  &  &  &  & \\
            \cline{1-8}
            Mejor & 1.3533 & 0.7389  & 17.2768 & 244 & 0.0043 & 9 & $[5-10]$ & 40 & 12 & 1 & 12 & 15\\
            \cline{1-8}
            Peor & 1.2125 & 0.8247  & 16.8693 & 284 & 0.0042 & 9 & $[5-10]$ &  &  &  &  & \\
        \hline
        \hline
            Promedio  & 1.3047 & 0.7668 & 16.9281 & 352.7667 & 0.0053 & 9.0 & $[5-10]$ &  &  &  &  & \\
            \cline{1-8}
            Mejor & 1.3512 & 0.7401  & 16.8872 & 284 & 0.0042 & 9 & $[5-10]$ & 40 & 10 & 1 & 15 & 7\\
            \cline{1-8}
            Peor & 1.2522 & 0.7986  & 17.3715 & 204 & 0.004 & 9 & $[5-10]$ &  &  &  &  & \\
        \hline
        \hline
            Promedio  & 1.3078 & 0.7649 & 16.9308 & 432.0333 & 0.0043 & 9.0 & $[5-10]$ &  &  &  &  & \\
            \cline{1-8}
            Mejor & 1.3496 & 0.741  & 16.5407 & 204 & 0.004 & 9 & $[5-10]$ & 40 & 13 & 1 & 12 & 5\\
            \cline{1-8}
            Peor & 1.2416 & 0.8054  & 17.6686 & 364 & 0.0041 & 9 & $[5-10]$ &  &  &  &  & \\
        \hline
        \hline
            Promedio  & 1.3053 & 0.7664 & 16.9986 & 378.6667 & 0.0043 & 9.0 & $[5-10]$ &  &  &  &  & \\
            \cline{1-8}
            Mejor & 1.3494 & 0.741  & 16.931 & 364 & 0.0041 & 9 & $[5-10]$ & 40 & 13 & 1 & 15 & 7\\
            \cline{1-8}
            Peor & 1.2482 & 0.8011  & 17.1707 & 404 & 0.0061 & 9 & $[5-10]$ &  &  &  &  & \\
        \hline
        \hline
            Promedio  & 1.3049 & 0.7665 & 16.9161 & 295.1667 & 0.0045 & 9.0 & $[5-10]$ &  &  &  &  & \\
            \cline{1-8}
            Mejor & 1.3469 & 0.7424  & 17.0975 & 244 & 0.0042 & 9 & $[5-10]$ & 30 & 15 & 4 & 10 & 1\\
            \cline{1-8}
            Peor & 1.259 & 0.7943  & 17.0465 & 274 & 0.0043 & 9 & $[5-10]$ &  &  &  &  & \\
        \hline
        \hline
            Promedio  & 1.3007 & 0.769 & 16.9119 & 354.3 & 0.0046 & 9.0 & $[5-10]$ &  &  &  &  & \\
            \cline{1-8}
            Mejor & 1.3444 & 0.7438  & 16.6935 & 459 & 0.0042 & 9 & $[5-10]$ & 35 & 11 & 4 & 12 & 4\\
            \cline{1-8}
            Peor & 1.2584 & 0.7946  & 17.6382 & 214 & 0.0041 & 9 & $[5-10]$ &  &  &  &  & \\
        \hline
        \hline
            Promedio  & 1.3077 & 0.7649 & 16.9693 & 337.7 & 0.0043 & 9.0 & $[5-10]$ &  &  &  &  & \\
            \cline{1-8}
            Mejor & 1.3423 & 0.745  & 16.7189 & 319 & 0.0041 & 9 & $[5-10]$ & 35 & 13 & 1 & 11 & 2\\
            \cline{1-8}
            Peor & 1.266 & 0.7899  & 16.9588 & 179 & 0.0042 & 9 & $[5-10]$ &  &  &  &  & \\
        \hline
        \hline
            Promedio  & 1.2939 & 0.7732 & 16.9569 & 309.7 & 0.0044 & 9.0 & $[5-10]$ &  &  &  &  & \\
            \cline{1-8}
            Mejor & 1.3417 & 0.7453  & 16.5588 & 424 & 0.0044 & 9 & $[5-10]$ & 35 & 9 & 1 & 7 & 13\\
            \cline{1-8}
            Peor & 1.2458 & 0.8027  & 16.8434 & 284 & 0.0041 & 9 & $[5-10]$ &  &  &  &  & \\
        \hline
        \hline
            Promedio  & 1.302 & 0.7683 & 16.9919 & 385.5 & 0.0042 & 9.0 & $[5-10]$ &  &  &  &  & \\
            \cline{1-8}
            Mejor & 1.3396 & 0.7465  & 16.6402 & 529 & 0.0044 & 9 & $[5-10]$ & 35 & 15 & 4 & 14 & 5\\
            \cline{1-8}
            Peor & 1.2292 & 0.8136  & 17.34 & 179 & 0.0041 & 9 & $[5-10]$ &  &  &  &  & \\
        \hline
        \end{tabular}
        \caption{Resultados de las mejores corridas de \emph{BEE} hibridado para {\bf Lenna}}
        \label{tb:tablebeehibimg}
    \end{center}
\end{table}


\begin{table}[h!]
    \footnotesize
    \begin{center}
        \begin{tabular}{|c|c|c|c|c|c|c|c|c|c|c|c|c|}
        \hline
            & {\bf FO} & {\bf DB} & $J_e$ & {\bf E} & {\bf T} & {\bf KE} & {\bf KO} & $I$ & $m$ & $e$ & $eb$ & $ob$ \\
        \hline
        \hline
            Promedio  & 1.293 & 0.7737 & 16.9901 & 321.3333 & 0.0042 & 9.0 & $[5-10]$ &  &  &  &  & \\
            \cline{1-8}
            Mejor & 1.3364 & 0.7483  & 16.7379 & 354 & 0.0047 & 9 & $[5-10]$ & 35 & 9 & 1 & 15 & 1\\
            \cline{1-8}
            Peor & 1.2243 & 0.8168  & 16.2465 & 249 & 0.004 & 9 & $[5-10]$ &  &  &  &  & \\
        \hline
        \hline
            Promedio  & 1.3044 & 0.7668 & 17.0028 & 377.5667 & 0.0046 & 9.0 & $[5-10]$ &  &  &  &  & \\
            \cline{1-8}
            Mejor & 1.3333 & 0.75  & 16.6586 & 284 & 0.0042 & 9 & $[5-10]$ & 40 & 15 & 1 & 12 & 12\\
            \cline{1-8}
            Peor & 1.2515 & 0.799  & 16.7045 & 250 & 0.0123 & 9 & $[5-10]$ &  &  &  &  & \\
        \hline
        \hline
            Promedio  & 1.2853 & 0.7784 & 16.8993 & 268.7667 & 0.0052 & 9.0 & $[5-10]$ &  &  &  &  & \\
            \cline{1-8}
            Mejor & 1.3317 & 0.7509  & 16.6688 & 379 & 0.0042 & 9 & $[5-10]$ & 25 & 11 & 3 & 8 & 9\\
            \cline{1-8}
            Peor & 1.198 & 0.8347  & 17.3952 & 154 & 0.004 & 9 & $[5-10]$ &  &  &  &  & \\
        \hline
        \hline
            Promedio  & 1.3016 & 0.7685 & 16.8765 & 293.3 & 0.0046 & 9.0 & $[5-10]$ &  &  &  &  & \\
            \cline{1-8}
            Mejor & 1.3313 & 0.7511  & 16.9553 & 424 & 0.0042 & 9 & $[5-10]$ & 30 & 12 & 2 & 6 & 8\\
            \cline{1-8}
            Peor & 1.2438 & 0.804  & 17.2068 & 305 & 0.0056 & 9 & $[5-10]$ &  &  &  &  & \\
        \hline
        \hline
            Promedio  & 1.3021 & 0.7681 & 16.933 & 426.7333 & 0.0043 & 9.0 & $[5-10]$ &  &  &  &  & \\
            \cline{1-8}
            Mejor & 1.3264 & 0.7539  & 16.6326 & 484 & 0.0042 & 9 & $[5-10]$ & 40 & 7 & 1 & 8 & 4\\
            \cline{1-8}
            Peor & 1.2654 & 0.7902  & 16.9853 & 284 & 0.0042 & 9 & $[5-10]$ &  &  &  &  & \\
        \hline
        \end{tabular}
        \caption{Continuacion resultados de las mejores corridas de \emph{BEE} hibridado para {\bf Lenna}}
        \label{tb:tablebeehibimg}
    \end{center}
\end{table}


    \begin{table}[h!]
    \footnotesize
    \begin{center}
        \begin{tabular}{|c|c|c|c|c|c|c|c|c|c|c|}
        \hline
            & {\bf FO} & {\bf DB} & $J_e$ & {\bf E} & {\bf T} & $I$ & $m$ & $e$ & $eb$ & $ob$ \\
        \hline
        \hline
            Promedio  & 1.9167 & 0.5412 & 0.6281 & 185.0 & 0.0021 &  &  &  &  & \\
            \cline{1-6}
            Mejor & 2.6521 & 0.3771  & 0.5029 & 240 & 0.0022 & 30 & 15 & 12 & 6 & 9\\
            \cline{1-6}
            Peor & 1.6451 & 0.6079  & 0.6287 & 150 & 0.0027 &  &  &  &  & \\
        \hline
        \hline
            Promedio  & 1.9573 & 0.5311 & 0.6093 & 304.6154 & 0.0035 &  &  &  &  & \\
            \cline{1-6}
            Mejor & 2.6521 & 0.3771  & 0.5029 & 400 & 0.0036 & 40 & 14 & 13 & 9 & 10\\
            \cline{1-6}
            Peor & 1.6899 & 0.5918  & 0.6785 & 240 & 0.0024 &  &  &  &  & \\
        \hline
        \hline
            Promedio  & 1.9649 & 0.526 & 0.6033 & 175.0 & 0.002 &  &  &  &  & \\
            \cline{1-6}
            Mejor & 2.6521 & 0.3771  & 0.5029 & 150 & 0.0013 & 25 & 7 & 5 & 5 & 13\\
            \cline{1-6}
            Peor & 1.6622 & 0.6016  & 0.6343 & 150 & 0.0034 &  &  &  &  & \\
        \hline
        \hline
            Promedio  & 1.9093 & 0.5443 & 0.609 & 205.0 & 0.0022 &  &  &  &  & \\
            \cline{1-6}
            Mejor & 2.6521 & 0.3771  & 0.5029 & 300 & 0.0026 & 30 & 8 & 7 & 6 & 15\\
            \cline{1-6}
            Peor & 1.5828 & 0.6318  & 0.636 & 150 & 0.0013 &  &  &  &  & \\
        \hline
        \hline
            Promedio  & 2.1315 & 0.4808 & 0.5773 & 221.0 & 0.0021 &  &  &  &  & \\
            \cline{1-6}
            Mejor & 2.6521 & 0.3771  & 0.5029 & 300 & 0.0024 & 30 & 8 & 7 & 7 & 14\\
            \cline{1-6}
            Peor & 1.7091 & 0.5851  & 0.6867 & 210 & 0.0016 &  &  &  &  & \\
        \hline
        \hline
            Promedio  & 1.8418 & 0.5536 & 0.6193 & 212.0 & 0.0026 &  &  &  &  & \\
            \cline{1-6}
            Mejor & 2.4013 & 0.4164  & 0.5038 & 330 & 0.0048 & 30 & 14 & 12 & 7 & 10\\
            \cline{1-6}
            Peor & 1.6659 & 0.6003  & 0.6986 & 210 & 0.0038 &  &  &  &  & \\
        \hline
        \hline
            Promedio  & 2.0037 & 0.5075 & 0.5791 & 248.5 & 0.003 &  &  &  &  & \\
            \cline{1-6}
            Mejor & 2.2421 & 0.446  & 0.5258 & 175 & 0.0033 & 35 & 13 & 12 & 11 & 9\\
            \cline{1-6}
            Peor & 1.6317 & 0.6129  & 0.6328 & 175 & 0.0025 &  &  &  &  & \\
        \hline
        \hline
            Promedio  & 1.9874 & 0.5118 & 0.5897 & 282.6667 & 0.0035 &  &  &  &  & \\
            \cline{1-6}
            Mejor & 2.2421 & 0.446  & 0.5258 & 200 & 0.002 & 40 & 12 & 10 & 6 & 7\\
            \cline{1-6}
            Peor & 1.6281 & 0.6142  & 0.6306 & 240 & 0.0026 &  &  &  &  & \\
        \hline
        \hline
            Promedio  & 1.9874 & 0.5118 & 0.5897 & 282.6667 & 0.0035 &  &  &  &  & \\
            \cline{1-6}
            Mejor & 2.2421 & 0.446  & 0.5258 & 200 & 0.0045 & 40 & 12 & 10 & 5 & 8\\
            \cline{1-6}
            Peor & 1.6281 & 0.6142  & 0.6306 & 240 & 0.0025 &  &  &  &  & \\
        \hline
        \hline
            Promedio  & 1.8118 & 0.5617 & 0.606 & 193.2609 & 0.0023 &  &  &  &  & \\
            \cline{1-6}
            Mejor & 2.2421 & 0.446  & 0.5258 & 175 & 0.0017 & 35 & 15 & 13 & 9 & 15\\
            \cline{1-6}
            Peor & 1.6281 & 0.6142  & 0.6306 & 175 & 0.0015 &  &  &  &  & \\
        \hline
        \hline
            Promedio  & 1.7754 & 0.572 & 0.6332 & 205.3333 & 0.0022 &  &  &  &  & \\
            \cline{1-6}
            Mejor & 2.2421 & 0.446  & 0.5258 & 200 & 0.0018 & 40 & 14 & 13 & 5 & 13\\
            \cline{1-6}
            Peor & 1.5921 & 0.6281  & 0.6722 & 200 & 0.0017 &  &  &  &  & \\
        \hline
        \hline
            Promedio  & 1.7632 & 0.5711 & 0.6611 & 274.6667 & 0.0036 &  &  &  &  & \\
            \cline{1-6}
            Mejor & 2.1696 & 0.4609  & 0.5564 & 360 & 0.0079 & 40 & 14 & 13 & 8 & 11\\
            \cline{1-6}
            Peor & 1.6899 & 0.5918  & 0.6785 & 240 & 0.0025 &  &  &  &  & \\
        \hline
        \hline
            Promedio  & 1.9222 & 0.5261 & 0.5997 & 273.0 & 0.0029 &  &  &  &  & \\
            \cline{1-6}
            Mejor & 2.1696 & 0.4609  & 0.5564 & 455 & 0.0043 & 35 & 13 & 11 & 7 & 14\\
            \cline{1-6}
            Peor & 1.6685 & 0.5993  & 0.6369 & 175 & 0.0018 &  &  &  &  & \\
        \hline
        \hline
            Promedio  & 1.9063 & 0.5309 & 0.6245 & 185.0 & 0.002 &  &  &  &  & \\
            \cline{1-6}
            Mejor & 2.1696 & 0.4609  & 0.5564 & 150 & 0.0033 & 30 & 11 & 9 & 5 & 11\\
            \cline{1-6}
            Peor & 1.6975 & 0.5891  & 0.6339 & 150 & 0.0012 &  &  &  &  & \\
        \hline
        \hline
            Promedio  & 1.9465 & 0.5209 & 0.6016 & 192.5 & 0.0018 &  &  &  &  & \\
            \cline{1-6}
            Mejor & 2.1696 & 0.4609  & 0.5564 & 175 & 0.0014 & 25 & 15 & 13 & 5 & 12\\
            \cline{1-6}
            Peor & 1.6685 & 0.5993  & 0.6369 & 150 & 0.0012 &  &  &  &  & \\
        \hline
        \end{tabular}
        \caption{Resultados de las mejores corridas de \emph{BEE} no hibridado para {\bf Iris}}
        \label{tb:tablebeealgcsv}
    \end{center}
\end{table}


\begin{table}[h!]
    \footnotesize
    \begin{center}
        \begin{tabular}{|c|c|c|c|c|c|c|c|c|c|c|}
        \hline
            & {\bf FO} & {\bf DB} & $J_e$ & {\bf E} & {\bf T} & $I$ & $m$ & $e$ & $eb$ & $ob$ \\
        \hline
        \hline
            Promedio  & 1.8508 & 0.5461 & 0.6097 & 212.0 & 0.0027 &  &  &  &  & \\
            \cline{1-6}
            Mejor & 2.124 & 0.4708  & 0.5685 & 270 & 0.0049 & 30 & 12 & 9 & 8 & 10\\
            \cline{1-6}
            Peor & 1.6622 & 0.6016  & 0.6343 & 210 & 0.0025 &  &  &  &  & \\
        \hline
        \hline
            Promedio  & 1.8631 & 0.5428 & 0.6204 & 185.0 & 0.0022 &  &  &  &  & \\
            \cline{1-6}
            Mejor & 2.124 & 0.4708  & 0.5685 & 300 & 0.0035 & 30 & 10 & 9 & 15 & 8\\
            \cline{1-6}
            Peor & 1.6622 & 0.6016  & 0.6343 & 150 & 0.003 &  &  &  &  & \\
        \hline
        \hline
            Promedio  & 1.8292 & 0.5546 & 0.626 & 225.1667 & 0.0026 &  &  &  &  & \\
            \cline{1-6}
            Mejor & 2.124 & 0.4708  & 0.5685 & 280 & 0.0027 & 35 & 15 & 10 & 5 & 10\\
            \cline{1-6}
            Peor & 1.6281 & 0.6142  & 0.6306 & 210 & 0.0046 &  &  &  &  & \\
        \hline
        \hline
            Promedio  & 1.9735 & 0.5126 & 0.5916 & 262.9412 & 0.0023 &  &  &  &  & \\
            \cline{1-6}
            Mejor & 2.124 & 0.4708  & 0.5685 & 330 & 0.0025 & 30 & 14 & 13 & 5 & 14\\
            \cline{1-6}
            Peor & 1.6975 & 0.5891  & 0.6339 & 240 & 0.0018 &  &  &  &  & \\
        \hline
        \hline
            Promedio  & 1.6589 & 0.603 & 0.7113 & 175.0 & 0.0023 &  &  &  &  & \\
            \cline{1-6}
            Mejor & 1.6899 & 0.5918  & 0.6785 & 175 & 0.004 & 35 & 13 & 12 & 6 & 10\\
            \cline{1-6}
            Peor & 1.6099 & 0.6211  & 0.7712 & 175 & 0.004 &  &  &  &  & \\
        \hline
        \end{tabular}
        \caption{Continuacion resultados de las mejores corridas de \emph{BEE} no hibridado para {\bf Iris}}
        \label{tb:tablebeealgcsv}
    \end{center}
\end{table}

    
    \begin{table}[h!]
    \footnotesize
    \begin{center}
        \begin{tabular}{|c|c|c|c|c|c|c|c|c|c|c|c|c|}
        \hline
            & {\bf FO} & {\bf DB} & $J_e$ & {\bf E} & {\bf T} & {\bf KE} & {\bf KO} & $I$ & $m$ & $e$ & $eb$ & $ob$ \\
        \hline
        \hline
            Promedio  & 1.9167 & 0.5412 & 0.6281 & 189.0 & 0.0 & 3.0 & 3 &  &  &  &  & \\
            \cline{1-8}
            Mejor & 2.6521 & 0.3771  & 0.5029 & 244 & 0.0 & 3 & 3 & 30 & 15 & 12 & 6 & 9\\
            \cline{1-8}
            Peor & 1.6451 & 0.6079  & 0.6287 & 154 & 0.0001 & 3 & 3 &  &  &  &  & \\
        \hline
        \hline
            Promedio  & 1.9573 & 0.5311 & 0.6093 & 308.6154 & 0.0 & 3.0 & 3 &  &  &  &  & \\
            \cline{1-8}
            Mejor & 2.6521 & 0.3771  & 0.5029 & 404 & 0.0 & 3 & 3 & 40 & 14 & 13 & 9 & 10\\
            \cline{1-8}
            Peor & 1.6899 & 0.5918  & 0.6785 & 244 & 0.0 & 3 & 3 &  &  &  &  & \\
        \hline
        \hline
            Promedio  & 1.9649 & 0.526 & 0.6033 & 179.0 & 0.0 & 3.0 & 3 &  &  &  &  & \\
            \cline{1-8}
            Mejor & 2.6521 & 0.3771  & 0.5029 & 154 & 0.0 & 3 & 3 & 25 & 7 & 5 & 5 & 13\\
            \cline{1-8}
            Peor & 1.6622 & 0.6016  & 0.6343 & 154 & 0.0001 & 3 & 3 &  &  &  &  & \\
        \hline
        \hline
            Promedio  & 1.9093 & 0.5443 & 0.609 & 209.0 & 0.0 & 3.0 & 3 &  &  &  &  & \\
            \cline{1-8}
            Mejor & 2.6521 & 0.3771  & 0.5029 & 304 & 0.0 & 3 & 3 & 30 & 8 & 7 & 6 & 15\\
            \cline{1-8}
            Peor & 1.5828 & 0.6318  & 0.636 & 154 & 0.0 & 3 & 3 &  &  &  &  & \\
        \hline
        \hline
            Promedio  & 2.1315 & 0.4808 & 0.5773 & 225.0 & 0.0 & 3.0 & 3 &  &  &  &  & \\
            \cline{1-8}
            Mejor & 2.6521 & 0.3771  & 0.5029 & 304 & 0.0 & 3 & 3 & 30 & 8 & 7 & 7 & 14\\
            \cline{1-8}
            Peor & 1.7091 & 0.5851  & 0.6867 & 214 & 0.0 & 3 & 3 &  &  &  &  & \\
        \hline
        \hline
            Promedio  & 1.8418 & 0.5536 & 0.6193 & 216.0 & 0.0 & 3.0 & 3 &  &  &  &  & \\
            \cline{1-8}
            Mejor & 2.4013 & 0.4164  & 0.5038 & 334 & 0.0 & 3 & 3 & 30 & 14 & 12 & 7 & 10\\
            \cline{1-8}
            Peor & 1.6659 & 0.6003  & 0.6986 & 214 & 0.0001 & 3 & 3 &  &  &  &  & \\
        \hline
        \hline
            Promedio  & 2.0037 & 0.5075 & 0.5791 & 252.5 & 0.0 & 3.0 & 3 &  &  &  &  & \\
            \cline{1-8}
            Mejor & 2.2421 & 0.446  & 0.5258 & 179 & 0.0001 & 3 & 3 & 35 & 13 & 12 & 11 & 9\\
            \cline{1-8}
            Peor & 1.6317 & 0.6129  & 0.6328 & 179 & 0.0 & 3 & 3 &  &  &  &  & \\
        \hline
        \hline
            Promedio  & 1.9874 & 0.5118 & 0.5897 & 286.6667 & 0.0 & 3.0 & 3 &  &  &  &  & \\
            \cline{1-8}
            Mejor & 2.2421 & 0.446  & 0.5258 & 204 & 0.0 & 3 & 3 & 40 & 12 & 10 & 6 & 7\\
            \cline{1-8}
            Peor & 1.6281 & 0.6142  & 0.6306 & 244 & 0.0 & 3 & 3 &  &  &  &  & \\
        \hline
        \hline
            Promedio  & 1.9874 & 0.5118 & 0.5897 & 286.6667 & 0.0 & 3.0 & 3 &  &  &  &  & \\
            \cline{1-8}
            Mejor & 2.2421 & 0.446  & 0.5258 & 204 & 0.0001 & 3 & 3 & 40 & 12 & 10 & 5 & 8\\
            \cline{1-8}
            Peor & 1.6281 & 0.6142  & 0.6306 & 244 & 0.0 & 3 & 3 &  &  &  &  & \\
        \hline
        \hline
            Promedio  & 1.8118 & 0.5617 & 0.606 & 197.2609 & 0.0 & 3.0 & 3 &  &  &  &  & \\
            \cline{1-8}
            Mejor & 2.2421 & 0.446  & 0.5258 & 179 & 0.0 & 3 & 3 & 35 & 15 & 13 & 9 & 15\\
            \cline{1-8}
            Peor & 1.6281 & 0.6142  & 0.6306 & 179 & 0.0 & 3 & 3 &  &  &  &  & \\
        \hline
        \hline
            Promedio  & 1.7754 & 0.572 & 0.6332 & 209.3333 & 0.0 & 3.0 & 3 &  &  &  &  & \\
            \cline{1-8}
            Mejor & 2.2421 & 0.446  & 0.5258 & 204 & 0.0 & 3 & 3 & 40 & 14 & 13 & 5 & 13\\
            \cline{1-8}
            Peor & 1.5921 & 0.6281  & 0.6722 & 204 & 0.0 & 3 & 3 &  &  &  &  & \\
        \hline
        \hline
            Promedio  & 1.7632 & 0.5711 & 0.6611 & 278.6667 & 0.0 & 3.0 & 3 &  &  &  &  & \\
            \cline{1-8}
            Mejor & 2.1696 & 0.4609  & 0.5564 & 364 & 0.0 & 3 & 3 & 40 & 14 & 13 & 8 & 11\\
            \cline{1-8}
            Peor & 1.6899 & 0.5918  & 0.6785 & 244 & 0.0 & 3 & 3 &  &  &  &  & \\
        \hline
        \hline
            Promedio  & 1.9222 & 0.5261 & 0.5997 & 277.0 & 0.0 & 3.0 & 3 &  &  &  &  & \\
            \cline{1-8}
            Mejor & 2.1696 & 0.4609  & 0.5564 & 459 & 0.0 & 3 & 3 & 35 & 13 & 11 & 7 & 14\\
            \cline{1-8}
            Peor & 1.6685 & 0.5993  & 0.6369 & 179 & 0.0001 & 3 & 3 &  &  &  &  & \\
        \hline
        \hline
            Promedio  & 1.9063 & 0.5309 & 0.6245 & 189.0 & 0.0 & 3.0 & 3 &  &  &  &  & \\
            \cline{1-8}
            Mejor & 2.1696 & 0.4609  & 0.5564 & 154 & 0.0001 & 3 & 3 & 30 & 11 & 9 & 5 & 11\\
            \cline{1-8}
            Peor & 1.6975 & 0.5891  & 0.6339 & 154 & 0.0 & 3 & 3 &  &  &  &  & \\
        \hline
        \hline
            Promedio  & 1.9465 & 0.5209 & 0.6016 & 196.5 & 0.0 & 3.0 & 3 &  &  &  &  & \\
            \cline{1-8}
            Mejor & 2.1696 & 0.4609  & 0.5564 & 179 & 0.0 & 3 & 3 & 25 & 15 & 13 & 5 & 12\\
            \cline{1-8}
            Peor & 1.6685 & 0.5993  & 0.6369 & 154 & 0.0 & 3 & 3 &  &  &  &  & \\
        \hline
        \end{tabular}
        \caption{Resultados de las mejores corridas de \emph{BEE} hibridado para {\bf Iris}}
        \label{tb:tablebeehibcsv}
    \end{center}
\end{table}


\begin{table}[h!]
    \footnotesize
    \begin{center}
        \begin{tabular}{|c|c|c|c|c|c|c|c|c|c|c|c|c|}
        \hline
            & {\bf FO} & {\bf DB} & $J_e$ & {\bf E} & {\bf T} & {\bf KE} & {\bf KO} & $I$ & $m$ & $e$ & $eb$ & $ob$ \\
        \hline
        \hline
            Promedio  & 1.8508 & 0.5461 & 0.6097 & 216.0 & 0.0 & 3.0 & 3 &  &  &  &  & \\
            \cline{1-8}
            Mejor & 2.124 & 0.4708  & 0.5685 & 274 & 0.0001 & 3 & 3 & 30 & 12 & 9 & 8 & 10\\
            \cline{1-8}
            Peor & 1.6622 & 0.6016  & 0.6343 & 214 & 0.0 & 3 & 3 &  &  &  &  & \\
        \hline
        \hline
            Promedio  & 1.8631 & 0.5428 & 0.6204 & 189.0 & 0.0 & 3.0 & 3 &  &  &  &  & \\
            \cline{1-8}
            Mejor & 2.124 & 0.4708  & 0.5685 & 304 & 0.0 & 3 & 3 & 30 & 10 & 9 & 15 & 8\\
            \cline{1-8}
            Peor & 1.6622 & 0.6016  & 0.6343 & 154 & 0.0001 & 3 & 3 &  &  &  &  & \\
        \hline
        \hline
            Promedio  & 1.8292 & 0.5546 & 0.626 & 229.1667 & 0.0 & 3.0 & 3 &  &  &  &  & \\
            \cline{1-8}
            Mejor & 2.124 & 0.4708  & 0.5685 & 284 & 0.0 & 3 & 3 & 35 & 15 & 10 & 5 & 10\\
            \cline{1-8}
            Peor & 1.6281 & 0.6142  & 0.6306 & 214 & 0.0001 & 3 & 3 &  &  &  &  & \\
        \hline
        \hline
            Promedio  & 1.9735 & 0.5126 & 0.5916 & 266.9412 & 0.0 & 3.0 & 3 &  &  &  &  & \\
            \cline{1-8}
            Mejor & 2.124 & 0.4708  & 0.5685 & 334 & 0.0 & 3 & 3 & 30 & 14 & 13 & 5 & 14\\
            \cline{1-8}
            Peor & 1.6975 & 0.5891  & 0.6339 & 244 & 0.0 & 3 & 3 &  &  &  &  & \\
        \hline
        \hline
            Promedio  & 1.6589 & 0.603 & 0.7113 & 179.0 & 0.0 & 3.0 & 3 &  &  &  &  & \\
            \cline{1-8}
            Mejor & 1.6899 & 0.5918  & 0.6785 & 179 & 0.0001 & 3 & 3 & 35 & 13 & 12 & 6 & 10\\
            \cline{1-8}
            Peor & 1.6099 & 0.6211  & 0.7712 & 179 & 0.0001 & 3 & 3 &  &  &  &  & \\
        \hline
        \end{tabular}
        \caption{Continuacion resultados de las mejores corridas de \emph{BEE} hibridado para {\bf Iris}}
        \label{tb:tablebeehibcsv}
    \end{center}
\end{table}


\section{Algoritmo de Hormiga (AntH)}\label{sect:aant}

    La variable del \emph{AntH} (\ref{sect:ihormiga}) es la siguiente:
    \begin{itemize}
        \item $I$: tamaño de la población. Se varió su valor en el rango
    $[5, 10, \cdots, 40]$.
    \end{itemize}

    \begin{table}[h!]
    \footnotesize
    \begin{center}
        \begin{tabular}{|c|c|c|c|c|c|c|}
        \hline
            & {\bf FO} & {\bf DB} & $J_e$ & {\bf E} & {\bf T} & $I$\\
        \hline
        \hline
            Promedio  & 0.0369 & 5137878816.15 & 79529.3282 & - & 8.2478 & \\
            \cline{1-6}
            Mejor & 0.2296 & 4.3554  & 78916.2656 & - & 7.595 & 40\\
            \cline{1-6}
            Peor & 0.0 & 10165431296.0  & 67521.75 & - & 7.4013 & \\
        \hline
        \hline
            Promedio  & 0.0177 & 6163481029.14 & 81483.2442 & - & 8.2185 & \\
            \cline{1-6}
            Mejor & 0.2032 & 4.9218  & 67647.9297 & - & 8.1248 & 35\\
            \cline{1-6}
            Peor & 0.0 & 594540.6875  & 57427.1602 & - & 7.6854 & \\
        \hline
        \hline
            Promedio  & 0.0272 & 5829294829.59 & 78284.9675 & - & 8.0481 & \\
            \cline{1-6}
            Mejor & 0.1921 & 5.2055  & 76798.5156 & - & 8.2154 & 15\\
            \cline{1-6}
            Peor & 0.0 & 8783278080.0  & 60623.9688 & - & 7.9738 & \\
        \hline
        \hline
            Promedio  & 0.0365 & 3890123585.01 & 67665.2609 & - & 8.1726 & \\
            \cline{1-6}
            Mejor & 0.1731 & 5.7767  & 61577.2227 & - & 8.3185 & 10\\
            \cline{1-6}
            Peor & 0.0 & 19866507264.0  & 114472.1328 & - & 9.6955 & \\
        \hline
        \hline
            Promedio  & 0.0269 & 6920094984.7 & 82721.1441 & - & 8.1835 & \\
            \cline{1-6}
            Mejor & 0.1466 & 6.8212  & 37402.8984 & - & 8.3085 & 30\\
            \cline{1-6}
            Peor & 0.0 & 6620387328.0  & 38828.0977 & - & 8.5141 & \\
        \hline
        \hline
            Promedio  & 0.0191 & 5677045971.71 & 71329.1938 & - & 8.1133 & \\
            \cline{1-6}
            Mejor & 0.139 & 7.1946  & 54899.0664 & - & 8.6996 & 25\\
            \cline{1-6}
            Peor & 0.0 & 17461166080.0  & 112302.1641 & - & 6.8254 & \\
        \hline
        \hline
            Promedio  & 0.0239 & 5583854095.58 & 65941.7116 & - & 8.3475 & \\
            \cline{1-6}
            Mejor & 0.1232 & 8.1151  & 59408.5508 & - & 7.3055 & 5\\
            \cline{1-6}
            Peor & 0.0 & 607603456.0  & 61214.207 & - & 8.1664 & \\
        \hline
        \hline
            Promedio  & 0.0098 & 7345290500.01 & 85573.7683 & - & 8.1365 & \\
            \cline{1-6}
            Mejor & 0.1155 & 8.6557  & 80215.7812 & - & 6.465 & 20\\
            \cline{1-6}
            Peor & 0.0 & 11462755.0  & 68440.4531 & - & 7.887 & \\
        \hline
        \end{tabular}
        \caption{Resultados de las mejores corridas de \emph{ANT} no hibridado para {\bf Lenna}}
        \label{tb:tableantalgimg}
    \end{center}
\end{table}

    
    \begin{table}[h!]
    \footnotesize
    \begin{center}
        \begin{tabular}{|c|c|c|c|c|c|c|c|c|}
        \hline
            & {\bf FO} & {\bf DB} & $J_e$ & {\bf E} & {\bf T} & {\bf KE} & {\bf KO} & $I$\\
        \hline
        \hline
            Promedio  & 1.0581 & 0.9483 & 11.4482 & - & 0.0233 & 25.4667 & $[5-10]$ & \\
            \cline{1-8}
            Mejor & 1.2015 & 0.8323  & 11.2246 & - & 0.0204 & 25 & $[5-10]$ & 40\\
            \cline{1-8}
            Peor & 0.9367 & 1.0676  & 9.8267 & - & 0.0261 & 35 & $[5-10]$ & \\
        \hline
        \hline
            Promedio  & 1.0588 & 0.946 & 11.4965 & - & 0.0213 & 25.0 & $[5-10]$ & \\
            \cline{1-8}
            Mejor & 1.1937 & 0.8377  & 12.438 & - & 0.017 & 19 & $[5-10]$ & 35\\
            \cline{1-8}
            Peor & 0.9868 & 1.0133  & 11.7159 & - & 0.0233 & 25 & $[5-10]$ & \\
        \hline
        \hline
            Promedio  & 1.0611 & 0.9439 & 10.8732 & - & 0.0304 & 28.3667 & $[5-10]$ & \\
            \cline{1-8}
            Mejor & 1.1581 & 0.8635  & 11.9112 & - & 0.0202 & 23 & $[5-10]$ & 10\\
            \cline{1-8}
            Peor & 0.986 & 1.0142  & 8.4525 & - & 0.0429 & 45 & $[5-10]$ & \\
        \hline
        \hline
            Promedio  & 1.0538 & 0.9506 & 11.4248 & - & 0.0233 & 25.5 & $[5-10]$ & \\
            \cline{1-8}
            Mejor & 1.1573 & 0.8641  & 11.9637 & - & 0.0197 & 20 & $[5-10]$ & 20\\
            \cline{1-8}
            Peor & 0.9594 & 1.0423  & 11.4198 & - & 0.0247 & 27 & $[5-10]$ & \\
        \hline
        \hline
            Promedio  & 1.0594 & 0.9453 & 11.4921 & - & 0.0212 & 25.0667 & $[5-10]$ & \\
            \cline{1-8}
            Mejor & 1.1534 & 0.867  & 11.5418 & - & 0.0209 & 23 & $[5-10]$ & 30\\
            \cline{1-8}
            Peor & 0.9894 & 1.0108  & 11.6893 & - & 0.0188 & 25 & $[5-10]$ & \\
        \hline
        \hline
            Promedio  & 1.0563 & 0.9486 & 11.1834 & - & 0.0235 & 26.7333 & $[5-10]$ & \\
            \cline{1-8}
            Mejor & 1.1402 & 0.8771  & 10.15 & - & 0.0187 & 29 & $[5-10]$ & 25\\
            \cline{1-8}
            Peor & 0.9481 & 1.0547  & 11.3961 & - & 0.0232 & 25 & $[5-10]$ & \\
        \hline
        \hline
            Promedio  & 1.0432 & 0.9601 & 11.1676 & - & 0.0268 & 26.9667 & $[5-10]$ & \\
            \cline{1-8}
            Mejor & 1.1238 & 0.8899  & 11.0716 & - & 0.0232 & 25 & $[5-10]$ & 15\\
            \cline{1-8}
            Peor & 0.9681 & 1.033  & 10.015 & - & 0.0248 & 32 & $[5-10]$ & \\
        \hline
        \hline
            Promedio  & 1.0494 & 0.9539 & 10.9997 & - & 0.0263 & 28.0 & $[5-10]$ & \\
            \cline{1-8}
            Mejor & 1.1015 & 0.9079  & 11.1517 & - & 0.0241 & 26 & $[5-10]$ & 5\\
            \cline{1-8}
            Peor & 0.9638 & 1.0376  & 11.0072 & - & 0.023 & 27 & $[5-10]$ & \\
        \hline
        \end{tabular}
        \caption{Resultados de las mejores corridas de \emph{ANT} hibridado para {\bf Lenna}}
        \label{tb:tableanthibimg}
    \end{center}
\end{table}


    \begin{table}[h!]
    \footnotesize
    \begin{center}
        \begin{tabular}{|c|c|c|c|c|c|c|}
        \hline
            & {\bf FO} & {\bf DB} & $J_e$ & {\bf E} & {\bf T} & $I$\\
        \hline
        \hline
            Promedio  & 0.0173 & 6007677.0002 & 50.5294 & - & 0.063 & \\
            \cline{1-6}
            Mejor & 0.5183 & 1.9294  & 12.7227 & - & 0.0565 & 40\\
            \cline{1-6}
            Peor & 0.0 & 85281.7109  & 13.3658 & - & 0.0578 & \\
        \hline
        \hline
            Promedio  & 0.112 & 1645758.6818 & 21.9977 & - & 0.0703 & \\
            \cline{1-6}
            Mejor & 0.4301 & 2.3252  & 55.5032 & - & 0.0603 & 20\\
            \cline{1-6}
            Peor & 0.0 & 10690933.0  & 58.9907 & - & 0.0701 & \\
        \hline
        \hline
            Promedio  & 0.0273 & 3485332.9297 & 37.0707 & - & 0.0691 & \\
            \cline{1-6}
            Mejor & 0.4093 & 2.4431  & 17.0787 & - & 0.0652 & 25\\
            \cline{1-6}
            Peor & 0.0 & 11930849.0  & 74.264 & - & 0.0563 & \\
        \hline
        \hline
            Promedio  & 0.0945 & 712117.542 & 33.8483 & - & 0.0758 & \\
            \cline{1-6}
            Mejor & 0.3652 & 2.7386  & 1.0657 & - & 0.0889 & 10\\
            \cline{1-6}
            Peor & 0.0 & 60951.3555  & 17.5941 & - & 0.0699 & \\
        \hline
        \hline
            Promedio  & 0.0981 & 5997581.3398 & 49.1423 & - & 0.0745 & \\
            \cline{1-6}
            Mejor & 0.3619 & 2.7635  & 37.8509 & - & 0.0731 & 5\\
            \cline{1-6}
            Peor & 0.0 & 63175.1211  & 36.2951 & - & 0.0892 & \\
        \hline
        \hline
            Promedio  & 0.0328 & 4690957.3412 & 40.6344 & - & 0.0677 & \\
            \cline{1-6}
            Mejor & 0.3332 & 3.001  & 59.4798 & - & 0.0645 & 30\\
            \cline{1-6}
            Peor & 0.0 & 96485.3906  & 1.1537 & - & 0.069 & \\
        \hline
        \hline
            Promedio  & 0.0552 & 1381683.815 & 31.6493 & - & 0.0722 & \\
            \cline{1-6}
            Mejor & 0.307 & 3.2576  & 30.7605 & - & 0.0682 & 15\\
            \cline{1-6}
            Peor & 0.0 & 1852458.75  & 9.5635 & - & 0.0731 & \\
        \hline
        \hline
            Promedio  & 0.023 & 2752012.1906 & 26.378 & - & 0.0657 & \\
            \cline{1-6}
            Mejor & 0.2986 & 3.3491  & 1.099 & - & 0.0611 & 35\\
            \cline{1-6}
            Peor & 0.0 & 8590559.0  & 43.6928 & - & 0.0691 & \\
        \hline
        \end{tabular}
        \caption{Resultados de las mejores corridas de \emph{ANT} no hibridado para {\bf Iris}}
        \label{tb:tableantalgcsv}
    \end{center}
\end{table}

    
    \begin{table}[h!]
    \footnotesize
    \begin{center}
        \begin{tabular}{|c|c|c|c|c|c|c|c|c|}
        \hline
            & {\bf FO} & {\bf DB} & $J_e$ & {\bf E} & {\bf T} & {\bf KE} & {\bf KO} & $I$\\
        \hline
        \hline
            Promedio  & 1.053 & 0.9733 & 0.4211 & - & 0.0002 & 8.2667 & 3 & \\
            \cline{1-8}
            Mejor & 1.6586 & 0.6029  & 0.4881 & - & 0.0001 & 4 & 3 & 15\\
            \cline{1-8}
            Peor & 0.8298 & 1.2051  & 0.4864 & - & 0.0001 & 7 & 3 & \\
        \hline
        \hline
            Promedio  & 1.1289 & 0.9107 & 0.4435 & - & 0.0001 & 7.5 & 3 & \\
            \cline{1-8}
            Mejor & 1.6451 & 0.6079  & 0.6287 & - & 0.0001 & 3 & 3 & 25\\
            \cline{1-8}
            Peor & 0.8992 & 1.1121  & 0.3579 & - & 0.0002 & 12 & 3 & \\
        \hline
        \hline
            Promedio  & 1.0306 & 0.9796 & 0.4008 & - & 0.0002 & 8.8667 & 3 & \\
            \cline{1-8}
            Mejor & 1.3772 & 0.7261  & 0.4725 & - & 0.0001 & 5 & 3 & 20\\
            \cline{1-8}
            Peor & 0.8788 & 1.1379  & 0.4416 & - & 0.0001 & 8 & 3 & \\
        \hline
        \hline
            Promedio  & 1.0541 & 0.9603 & 0.4408 & - & 0.0001 & 7.3667 & 3 & \\
            \cline{1-8}
            Mejor & 1.3475 & 0.7421  & 0.3947 & - & 0.0001 & 6 & 3 & 10\\
            \cline{1-8}
            Peor & 0.8921 & 1.121  & 0.5304 & - & 0.0001 & 6 & 3 & \\
        \hline
        \hline
            Promedio  & 1.089 & 0.9295 & 0.4512 & - & 0.0001 & 7.1 & 3 & \\
            \cline{1-8}
            Mejor & 1.3395 & 0.7465  & 0.475 & - & 0.0001 & 5 & 3 & 5\\
            \cline{1-8}
            Peor & 0.899 & 1.1123  & 0.3852 & - & 0.0001 & 7 & 3 & \\
        \hline
        \hline
            Promedio  & 1.035 & 0.9746 & 0.4293 & - & 0.0001 & 8.2667 & 3 & \\
            \cline{1-8}
            Mejor & 1.3306 & 0.7516  & 0.5726 & - & 0.0001 & 4 & 3 & 40\\
            \cline{1-8}
            Peor & 0.8829 & 1.1326  & 0.4008 & - & 0.0001 & 10 & 3 & \\
        \hline
        \hline
            Promedio  & 1.0734 & 0.9379 & 0.4255 & - & 0.0001 & 7.6667 & 3 & \\
            \cline{1-8}
            Mejor & 1.2451 & 0.8032  & 0.4703 & - & 0.0001 & 5 & 3 & 30\\
            \cline{1-8}
            Peor & 0.883 & 1.1325  & 0.4397 & - & 0.0001 & 7 & 3 & \\
        \hline
        \hline
            Promedio  & 1.0195 & 0.9925 & 0.3922 & - & 0.0001 & 9.1667 & 3 & \\
            \cline{1-8}
            Mejor & 1.1735 & 0.8522  & 0.3559 & - & 0.0002 & 10 & 3 & 35\\
            \cline{1-8}
            Peor & 0.7897 & 1.2663  & 0.3965 & - & 0.0001 & 11 & 3 & \\
        \hline
        \end{tabular}
        \caption{Resultados de las mejores corridas de \emph{ANT} hibridado para {\bf Iris}}
        \label{tb:tableanthibcsv}
    \end{center}
\end{table}

