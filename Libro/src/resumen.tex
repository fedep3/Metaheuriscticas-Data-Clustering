\vspace{5 mm}

\setcounter{page}{4}
\begin{center}
	{\bf Resumen}
\end{center}

\vspace{5 mm}

La agrupación de datos (\emph{data clustering}) es el proceso de particionar una
colección de datos en conjuntos de clases significativas, llamadas \emph{clusters},
donde los objetos de una clase comparten ca\-rac\-te\-rís\-ti\-cas comunes. En este trabajo,
se presenta un estudio comparativo de cinco metaheurísticas basadas en población
e inteligencia colectiva (\emph{algoritmo genético}, \emph{optimizador de enjambre
de partículas}, \emph{evolución diferencial}, \emph{algoritmo de abeja} y
\emph{algoritmo de clustering de hormigas}) para resolver el problema de \emph{data
clustering} en calidad de soluciones finales. Los datos usados para el estudio son
de tipo numérico, comúnmente utilizados en la literatura afín. Del estudio se
llega a la conclusión que para resolver el problema de \emph{clustering} de datos
numéricos se deben utilizar el \emph{algoritmo genético} o el \emph{algoritmo de
abeja}, ambos hibridados con la heurística \emph{K-means} como método de mejoramiento
de las soluciones finales.

\newpage

