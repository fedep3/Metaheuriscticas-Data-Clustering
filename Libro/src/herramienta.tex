% Planteamiento del Problema
\chapter{Planteamiento del problema} \label{chap:planning}

%Puedes quitar esto(es opcional)
\vspace{5 mm}

Para comprobar que las metaheur\'isticas descritas anteriormente son 
efectivas para atacar el problema de Data Clustering, es necesarias implementarlas,
hacer bastantes pruebas con ellas e interpretar los resultados.

\section{Requerimientos} \label{sect:requirements}

Los requerimientos de la herramienta o software a implementar se mencionan a
continuaci\'on: 

\begin{itemize}
\item La implementaci'on de la soluci\'on 
debe ser de c\'odigo libre de este modo cualquier persona esta en la capacidad
de verificar la veracidad de lo dicho en este escrito.
\item Tener la posibilidad de leer diferente tipos de archivos: CSV, PNG y TIFF.
\item Poder limitar las corridas a un m\'aximo de 10 minutos del siguiente modo:
5 minutos corriendo el algoritmos y los 5 restantes optimizando el resultado
mediante un Kmeans.
\item La herramienta de l'inea de comando debe desarrollarse en el lenguaje
{\tt C} o {\tt C++}, ya que se quiere utilizar la librer'ia {\tt libtiff} para manipular
las im\'agenes TIFF y otra para manipular las PNG. Adem\'as, coviene
usar \'estos ya que compilan a c\'odigo de m\'aquina y se busca rapidez.
\item Una m\'etrica com\'un entre las metaheur\'isticas que se
implementen para poder hacer comparaciones.
\item Flexibilidad: esto incluye el hacer todos los aspectos
importantes de las metaheur\'isticas parametrizables.
\item Tiene que incluir al menos 5 metaheur\'isticas basadas en poblaci\'on.
\'Estas fueron escogidas tomando en cuanta los siguientes factores: cantidad de 
informaci\'on disponible, dificultad de implementaci\'on y resultados que han
reportado. A la final se eligieron las descritas en el \ref{chap:ssimilar}.
\item El Kmeans debe estar implementado por ser el algoritmos m\'as famoso
de este problema y es \'util como referencia a la hora de comparar.
\end{itemize}


