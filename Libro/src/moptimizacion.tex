\chapter{Métodos de Optimización} \label{chap:mopt}

El objetivo de la optimización es buscar valores para un conjunto de
parámetros que maximicen o minimicen la función objetivo del problema de modo
que se satisfagan las restricciones del mismo. Una solución factible es una
elección de valores para el conjunto de parámetros que satisfacen todas las
las restricciones del problema. Las soluciones factibles con mejores valores de
función objetivo son llamadas soluciones óptimas.

    Las técnicas de optimización son usadas diariamente para planificación
industrial, administración de recursos, programación de horarios, toma de
decisiones, agrupación no supervisada de datos, etc. Por lo tanto, las técnicas de
optimización son ampliamente usadas en muchos campos como ingeniería, industria,
medicina y negocios. La investigación en el campo de la optimización es bastante
activa y nuevos métodos de optimización son desarrollados regularmente \cite{GO_1}.

    La optimización abarca tanto problemas de maximización como de minimización.
Cualquier problema de maximización puede ser convertido en un problema de
minimización al tomar la inversa de la función objetivo del mismo y
viceversa. En general, un problema de optimización puede ser definido como:

    Sea $S$ el espacio de soluciones y $d$ la dimensión del mismo donde:
\begin{itemize}
    \item $f: S \rightarrow \mathbb{R}$ es la función objetivo de maximización
del problema.
    \item $S \subseteq \mathbb{R}^d$
\end{itemize}
    entonces se debe encontrar $x^* \in S$ tal que:
\begin{itemize}
    \item $f(x^*) \geq f(x), \forall x \in S$ si se desea una función maximizadora.
    \item $f(x^*)^{-1} \leq f(x)^{-1}, \forall x \in S$, si se desea una función
minimizadora.
\end{itemize}

    El valor de $f(x^*)$ (ó $f(x^*)^{-1}$) es llamado óptimo global. La
optimización global es la tarea de encontrar la solución global óptima. En
general, existen soluciones que son localmente óptimas pero no globalmente
óptimas. Por lo tanto los problemas de optimización global son difíciles de
resolver de manera exacta. En el contexto de problemas combinatorios, estos
problemas son frecuentemente NP-hard \cite{GO_2}. Un buen algoritmo de
optimización global encontrará $x^*$ sin importar el punto inicial $x_0 \in S$.

    Algunos ejemplos de problemas de optimización global son \cite{GO_2}:
\begin{itemize}
    \item Problemas combinatorios: donde la función lineal o no-lineal es definida
sobre un conjunto finito, pero bastante grande de soluciones, Por ejemplo,
agrupación no supervisada de datos, programación de horarios, entre otros.
    \item Problemas generales sin restricciones: donde se define una función
no-lineal sobre un conjunto de valores reales sin restricciones.
    \item Problemas generales con restricciones: donde se define una función
no-lineal sobre un conjunto de valores reales con restricciones.
\end{itemize}

\subsection{Algoritmos Tradicionales de Optimización}

    Los algoritmos de optimización tradicionales usan métodos exactos para
encontrar la mejor solución. Si el problema tiene solución, entonces estos algoritmos
son capaces de encontrar la solución óptima global.

    Algunos de estos algoritmos son \cite{TA_1}:
\begin{itemize}
    \item \emph{Fuerza bruta}: comparan todas las soluciones del espacio de
soluciones de modo que encontrar la solución global óptima esté garantizada. Sin
embargo, a medida que crece el espacio de soluciones el costo de los algoritmos
de fuerza bruta incrementa. Por lo tanto no son apropiados para la clase de
problemas pertenecientes al grupo NP-hard.
    \item Programación lineal: sirven para problemas donde una variable depende
de una o más variables de manera lineal:
\begin{center}
    $f(X_1, X_2, \cdots, X_n) = a_1 \cdot X_1 + a_2 \cdot X_2 + \cdots + a_n \cdot X_n$
\end{center}

    En otras palabres, $f(X_1, X_2, \cdots, X_n)$ (función objetivo del algoritmo)
puede ser expresada como una combinación lineal de las variables
$X_1, X_2, \cdots, X_n$. 
    \item Programación dinámica: funciona bajo el principio de encontrar una
solución total, operando en un punto intermedio que está entre la ubicación actual
y la ubicación final. El procedimiento es recursivo y cada punto intermedio es una
función de los puntos ya visitados. Una vez alcanzada la meta, se puede reconstruir
de manera inversa el camino óptimo generado por el algoritmo.
\end{itemize}

\subsection{Algoritmos Estocásticos}

    Los algoritmos estocásticos son usados para hallar soluciones
cercanas al óptimo global para problemas NP-hard en tiempo polinomial. Estos
algoritmos logran su objetivo al asumir que las soluciones buenas están
cercanas unas de otras en el espacio de búsqueda. Sin embargo, a diferencia de
los algoritmos tradicionales, estos podrían no encontrar la solución óptima
global \cite{PSO_0}.

    Las ventajas de los algoritmos estocásticos tienen varias
ventajas comparados con otros algoritmos \cite{SA_1}:
\begin{itemize}
    \item Son generalmente fáciles de implementar.
    \item Pueden ser usados eficientemente en paralelo.
    \item No requieren de la función de definición del problema para ser
continuos.
    \item Generalmente pueden encontrar la solución óptima o cercana a la
óptima.
    \item Sirven para problemas discretos y combinatorios.
\end{itemize}

    Algunos algoritmos estocásticos son \cite{PSO_0}:
\begin{itemize}
    \item Hill-Climbing: consiste en tomar una solución potencial aleatoria del
problema $x_0$ y buscar en la vecindad $N_0$ de $x_0$ una nueva mejor solución
al problema. Si se encentra un $x_1 \in N_0$ tal que $x_1$ es mejor que $x_0$,
entonces $x_1$ es la nueva solución potencial y se repite el procedimiento que
se hizo con $x_0$. Si no se encuentra una solución $x_1 \in N_0$ tal que
$x_1$ es mejor que $x_0$, entonces $x_0$ es la mejor solución encontrada por el
algoritmo \cite{TA_1}.
    \item Simulated Annealing: consiste en tomar una solución aleatoria del
problema $x_0$ y un valor pequeño $\epsilon$ para perturbar las soluciones.
Si $x_0 + \epsilon$ es mejor que $x_0$, entonces la nueva mejor solución es 
$x_0 + \epsilon$ y se repite el procedimiento. Sino, la solución $x_0$ será
perturbada con una probabilidad que se va decrementando a medida que avanza la
ejecución del algoritmo para seguir buscando \cite{SA_2}.
    \item Búsqueda Tabú: consiste en una búsqueda heurística que mantiene una
lista de memoria tabú de las soluciones previamente visitadas, de modo que se
mejore el proceso de búsqueda. La lista tabú es usada como una guía de
movimiento de una solución a la siguiente para evitar ciclos \cite{SA_5} y que 
se quede atrapado en un óptimo local. Inicialmente, Tabú comienza su búsqueda
con una solución elegida aleatoriamente. A partir de la solución actual se
generan un conjunto de soluciones de prueba. La mejor solución de prueba se
coloca como solución actual si no está en la lista tabú o, si está en la lista
tabú, pero satisface el criterio de aspiración. Una solución satisface un
criterio de aspiración si está en la lista tabú y es la mejor solución
encontrada hasta el momento. Este proceso se repite hasta que el criterio de
parada se satisfaga \cite{SA_3}\cite{SA_4}.
\end{itemize}
    
\subsection{Algoritmos Evolutivos (EA)}
    Los algoritmos evolutivos (\emph{EAs}) son métodos de búsqueda estocásticos de
propósito general que simulan la selección natural y la evolución de los
organismos biológicos. Los \emph{EAs} difieren de otros métodos de optimización,
como Hill-Climbing y Simulated Annealing, en que los \emph{EAs} mantienen una
población de soluciones potenciales a un problema y no sólo una solución.

    En general, todos los EAs funcionan de la siguiente manera:
\begin{itemize}
    \item Se inicializa una población de individuos (soluciones potenciales al
problema).
    \item La calidad de cada solución está dada por la función de
\emph{fitness}, que no es más que la función objetivo del algoritmo.
    \item En cada iteración, se aplica un proceso de selección para formar a la
población de la siguiente iteración. Este proceso está orientado a elegir
individuos que tengan una buen valor de \emph{fitness}.
    \item Los individuos son alterados a través de dos procesos: una
transformación unaria (mutación) y una transformación de orden superior (cruce o
crossover) \cite{PSO_0}.
    \item Se espera que la mejor solución encontrada esté cerca del óptimo.
\end{itemize}

    Se llaman operadores evolutivos a las transformaciones unaria y de orden
superior. Los operadores evolutivos más comunes son \cite{PSO_0}:
\begin{itemize}
    \item \textbf{Selección}: es un operador que elige a uno o más individuos para
aplicarles los otros operadores evolutivos. Las estrategías varían desde elegir
a los mejores individuos de la población hasta elegir a los mejores individuos
de un conjunto elegido aleatoriamente.
    \item \textbf{Mutación}: es un operador que modifica a un individuo a través
de un pequeño cambio para generar un nuevo individuo. El objetivo principal de
la mutación es introducir diversidad ``genética'' para evitar quedar atascado en
un óptimo local.
    \item \textbf{Recombinación o Cruce}: es un operador que combina a dos o más
individuos para generar nuevos individuos. El principal objetivo del cruce es
explorar nuevas áreas en el espacio de búsqueda.
    
\end{itemize}

\begin{lstlisting}[float=h, caption=Algoritmo Evolutivo General]
- Inicializar a cada individuo de la población.

- Evaluar el %\emph{fitness}% de cada individuo de la población.

- Mientras no se satisfaga el criterio de parada:

    - Aplicar el proceso de %\textbf{selección}%.

    - Alterar los individuos usando los operadores de %\textbf{cruce}% y
    %\textbf{mutación}%.

    - Evaluar el %\emph{fitness}% de cada individuo de la población.

- FinMientras.

\end{lstlisting}

    Las cuatro técnicas evolutivas más relevantes son \cite{PSO_0}:
\begin{itemize}
    \item La \textbf{programación genética} que es usada para buscar el
programa más eficiente para resolver un problema específico.
    \item La \textbf{programación evolutiva} que es usada generalmente para
optimizar funciones reales continuas. No utilizan el operador de recombinación.
    \item Las \textbf{estrategias evolutivas} que son usadas para optimizar
funciones reales continuas. Usa los operadores de selección, cruce y mutación.
No solo optimiza a la población, sino también al proceso de optimización al
evolucionar los parámetros de estrategia.
    \item Los \textbf{algoritmos genéticos} que son generalmente usados para
optimizar problemas combinatorios.
\end{itemize}

    Los algoritmos evolutivos pueden evitar quedar atrapados en un óptimo local
al tener varias soluciones potenciales al mismo tiempo y pueden encontrar
frecuentemente las soluciones óptimas globales. Sin embargo, existe cierto
riesgo de que no converjan a la solución global óptima \cite{SA_4}.

\subsection{Algoritmo Genético (GA)}

Imita la evoluci\'on gen\'etica de las especias. Un aspecto importante es que
explora varias soluciones a la vez, y no se enfoca en s\'olo una. A medida que
se ejecuta, las soluciones, tambi\'en llamadas individuos o cromosomas, 
toman parte en un proceso reporoductivo, donde interactuan, mezclan y producen
hijos que retiene algunas caracter\'isticas de sus padres. \'Este proceso
que conlleva a la creaci\'on de nuevas soluciones (hijos) esta basada en
la selecci\'on, cruce y mutaci\'on\cite{DoGeGr2007}.

Los cromosomas se pueden codificarse de bastantes modos, donde el m\'as 
com\'un es mediante strings de n\'umeros binarios (100101110101). El modo en
que se codifiquen tamb\'en depende del problela.


El pseudoc\'odigo es de la siguiente manera (hay m\'as interpretaciones posibles):

\begin{lstlisting}[float=h, caption=Algoritmo General Genetico]
    - Inicialización.
    Mientras no se cumpla el criterio de parada.
        Si rand(0,1) <= pc
            - Selección.
            - Cruce.
            Si rand(0,1) <= pm
                - Mutación.
            FinSi
            Si los hijos tiene mejor función objetico que los papas.
                - Colocarlos en la población y eliminar a los padres.
            FinSi
        FinSi
    FinMientras
\end{lstlisting}

Donde pc es la probabilidad de cruce, pm la de mutaci\'on y rand(0,1)
es una funci\'on que devuelve un n\'umero aleatorio entre 0 y 1. Y el resto
se define como :

\begin{itemize}

\item {\bf Inicializaci\'on:} Los par\'ametros son fijados y se
crean los cromosomas de la poblaci\'on.

\item {\bf Selecci\'on:} Consiste en elegir los padres. 

En \cite{GePo2010} comentan sus formas de implementar m\'as comunes:

\begin{itemize}

\item Tomar m\'as cuenta los individuos
con mejor funci\'on objetvo. Para ello se puede usar el m\'etodo bastante conocido
llamado roulette-wheel. 

\item Selecci\'on por torneo, en
donde un conjunto de cromosomas $\tau$ son elegidos y comparados, los mejores
son elegidos para ser padres. 

\end{itemize}

\item {\bf Cruce:}

Consiste en replazar los genes de uno de los padres por los del otro. Se puede hacer
de la siguiente manera continuando \cite{GePo2010}:

\begin{itemize}

\item {\bf De un punto:} Si se tienen dos padres de longitud $X$, se va a tomar
un punto entre $[1,X-1]$ y se generan dos hijos de la siguiente manera: el primero,
de 0 al punto con los genes del primer padre y del punto a $X$, y el segundo es 
lo contrario.

Si se tienen los padres [1 3 2 4 6]  y [5 1 3 6 2] y el punto 
de crude es el 2, entonces se generan los siguientes hijos:

[1 3 3 6 2] y [5 1 2 4 6]

\item {\bf De dos punto:} Ac\'a se eligen dos puntos distintos que cumplan con la 
misma condici\'on como en el de un punto. Para explicar es mucho mas sencillo
mediante un ejemplo:

Si se tienen los padres [1 3 2 4 6]  y [5 1 3 6 2] y los puntos 
de crude son el 2 y 4, entonces se generan los siguientes hijos:

[1 3 3 6 6] y [5 1 2 6 2]

De esta misma forma se puede extender para m\'as puntos.

\end{itemize}

\item {\bf Mutaci\'on:}

Consiste en modificar uno de los cromosomas bien sea de la poblaci\'on
o uno de los hijos generados. Un caso bastante com\'un es el de usar una m\'ascara
cuando los cromosomas son binarios\cite{GePo2010}.

Por su lado en esa misma referencia se comenta que para clustering se pueden usar los siguientes:

\begin{itemize}

\item En el caso que se use la representaci\'on binaria, alternar los bits es un
buen operador.

\item Cuando la representaci\'on es de enteros de etiquedas, se puede cambiar
el cluster al cual pertenece uno de los objetos.

\item Remplazar uno de los centroides por uno de los objetos pertenecientes a ese
cluster.

\end{itemize}

\end{itemize}

\subsection{Optimizador de Enjambre de partículas (PSO)}

    Un optimizador de enjambre de partículas (\emph{PSO}: Particle Swarm Optimizer)
es un algoritmo poblacional de optimización estocástico modelado a partir de la
simulación del comportamiento social de las bandadas de aves \cite{PSO_1}
\cite{PSO_2}.

    En un sistema PSO, un enjambre (swarm) de individuos, llamados partículas, se
mueven a través de un espacio de búsqueda. Cada partícula representa una
solución candidata al problema de optimización. La posición de  cada partícula
es influenciada por la mejor posición visitada por ella (su propia experiencia)
y la posición de la mejor partícula en su vecindad (la experiencia conjunta de
todas las partículas de su vecindad). Cuando la vecindad de una partícula es
el enjambre completo, la mejor posición en su vecindad se refiere a la mejor
partícula global y el algoritmo resultante es el \emph{gbest PSO} (Global Best
PSO). Por otra parte, cuando se utilizan vecindades más pequeñas, el algoritmo
generalmente se refiere al \emph{lbest PSO} (Local Best PSO) \cite{PSO_0}.

    Las principales diferencias entre el \emph{PSO} y los \emph{EAs} son
\cite{PSO_0}:
\begin{itemize}
    \item El \emph{PSO} se ve influenciado más en la experiencia social que la
supervivencia del más apto, como ocurren el los \emph{EAs}.
    \item En el \emph{PSO}, cada individuo se ve beneficiado de su historia. Esto
no ocurre en los \emph{EAs}.
\end{itemize}

\subsubsection{Características de las partículas}

    En el \emph{PSO}, cada partícula tiene una serie de características
\cite{PSO_0}:
\begin{itemize}
    \item $x_i$: La posición actual de la partícula $i$. Dependiendo del
problema, la posición de cada partícula tendrá una o más dimensiones.
    \item $v_i$: La velocidad actual de la partícula $i$. La dimensión de la
velocidad dependerá de la dimensión de la posición de la partícula.
    \item $y_i$: La mejor posición de la partícula $i$. La elección de $y_i$, en
la iteración $t + 1$ del algoritmo, es de la siguiente forma:
\begin{center}
    \[
      y_i(t+1) =
      \begin{cases}
        y_i(t)   & \text{si } f(x_i(t+1)) \geq f(y_i(t)) \\
        x_i(t+1) & \text{si } f(x_i(t+1)) < f(y_i(t))
      \end{cases}
    \]
\end{center}
donde $f$ es una función de $fitness$ y el problema de optimización es de
minimización.
    \item $\hat{y}_i$: La mejor posición de una partícula en la vecindad de la
partícula $i$. La elección de $\hat{y}_i$ para la iteración $t + 1$ es similar a
la de $y_i$:
\begin{center}
    $\hat{y}_i(t + 1) = \displaystyle\min_{j \in N_k} y_j(t)$
\end{center}
donde $N_k$ es el vecindaria $k$, la partícula $i$ pertenece a $N_k$ y el
problema de optimización es de minimización.

    Si sólo existe una vecindad y todas las partículas están en él
(\emph{gbest PSO}), entonces se calcula la mejor posición de una partícula en el
enjambre de partículas $\hat{y}$:
\begin{center}
    $\hat{y}(t + 1) = \displaystyle\min_{j = 1}^{P} y_j(t)$
\end{center}
donde $P$ es la cantidad total de partículas.
\end{itemize}

\subsubsection{Actualización de la Velocidad y la Posición}

    Para cada dimensión $j \in {1, \cdots, M}$, donde $M$ es el número de
dimensiones, la velocidad de la partícula $i$ en la iteración $t + 1$ del
algoritmo es \cite{PSO_0}:
\begin{center}
$v_{i,j}(t + 1) = w \cdot v_{i,j}(t) + c_1 \cdot r_{1,j}(t) \cdot (y_{i,j}(t) - x_{i,j}(t)) + c_2 \cdot r_{2,j}(t) \cdot (\hat{y}_j(t) - x_{i,j}(t))$
\end{center}
donde,
\begin{itemize}
    \item $w$: es una constante que acompaña al \emph{peso inercial}, que sirve
como memoria de velocidades previas: $v_{i,j}(t)$. Una alta inercia favorece la
exploración y una baja inercia, la intensificación.
    \item $c_1$: Es una \emph{constante de aceleración} que acompaña a la
\emph{componente cognitiva} representa la experiencia de la mejor solución de
la partícula: $y_{i,j} (t) - x_{i,j}(t)$. Es decir, cuanto se aleja de la mejor
solución personal.
    \item $c_2$: Es una \emph{constante de aceleración} que acompaña a la
\emph{componente social}, que representa cuan alejada está la solución personal
de la mejor obtenida por el enjambre: $\hat{y}_j(t)-x_{i,j}(t)$
    \item $r_{1,j}(t)$ y $r_{2,j}(t)$: Son valores aleatorios que están en el
intérvalo (0, 1).
\end{itemize}

    Según Van den Bergh\cite{PSO_3}, la relación entre el \emph{peso inercial} y
las \emph{constantes de aceleración} debe satisfacer la siguiente ecuación de modo que
se garantice convergencia:
\begin{center}
$\displaystyle\frac{c_1 + c_2}{2} - 1 < w$
\end{center}
sino el \emph{PSO} mostraría ser divergente o un comportamiento acíclico.

    Así que la actualización de la posición para la partícula $i$ para la
iteración $t + 1$ del algoritmo es de la siguiente forma \cite{PSO_0}:
\begin{center}
$x_i(t + 1) = x_i(t) + v_i(t + 1)$
\end{center}

\subsubsection{Algoritmo General}

    Finalmente, el algoritmo general del \emph{PSO} es como sigue\cite{PSO_0}:
\begin{lstlisting}[float=h, caption=Algoritmo General PSO]
- Para cada partícula $i \in  \{1, \cdots, P\}$:
    - Inicializar $x_i$.
    //Puede ser inicializado con velocidad cero.
    - Inicializar $v_i$.
    - $y_i = x_i$.

- Mientras no se cumpla la condición de parada:

    - Para cada partícula $i \in \{1, \cdots, P\}$:
        - Evaluar el %\emph{fitness}% de la partícula $i$, $f(x_i)$.
        - Actualizar $y_i$.

        - Si es %\emph{gbest PSO}%:
            - Actualizar $\hat{y}$.
        - Sino:
            - Actualizar $\hat{y}_i$.

        - Para cada dimensión $j \in \{1, \cdots, M\}$:
            - Actualizar velocidad $v_{i,j}$.

        - Actualizar posición $x_i$.
\end{lstlisting}

\subsection{Evolución Diferencial (DE)}

Es un algoritmo basado en poblaci\'on de
optimizaci\'on global que hace uso de una representaci\'on de punto flotante
(codificaci\'on real), bastante parecido al gen\'etico. Se tienen los pasos de cruce 
y selecci\'on, pero no mutaci\'on. Su funcionamiento seg\'un \cite{SwAjAm2008}
es el siguiente:

El $i$-\'esimo vector individual (cromosoma)
de la poblaci\'on con tiempo (generaci\'on) $t$ tiene $d$ componentes
dimensiones:

%Ejemplo de vector.
%INICIO
\begin{center}
$ \overrightarrow{Z_i}(t) = [ Z_{i,1}(t), Z_{i,2}, \cdots, Z_{i,d}(t) ] $
\end{center}
%END

Para cada vector individual $\overrightarrow{Z_k}(t)$ que pertenece
a la poblaci\'on actual, el DE aleatoriamente toma tres individuos
$\overrightarrow{Z_i}(t)$, $\overrightarrow{Z_j}(t)$ y $\overrightarrow{Z_m}(t)$ de la misma generaci\'on (de modo que sean distintos $k$, 
$i$, $j$ y $m$). Entonces calcula la diferencia entre $\overrightarrow{Z_i}(t)$ y $\overrightarrow{Z_j}(t)$, lo escala por un escalar $F$
(usualmente $F \in [0, 1]$) y crea un hijo prueba $\overrightarrow{U_i}(t + 1)$ a\~nadiendo el resultado a $\overrightarrow{Z_m}(t)$. De modo
que para la $n$-\'esima componente del vector:

\[
  U_{k,n}(t+1) =
  \begin{cases}
    Z_{m,n}(t) + F(Z_{i,n}(t) - Z_{j,n}(t))  & \text{si } rand_n(0,1) < Cr\\
    Z_{k,n}(t)                               & \text{sino}
  \end{cases}
\]

Donde $Cr \in [0, 1]$ es un escalar que es par\'ametro de el algoritmo,
llamado la \emph{tasa de cruce}. Si el nuevo hijo tiene mejor valor
con la funci\'on objetivo, entonces reemplaza al padre en la siguiente
generaci\'on, sino, el padre entonces se queda en la misma:
\[
  \overrightarrow{Z_i}(t+1) =
  \begin{cases}
    \overrightarrow{U_i}(t+1) & \text{si } f(\overrightarrow{U_i}(t+1)) > f(\overrightarrow{Z_i}(t)) \\
    \overrightarrow{Z_i}(t)   & \text{si } f(\overrightarrow{U_i}(t+1)) \leq f(\overrightarrow{Z_i}(t))
  \end{cases}
\]
donde $f$ es la funci\'on objetivo a ser maximizada.

\subsection{Algoritmo de Abeja (Bee)}

    El algoritmo de Abeja (\emph{Bee}) es un algoritmo poblacional de
optimización estocástico modelado a partir de la simulación del comportamiento
social y habilidades de búsqueda de las colonias de abejas mieleras
\cite{BEE_0}.

\subsubsection{Abejas en la Naturaleza}

    Una colonia de abejas mieleras pueden extenderse largas distancias
de modo que se exploten grandes números de fuentes de comida al mismo
tiempo. Este proceso comienza en una colonia con abejas exploradoras
que son enviadas a los sitios de flores más prometedores. Los sitios
prometedores son aquellos con grandes cantidades de néctar o polen
que pueden ser recolectados con menos esfuerzo. Estos sitios tienden a ser
visitados por más  abejas, mientras que los sitios con menor calidad tienden
a ser menos visitados \cite{BEE_0}.

    Durante la época de cosecha, una colonia continúa su exploración
manteniendo un porcentaje de la población como abejas exploradoras.
Cuando retornan a la colmena, esas abejas exploradoras depositan el
polen o néctar y van al \emph{piso de la danza} a ejecutar una danza
conocida como \emph{danza oscilante}. Esta danza misteriosa es esencial
para la comunicación en la colonia y contiene tres piezas de información
de cada uno de los sitios visitados \cite{BEE_0}: 
\begin{itemize}
    \item La dirección hacia donde está el sitio.
    \item La distancia desde la colonia.
    \item Su calidad (\emph{fitness}).
\end{itemize}
    Esta información, ayuda a la colonia a enviar sus abejas a los sitios de
flores de manera más precisa, sin usar guías o mapas \cite{BEE_0}.

    Después de la \emph{danza oscilante}, las abejas exploradoras van a los
sitios que encontraron con abejas que estaban esperando dentro de la colmena.
Más seguidoras son enviadas a los mejores sitios. Esto le permite a la colonia
recolectar comida más rápido y eficientemente \cite{BEE_0}.

    Mientras recolectan comida de un sitio, las abejas monitorean su nivel de
comida. Esto es necesario para decidir cuando ocurrirá la siguiente
\emph{danza oscilante}\cite{BEE_0}.

\subsubsection{Abejas en el Algoritmo}

    En el \emph{Bee}, cada abeja tiene una solución al problema de optimización
(sitio de flores) y la función de \emph{fitness} (calidad del sitio) asociada a
ella \cite{BEE_0}.

    El algoritmo depende de los siguientes parámetros \cite{BEE_0}:
\begin{itemize}
    \item $m$: Cantidad de sitios de flores disponibles en cada iteración.
Cada sitio de flores es una solución factible al problema de optimización y
cada abeja tiene la dirección hacia un sitio de flores.
    \item $e$: Cantidad de sitios de flores élite. Son los $e$ mejores sitios de
flores encontrados por las abejas.
    \item $eb$: Cantidad de abejas élite. Esta será la cantidad de abejas que
serán enviadas a los mejores sitios de flores.
    \item $ob$: Cantidad de abejas exploradoras. Estas abejas será asignadas a
buscar nuevos sitios de flores (nuevas soluciones).
    \item $N$: Cantidad de abejas de la colmena.
\end{itemize}

    El algoritmo entonces consiste en emular lo que ocurre en la naturaleza al
eligir los $e$ mejores sitios de flores y asignar a cada abeja a un sitio
dependiendo de su calidad. Luego, todas las abejas harían una búsqueda en
vecindad para encontrar nuevas soluciones. El proceso se repite una y otra vez
hasta que la condición de parada se cumpla \cite{BEE_0}.

    Así que el algoritmo general de Abeja sería \cite{BEE_0}:
\begin{lstlisting}[float=h, caption=Algoritmo General de Abeja]
- Para cada abeja $i \in  \{1, \cdots, N\}$:
    - Inicializar los sitios de flores de cada abeja.
    - Evaluar función de %\emph{fitness}% de cada abeja.

- Mientras no se cumpla la condición de parada:
    - Seleccionar los mejores sitios. //%\color{darkgrey}\emph{Danza Oscilante}%
    - Asignar $e$ abejas a los mejores sitios.
    - Asignar $m - e$ abejas sitios aleatorios.

    - Para cada abeja $i \in \{1, \cdots, N\}$:
        - Hacer búsqueda en la vencindad del sitio asignado.
        - Evaluar el %\emph{fitness}% de cada nuevo sitio.
\end{lstlisting}

\subsection{Algoritmo de Hormiga (Ant)}

Hormiga es una metaheurística usada para resolver problemas combinatorios
difíciles. \'Esta se inspira en los diversos comportamientos de las hormigas.
Comunmente se basa en el rastro de feromona y el comportamiento de seguirlo 
de éstas, usado en la búsqueda de comida. Una hormiga que se est\'a moviendo 
suelta feromonas en el suelo, as\'i marcando un camino. \'Este químico, que 
desaparece con el tiempo, el reforzado si otras hormigas usan ese mismo camino.
Por lo tanto, las mejores v\'ias incrementan su nivel de feromonas con el tiempo, 
y al contrario con los peores. Fue propuesto por Marco Dorigo en 1992, 
y lo ha expandidos en sus trabajos posteriores.
\cite{GePo2010} \cite{Le2007}

Su funcionamiento como explica \cite{GePo2010} es el siguiente: primero, $m$ hormigas contruyen
soluciones del problema, sesgada por la informaci\'on de las feromonas y posiblemente
por las disponible por parte de las heur\'isticas. Una vez que las hormigas hallan completado
sus soluciones, se pueden mejorar mediante una b\'usqueda local. Finalmente,
antes de empezar con la siguiente iteraci\'on , los ratros de feromonas son
actualizados para reflejar la experiencia de b\'usqueda de las hormigas:

\begin{lstlisting}[float=h, caption=Algoritmo General Hormiga]
    - Incialización.
    Mientras no se cumpla el criterio de parada.
        - Construcción de las soluciones.
        - Aplicar búsqueda local.
        - Actualizar feromonas.
    FinMientras
\end{lstlisting}

\begin{itemize}

\item {\bf Inicializaci\'on:} Los par\'ametros son establecidos y todas las variables 
de feromonas son puestas en t$_{0}$, el cual es un par\'ametro del algoritmo.

\item {\bf Construcci\'on de las soluciones:} Cada hormiga empieza con una soluci\'on vac\'ia
$s_p = \emptyset $. En cada paso de las construcci\'on , una hormiga
extiende su soluci\'on parcial actual $s_p$ eligiendo un posible componente $c_i^j \in N(s_p) \subseteq C$ 
y agregandolo a \'esta. $N(s_p)$ es el conjunto de los
componentes de soluci\'on que pueden ser agregados manteniendo su validez
y es definido implicitamente por el proceso de construcci\'on de soliciones
que las hormigas implementan. 

La elecci\'on del componente de la soluci\'on que se quiere agregar es 
hecha probabil\'isticamente en cada paso de la construcci\'on. La manera m\'as
com\'un es la siguiente:

\[
p(c_i^j) = {\tau_{ij}^{\alpha}  \times [ \eta (c_i^j) ]^{\beta}}  \over { \sum_{c_i^l \in N(s_p)} \tau_{il}^{\alpha} \times [\eta (c_i^l)]^{\beta}] } 
\]

donde $\eta(.)$ es una funci\'on  que asigna a cada posible componente 
de la soluci\'on $c_i^j \in N(s_p)$ un valor heur\'istico, que es usualmente
llamada infomaci\'on heur\'istica. Los par\'ametos $\alpha$ y $\beta$ 
determinan la relativa influencia de los rastros de feromonas e informaci\'on
heur\'istica, por lo tanto influyendo significativamente en el comportamiento
del algoritmo.


\item {\bf Aplicar b\'usqueda local:} Una vez obtenidas las soluciones candidatos,
estas puede ser mejoradas aplicando algoritmos de b\'usqueda local.

\item {\bf Actualizaci\'on de las feromonas:}: tiene como objetivo
hacer que los compentes de una buena soluci\'on, 
sean m\'as deseables para las hormigas en las siguientes iteraciones. Hay esencialemente
dos mecanismos para lograr este objetico. El primero es el dep\'osito de feromonas,
el cual incrementa el nivel de feromonas de los coponentes de una soluci\'on
que est\'an asociados con un conjunto selccionado $S_{upd}$ de buenas soluciones.
El segundo es la evaporaci\'on del rastro de feromonas, el cual es un mecanismo
que decrece a medida que el tiempo pasa el dep\'osito de feromonas. Desde el punto de
vista pr\'actico, \'este es necesario para evitar la r\'apida convergencia del algoritmo
en una regi\'on sub\'optima. Es comunmente implementada de la siguiente
manera:

\[
\tau_{ij} = (1-\rho)\tau_{ij} + \sum_{s \in S_{upd} \land c_i^j \in s} g(s)
\]

donde $S_{upd}$ es el conjunto de soluciones usadas para depositar feromonas, 
$\rho in (0,1]$ es un par\'ametro llamada constante de evaporaci\'on,
$g(.):S \rightarrow  \Re^+$ es una funci\'on que determina la calidad
de la soluci\'on.
\end{itemize}

Adem\'as en \cite{OuBa2007} exponen que actualmente los cient\'ificos est\'an 
empezando a inspirarse en la manera que las hormigas crean sus cementerios: limpian sus nidos
y crean pilas de cad\'averes. 
Tambi\'en de la organizaci\'on de cr\'ias, donde son agrupadas
de acuerdo a su tama\~no. El principio recae en la atracci\'on
entre los objetos transportados. Los clusters pequeños de objetos
similares van creaciendo atrayendo a las hormigas a despositar m\'as
objetos de acuerdo con su tama\~no o tipo. Este feedback positivo
conlleva a la formaci\'on de clusters homog\'enos.

El pioner en este trabajo es Deneubourg et al., donde aplican el
m\'etodo para tareas en rob\'otica. Este ha sido modificado por Lumer
y Faita para extenderlo a an\'alisis num\'erico de datos.
En estos algoritmos los datos son dispersados aleatoriamente
en un grid de dos dimensiones. Cada hormiga se mueve aleatoriamente
dentro de \'este agarrando y soltando estos datos. La decisi\'on
de agarrar o soltar un dato es aleatoria,
pero es influenciada por los datos en el vecindario, 
causando que datos similares tengan m\'as probabilidad de
estar juntos. La probabilidad de soltar un dato incrementa
en zonas de mayor densiadad de datos similares, y decrementa
cuando sucede el contrario.  En contraste
la probabilidad de agarrar un dato incrementa en las zonas de
menor densidad y decrementa en el opuesto.
\'Esta est\'an dadas por:

\[
P_p(i) = {k_1 \over {k_1 + f(i)}}
\]

\[
P_d(i) = 
  \begin{cases}
    2f(i) & \quad \text{si $f(i)<k_2$}\\
    1     & \quad \text{si $f(i) < k_2s$}\\
  \end{cases}
\]

\[
f(i) =
  \begin{cases}
    {{1} \over {s^2}} \sum_{j \in R(r(i))} {{1 - {d(i,j)}} \over \alpha} & \quad \text{si $f > 0$}\\
    0     & \quad \text{sino}
  \end{cases} 
\]

Donde $r(i)$ es la posici\'on del dato $i$ dado en el grid y $f(i)$ es una medida
del promedio de disimilaridad del dato $i$ con respecto a los otros $j$
presentes en su vencindario $R$ con tama\~no $s \times s$. $\alpha$
es la escala de disimilaridad y es clave en la ejecuci\'on del algoritmo:

\[
\alpha = {{1 \over {N(N-1)}} \sum_{i=1}^N \sum_{j=1}^N d(p_i,p_j)}
\]

El n\'umero de aplicaciones es bastante grande: resolver problemas
desde data clustering, programaci\'on (scheduling), balanceo
de l\'inea de equilibrio, TSP probabil\'istico, secuenciaci\'on de 
ADN, etc. \cite{GePo2010}
